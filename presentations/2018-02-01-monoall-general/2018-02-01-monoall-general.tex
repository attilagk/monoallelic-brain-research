\documentclass[usenames,dvipsnames]{beamer} %\documentclass[handout]{beamer}


%\includeonlyframes{bsm,title,toc,imprint-mouse-devel,igf2-imprint-evol,sister-disorders,previous-age-studies,our-study,cmc,toc-current,filtering-calling,fitting-models,ll-surface,all-betas,orthogonality,identifiability,anova,betas-cluster,signif-genes,cf-overall-expression,summary,improving-model,p-val,imprinted-brain,chess-lab}

% language settings
%\usepackage{fontspec, polyglossia}
%\setdefaultlanguage{magyar}

% common packages
\usepackage{amsmath, multimedia, hyperref, color, multirow}
%\usepackage{graphicx}

% TikZ
\usepackage{tikz}
%\usetikzlibrary{arrows.meta, decorations.pathmorphing, decorations.pathreplacing, shapes.geometric,mindmap}
%\usetikzlibrary{shapes.geometric,fadings,bayesnet}

% beamer styles
\mode<presentation>{
%\usetheme{Pittsburgh}
\usetheme{Singapore}
%\usecolortheme{dove}
\usecolortheme{seahorse}
%\usefonttheme{structureitalicserif}
\setbeamercovered{transparent}
}
\setbeamertemplate{blocks}[rounded][shadow=true]
\AtBeginSubsection[]{
  \begin{frame}<beamer>{Contents}
    %\tableofcontents[currentsection]
    \tableofcontents[currentsubsection]
  \end{frame}
}
%\useoutertheme[]{tree}

\title{Imprinting, schizophrenia, and aging}
\subtitle{investigated with mixed models and CommonMind data}
\author{Attila Guly\'{a}s-Kov\'{a}cs}
\date{Chess Lab}

\newcommand{\platefigscale}[0]{0.7}
\newcommand{\ownfigscale}[0]{0.4}

\begin{document}

\begin{frame}[plain, label=title]
\maketitle
\end{frame}

\begin{frame}{Contents}
\tableofcontents
\end{frame}

\section{Introduction}

\begin{frame}{Schizophrenia: complex genetic architecture}
\includegraphics[width=1.0\textwidth]{figures/from-others/sullivan-natrevgenet-2012-fig1b.jpg}
\end{frame}

\begin{frame}{Imprinting and allelic bias}
\begin{enumerate}
\item epigenetic mechanism
\item variation across age and tissue
\item biological function 
\end{enumerate}
\includegraphics[width=0.6\textwidth]{figures/from-others/renfree-2012-fig2.jpg}

{\tiny Renfree et al 2012 Philos Trans R Soc Lond B}

\end{frame}

\begin{frame}{My \only<1>{\emph{previous}} \only<2>{\emph{current}} presentation}
\begin{itemize}
%\item genomic imprinting and schizophrenia (SCZ)
\item<1> CommonMind Consortium data
\item<1> 30 imprinted genes in DLPFC, some novel
\item<1> dependence of allelic bias on SCZ? 
\item<1> weak support
\begin{itemize}
\item  differential dependence on SCZ, age
\end{itemize}
\item<2> stronger support
\begin{itemize}
\item  differential dependence on age, \alert{not} SCZ
\end{itemize}
\end{itemize}
\end{frame}

\section{Our previous work}

\begin{frame}{Read count ratio measures allelic expression bias}
\includegraphics[height=0.75\textheight]{figures/by-me/commonmind-rna-seq/commonmind-rna-seq.pdf}
\end{frame}

\begin{frame}{The data}
\footnotesize
\begin{tabular}{|c|ccc|cccc|}
 & \multicolumn{3}{|c|}{read count ratios} & \multicolumn{4}{|c|}{predictors} \\
\(i\) & \(Y_1\) & \(\hdots\) & \(Y_{30}\) & \(X_1\) & \(X_2\) & \(\hdots\) & \(X_{12}\) \\
\textbf{Individual} & \textbf{MAGEL2} & \(\hdots\) & \textbf{UBE3A} & \textbf{Dx} & \textbf{Age} & \(\hdots\) & \textbf{RIN} \\
\hline
1 & NA & \(\hdots\) & NA & AFF & 42 & \(\hdots\) & 6.90 \\
2 & 1.00 & \(\hdots\) & NA & AFF & 58 & \(\hdots\) & 7.00 \\
3 & NA & \(\hdots\) & 0.89 & AFF & 28 & \(\hdots\) & 6.90 \\
\(\vdots\) & \(\vdots\) & \(\vdots\) & \(\vdots\) & \(\vdots\) & \(\vdots\) & \(\vdots\) & \(\vdots\) \\
%574 & NA & \(\vdots\) & 0.90 & SCZ & 59 & \(\vdots\) & 8.60 \\
%575 & NA & \(\vdots\) & NA & Control & 38 & \(\vdots\) & 8.30 \\
576 & 0.97 & \(\hdots\) & NA & Control & 42 & \(\hdots\) & 8.50 \\
577 & 1.00 & \(\hdots\) & NA & SCZ & 27 & \(\hdots\) & 7.50 \\
578 & 1.00 & \(\hdots\) & NA & Control & 57 & \(\hdots\) & 8.60 \\
579 & 1.00 & \(\hdots\) & 1.00 & Control & 28 & \(\hdots\) & 7.70 \\
\end{tabular}
\end{frame}

\begin{frame}%[plain]
\begin{center}
\includegraphics[height=0.95\textheight]{figures/2016-11-01-plotting-distribution-of-s/S-Dx-strip-1.pdf}
\end{center}
\end{frame}

\begin{frame}%[plain]
\begin{center}
\includegraphics[height=0.95\textheight]{figures/2016-06-26-trellis-display-of-data/evar-scatterplot-matrix-simple-1.pdf}
\end{center}
\end{frame}

\begin{frame}[label=S-Age]
\begin{center}
\includegraphics[height=0.95\textheight]{figures/2016-11-01-plotting-distribution-of-s/S-Dx-age-1.pdf}
\end{center}
\end{frame}

\begin{frame}
\begin{center}
\includegraphics[height=0.95\textheight]{figures/2018-02-10-fixed-mixed-p-values/beta-99-CI-1.pdf}
\end{center}
\end{frame}

\begin{frame}<1,2>[label=pval]
\begin{center}
\includegraphics<1>[height=0.95\textheight]{figures/2018-02-10-fixed-mixed-p-values/p-val-dotplot-wnlmQ-1.pdf}
\includegraphics<2>[height=0.95\textheight]{figures/2018-02-10-fixed-mixed-p-values/p-val-dotplot-wnlmQ-unlmQ-1.pdf}
\includegraphics<3>[height=0.95\textheight]{figures/2018-02-10-fixed-mixed-p-values/p-val-dotplot-wnlmQ-mixed-1.pdf}
\end{center}
\end{frame}

\begin{frame}
\begin{center}
\includegraphics[height=0.95\textheight]{figures/2016-09-23-model-checking/res-std-dev-wnlmQ-unlmQ-FAM50B-MEST-1.pdf}
\end{center}
\end{frame}

\section{Mixed models}

\againframe{S-Age}

\begin{frame}%{The power of mixed models}

\begin{columns}[t]
\begin{column}{0.70\textwidth}

\includegraphics<1>[width=\columnwidth]{figures/2017-10-16-fixed-and-mixed-models/data-1.pdf}

\includegraphics<2>[width=\columnwidth]{figures/2017-10-16-fixed-and-mixed-models/data-F0-F1-1.pdf}

\includegraphics<3>[width=\columnwidth]{figures/2017-10-16-fixed-and-mixed-models/data-M0-M1-1.pdf}

\includegraphics<4>[width=\columnwidth]{figures/2017-10-16-fixed-and-mixed-models/data-M0-M1-M2-1.pdf}

\includegraphics<5>[width=\columnwidth]{figures/2017-10-16-fixed-and-mixed-models/data-F1-M2-1.pdf}

\includegraphics<6>[width=\columnwidth]{figures/2017-10-16-fixed-and-mixed-models/log-p-val-1.pdf}
\end{column}

\begin{column}{0.30\textwidth}

\footnotesize
\begin{tabular}{p{0.30\textwidth} p{0.30\textwidth} p{0.30\textwidth}}
\only<2,5,6>{fixed} &
\only<3,4,6>{mixed} &
\only<4,5,6>{mixed} \\
\only<2,5,6>{\color{Red}F1} &
\only<3,4,6>{\color{Green}M1} &
\only<4,5,6>{\color{Blue}M2} \\
\vspace{0pt}
\includegraphics<2,5,6>[scale=0.7]{figures/by-me/fixed-mixed/fixed/fixed.pdf} &
\vspace{0pt}
\includegraphics<3,4,6>[scale=0.7]{figures/by-me/fixed-mixed/mixed/mixed.pdf} &
\vspace{0pt}
\includegraphics<4,5,6>[scale=0.7]{figures/by-me/fixed-mixed/random/random.pdf} \\
\end{tabular}
\end{column}
\end{columns}
\end{frame}

\section{Our revised work}

\againframe<1,3>{pval}

\begin{frame}
\begin{columns}[t]
\begin{column}{0.5\textwidth}

\includegraphics[width=\columnwidth]{figures/2017-03-08-model-checking/qqplot-families-M3-1.pdf}
\end{column}

\begin{column}{0.5\textwidth}

\includegraphics[width=\columnwidth]{figures/2017-03-08-model-checking/scedasticity-families-M3-1.pdf}
\end{column}
\end{columns}
\end{frame}

\end{document}


\begin{columns}[t]
\begin{column}{0.5\textwidth}

\end{column}

\begin{column}{0.5\textwidth}

\end{column}
\end{columns}



\begin{frame}{Two extreme views}
\begin{columns}[t]
\begin{column}{0.5\textwidth}

differential effects
\begin{itemize}
\item flexible but complex
\end{itemize}

\includegraphics[scale=0.8]{figures/by-me/monoall-dependencies-2/obs-simple-general/obs-simple-general}
\end{column}

\begin{column}{0.5\textwidth}

uniform effects
\begin{itemize}
\item rigid but parsimonious
\end{itemize}

\includegraphics[scale=0.8]{figures/by-me/monoall-dependencies-2/obs-simple-general-gene-aspec/obs-simple-general-gene-aspec}
\end{column}
\end{columns}
\end{frame}


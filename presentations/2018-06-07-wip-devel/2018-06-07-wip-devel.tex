\documentclass[usenames,dvipsnames]{beamer}
%\documentclass[handout]{beamer}

% language settings
%\usepackage{fontspec, polyglossia}
%\setdefaultlanguage{magyar}

% common packages
\usepackage{amsmath, multimedia, hyperref, color, multirow}
\usepackage[table]{xcolor}
%\usepackage{graphicx}

% TikZ
\usepackage{tikz}
%\usetikzlibrary{arrows.meta, decorations.pathmorphing, decorations.pathreplacing, shapes.geometric,mindmap}
%\usetikzlibrary{shapes.geometric,fadings,bayesnet}

% beamer styles
\mode<presentation>{
%\usetheme{Warsaw}
\usetheme{Boadilla}
%\usetheme{Antibes}
%\usecolortheme{beaver}
\usecolortheme{crane}
%\usefonttheme{structureitalicserif}
\setbeamercovered{transparent}
}
\setbeamertemplate{blocks}[rounded][shadow=true]
\AtBeginSubsection[]{
  \begin{frame}<beamer>{Contents}
    \tableofcontents[currentsection,currentsubsection]
  \end{frame}
}
%\useoutertheme[]{tree}

% title, etc
\title{Schizophrenia and Variation of Imprinting}
\subtitle{Unperturbed Expression Bias of Imprinted Genes in Schizophrenia}
\author{Attila Guly\'{a}s-Kov\'{a}cs}
\date{Chess Lab}

\begin{document}

\maketitle

\begin{frame}{Outline}
\begin{enumerate}
\item summarize our published study\\
{\small
\textbf{Unperturbed Expression Bias of Imprinted Genes in Schizophrenia}\\
Gulyas-Kovacs et al \textit{Nature Communications}, in press}
\item demo: lab notebook 
\item ongoing work: brain somatic mozaicism 
\end{enumerate}
\end{frame}

\begin{frame}{The complex genetic architecture of schizophrenia}
\includegraphics[width=1.0\textwidth]{figures/from-others/sullivan-natrevgenet-2012-fig1b.jpg}

{\tiny Sullivan 2012 Nat Rev Genet.}
\end{frame}

\begin{frame}[t, label=sister-disorders]{Imprinted genes: role in
schizophrenia?}
\includegraphics[width=1.0\textwidth]{figures/from-others/peters-2014-imprinting-fig1b.jpg}

%{\tiny Peters 2014}
\begin{columns}[t]
\begin{column}{0.5\textwidth}
%{\tiny Angelman syndrome. Boy with a puppet}

\includegraphics<2>[width=0.60\columnwidth]{figures/from-others/boy-with-a-puppet-Giovanni-Francesco-Caroto.jpg}

%{\tiny Boy with a Puppet}
\end{column}
\begin{column}{0.5\textwidth}
%{\tiny Prader-Willi syndr.  Eugenia ``La Monstrua''}

\includegraphics<2>[width=0.70\columnwidth]{figures/from-others/Eugenia-Martínez-Vallejo-clothed-cropped.jpg}

%{\tiny Eugenia Mart\'{i}nez Vallejo}
\end{column}
\end{columns}
\end{frame}

\begin{frame}{Our study as part of the CommonMind Consortium}
previous work
\begin{itemize}
\item RNA-seq on Control, SCZ, AFF individuals
\item allele nonspecific expression 
\item Fromer et al\footnote{Nature Neuroscience , 19(11):1442} found
differences
\end{itemize}
\vfill

present study
\begin{itemize}
\item allele specific expression
\item which genes are imprinted in human DLPF cortex?
\item inter-individual variation of imprinting: schizophrenia?
\end{itemize} 
\end{frame}

\begin{frame}[label=cmc]{We adopted the \emph{read count ratio} approach}
\includegraphics[width=1.0\textwidth]{figures/by-me/commonmind-rna-seq/commonmind-rna-seq.pdf}
\end{frame}

\begin{frame}{Identification of 30 imprinted genes in human DLFPC}
\begin{center}
\includegraphics[height=0.85\textheight]{figures/2016-07-19-genome-wide-S/complex-plot-1.png}
\end{center}
\end{frame}

\begin{frame}{Univariate analysis: Control, SCZ, AFF similar}
\begin{center}
\includegraphics[height=0.85\textheight]{figures/2016-11-01-plotting-distribution-of-s/S-Dx-strip-1.pdf}
\end{center}
\end{frame}

\begin{frame}{Univariate analysis: age effect}
\begin{center}
\includegraphics[height=0.85\textheight]{figures/2016-11-01-plotting-distribution-of-s/S-Dx-age-1.pdf}
\end{center}
\end{frame}

\begin{frame}{Data call for multivariate analysis}
\footnotesize
\begin{tabular}{|c|ccc|cccc|}
 & \multicolumn{3}{|c|}{read count ratios} & \multicolumn{4}{|c|}{explanatory variables} \\
\(i\) & \(Y_1\) & \(\hdots\) & \(Y_{30}\) & \(X_1\) & \(X_2\) & \(\hdots\) & \(X_{12}\) \\
\textbf{Individual} & \textbf{MAGEL2} & \(\hdots\) & \textbf{UBE3A} & \textbf{Dx} & \textbf{Age} & \(\hdots\) & \textbf{RIN} \\
\hline
1 & NA & \(\hdots\) & NA & AFF & 42 & \(\hdots\) & 6.90 \\
2 & 1.00 & \(\hdots\) & NA & AFF & 58 & \(\hdots\) & 7.00 \\
3 & NA & \(\hdots\) & 0.89 & AFF & 28 & \(\hdots\) & 6.90 \\
\(\vdots\) & \(\vdots\) & \(\vdots\) & \(\vdots\) & \(\vdots\) & \(\vdots\) & \(\vdots\) & \(\vdots\) \\
%574 & NA & \(\vdots\) & 0.90 & SCZ & 59 & \(\vdots\) & 8.60 \\
%575 & NA & \(\vdots\) & NA & Control & 38 & \(\vdots\) & 8.30 \\
576 & 0.97 & \(\hdots\) & NA & Control & 42 & \(\hdots\) & 8.50 \\
577 & 1.00 & \(\hdots\) & NA & SCZ & 27 & \(\hdots\) & 7.50 \\
578 & 1.00 & \(\hdots\) & NA & Control & 57 & \(\hdots\) & 8.60 \\
579 & 1.00 & \(\hdots\) & 1.00 & Control & 28 & \(\hdots\) & 7.70 \\
\end{tabular}
\end{frame}

\begin{frame}{Multivariate statistical models}
\begin{columns}[t]
\begin{column}{0.33\textwidth}
\emph{fixed I}

\includegraphics[scale=0.8]{figures/by-me/monoall-dependencies-2/obs-simple-general/obs-simple-general}
\end{column}

\begin{column}{0.33\textwidth}
\emph{fixed II}

\includegraphics[scale=0.8]{figures/by-me/monoall-dependencies-2/obs-simple-general-gene-aspec/obs-simple-general-gene-aspec}
\end{column}
\begin{column}{0.33\textwidth}
\emph{mixed}

\includegraphics[scale=0.8]{figures/by-me/monoall-dependencies-2/mixed/mixed}
\end{column}
\end{columns}
\vfill
\begin{itemize}
\item fixed I: too complex \(\Rightarrow\) low power
\item fixed II: too simplistic \(\Rightarrow\) bias
\item mixed: powerful middle ground---even with interactions
\end{itemize}
\end{frame}

\begin{frame}{Results and interpretation}
\footnotesize
\begin{tabular}{rll}
\hline
\textbf{Hypothesis}                      & \textbf{Results}             & \textbf{Interpretation}                               \\
predictor term                             & p-value                &                                              \\
\hline
\((1\,|\,\mathrm{Gene})\)                   & \(8.5\times 10^{-28}\) & imprinted genes vary in allelic bias     \\
\colorbox{pink}{\((1\,|\,\mathrm{Dx})\)}                     & \colorbox{pink}{\(1.0\)}                & similar allelic bias for Control, SCZ, AFF \\
\colorbox{pink}{\((1\,|\,\mathrm{Dx}:\mathrm{Gene})\)}       & \colorbox{pink}{\(0.21\)}               & similar gene specific allelic bias for Control, SCZ, AFF    \\
\(\mathrm{Age}\)                            & \(0.39\)               & no uniform effect of Age on allelic bias               \\
\colorbox{LimeGreen}{\((\mathrm{Age}\,|\,\mathrm{Gene})\)}        & \colorbox{LimeGreen}{\(2.5\times 10^{-5}\)}  & gene specific effect of Age on allelic bias                \\
\(\mathrm{Ancestry.1}\)                     & \(0.24\)               & no uniform genetic effect on allelic bias               \\
\colorbox{LimeGreen}{\((\mathrm{Ancestry.1}\,|\,\mathrm{Gene})\)} & \colorbox{LimeGreen}{\(4.6\times 10^{-16}\)} & gene specific genetic effect on allelic bias               \\ %\(\mathrm{Ancestry.3}\)                     & \(0.54\)               & no uniform genetic effect on allelic bias               \\
%\((\mathrm{Ancestry.3}\,|\,\mathrm{Gene})\) & \(3.8\times 10^{-5}\)  & gene specific genetic effect on allelic bias               \\
\((1\,|\,\mathrm{Gender})\)                 & \(1.0\)                & no uniform Male-Female difference in allelic bias       \\
\colorbox{LimeGreen}{\((1\,|\,\mathrm{Gender}:\mathrm{Gene})\)}   & \colorbox{LimeGreen}{\(5.5\times 10^{-3}\)}  & gene specific Male-Female difference in allelic bias       \\
\hline
\end{tabular}
\end{frame}

\begin{frame}{Lab notebook}
aims
\begin{enumerate}
\item efficient presentation 
\begin{itemize}
\item<2> web: shareability, hyperlinks, figures, math typesetting,...
\end{itemize}
\item reproducible research
\begin{itemize}
\item<2> version control: \texttt{git}
\item<2> literate programming: include code chunks
\item<2> dynamic documents: code \(\Rightarrow\) output
\end{itemize}
\end{enumerate}
\end{frame}

\begin{frame}{Brain Somatic Mosaicism}
\includegraphics[width=1.0\columnwidth]{figures/from-others/gage-curropsysbio-2016-1.jpg}

{\tiny Paquola, Erwin, Gage 2016}
\end{frame}

\begin{frame}{Detection of somatic variants}
\includegraphics[width=1.0\textwidth]{figures/from-others/bsm-science-fig2.jpg}

{\tiny Science. 2017 356(6336)}
\end{frame}

\begin{frame}{No single best variant caller method}
\begin{center}
\includegraphics{figures/by-me/var-calling-workflows/var-calling-workflows.pdf}
\end{center}
\end{frame}

\begin{frame}{Low concordance of callsets}
\begin{center}
\includegraphics[height=0.85\textheight]{figures/2018-04-08-call-set-concordance/venn-common-sample-wgs-snvs-1.png}
\end{center}
\end{frame}

\begin{frame}{Ongoing work}
dealing with error
\begin{itemize}
\item estimate error rates using gold standard DNA
\item optimization---machine learning
\end{itemize}
\vfill
somatic variants and schizophrenia
\begin{itemize}
\item comparison of Control and SCZ samples
\end{itemize} 
\end{frame}

\begin{frame}{Acknowledgements}
thanks to
\begin{itemize}
\item Andy Chess
\item Chaggai Rosenbluh
\item Eva Xia 
\end{itemize}
\vfill
publications
\begin{itemize}
\item Gulyas-Kovacs et al 2018 \\
\emph{Nature Communications, in press\\
bioRxiv 329748; doi: https://doi.org/10.1101/329748 }
\item my lab notebook\\
\emph{https://attilagk.github.io/monoallelic-brain-notebook/}
\end{itemize}
\begin{columns}[t]
\begin{column}{0.5\textwidth}
\end{column}

\begin{column}{0.5\textwidth}
\end{column}
\end{columns}
\end{frame}

\end{document}



\begin{columns}[t]
\begin{column}{0.5\textwidth}

\end{column}

\begin{column}{0.5\textwidth}

\end{column}
\end{columns}


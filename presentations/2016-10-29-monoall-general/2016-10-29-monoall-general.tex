\documentclass{beamer}
%\documentclass[handout]{beamer}

% language settings
%\usepackage{fontspec, polyglossia}
%\setdefaultlanguage{magyar}

% common packages
\usepackage{amsmath, multimedia, hyperref, color, multirow}
%\usepackage{graphicx}

% TikZ
\usepackage{tikz}
%\usetikzlibrary{arrows.meta, decorations.pathmorphing, decorations.pathreplacing, shapes.geometric,mindmap}
%\usetikzlibrary{shapes.geometric,fadings,bayesnet}

% beamer styles
\mode<presentation>{
%\usetheme{Pittsburgh}
\usetheme{Copenhagen}
\usecolortheme{beaver}
%\usecolortheme{seahorse}
%\usefonttheme{structureitalicserif}
\setbeamercovered{transparent}
}
\setbeamertemplate{blocks}[rounded][shadow=true]
\AtBeginSection[]{
  \begin{frame}<beamer>{Contents}
    \tableofcontents[currentsection]
    %\tableofcontents[currentsection,currentsubsection]
  \end{frame}
}
%\useoutertheme[]{tree}

\title{Genomic Imprinting in the Human Brain}
\subtitle{Links to Aging, Gender, and Schizophrenia}
\author{Attila Guly\'{a}s-Kov\'{a}cs}
\date{Mount Sinai School of Medicine}

\newcommand{\platefigscale}[0]{0.7}
\newcommand{\ownfigscale}[0]{0.4}

\begin{document}

\maketitle

% SEC: Introduction
\section{Introduction}
\subsection{(Hypothetical) roles of imprinting}

% ontogeny and imprinting
\begin{frame}{Genomic imprints during development}
\includegraphics[height=0.7\textheight]{figures/from-others/plasschaert-bartolomei-2014-fig1.jpeg}
\vfill
{\tiny Plasschaert \& Bartolomei 2014 Development.}
\end{frame}

% physiological role: growth and neuropsych. syndromes
\begin{frame}{Imprinting and placentation}{}
\begin{columns}[t]
\begin{column}{0.4\textwidth}

parental expression bias

\includegraphics[width=\columnwidth]{figures/from-others/renfree-2012-fig2.jpg}

{\tiny Renfree et al 2012 Philos Trans R Soc Lond B}

\end{column}

\begin{column}{0.6\textwidth}

function, evolution,
\onslide<2>{kinship theory}

\includegraphics[width=\columnwidth]{figures/from-others/smits-et-al-2008-fig5.jpg}

{\tiny Smits et al 2008 Nat Genet}
\end{column}
\end{columns}
\begin{center}
\end{center}
\end{frame}

%% physiological role: growth and neuropsych. syndromes
%\begin{frame}{Further physiological roles}{}
%\includegraphics[width=\columnwidth]{figures/from-others/peters-2014-imprinting-fig3.jpg}
%
%{\tiny Peters 2014 Nat Rev Genet.}
%\end{frame}

\begin{frame}[t]{Sister disorders, neuropsychiatric functions}
\includegraphics[width=0.65\textwidth]{figures/from-others/peters-2014-imprinting-fig1b.jpg}
{\tiny Peters 2014 Nat Rev Genet.}
\visible<1>{
\begin{columns}[t]
\begin{column}{0.5\textwidth}
{\footnotesize Angelman syndrome}

\includegraphics<1>[width=0.65\columnwidth]{figures/from-others/boy-with-a-puppet-Giovanni-Francesco-Caroto.jpg}

{\tiny Boy with a Puppet}
\end{column}
\begin{column}{0.5\textwidth}
{\footnotesize Prader-Willi syndrome}

\includegraphics<1>[width=0.65\columnwidth]{figures/from-others/Eugenia-Martínez-Vallejo-clothed-cropped.jpg}

{\tiny Eugenia Martínez Vallejo}
\end{column}
\end{columns}
}

\visible<2>{
\footnotesize genetic architecture of schizophrenia
\includegraphics[width=0.7\textwidth]{figures/from-others/sullivan-natrevgenet-2012-fig1b.jpg}
{\tiny Sullivan 2012 Nat Rev Genet.}
}
\end{frame}

\begin{frame}{The imprinted brain hypothesis}
\includegraphics[width=0.7\textwidth]{figures/from-others/crespi-2008-fig3.png}
\vfill
{\tiny \raggedright{Crespi \& Badcock 2008 Behav Brain Sci.}}
\end{frame}

\begin{frame}{Changes in postnatal age: predictions, results}
\includegraphics[height=0.4\textheight]{figures/from-others/ubeda-2012-fig1.jpg}
\hfill
\includegraphics[height=0.4\textheight]{figures/from-others/ubeda-2012-fig3a.jpg}
{\tiny Ubeda 2012 Evolution}
\vfill
\includegraphics[height=0.3\textheight]{figures/from-others/perez-2015-elife-fig4b.png}
{\tiny Perez et al 2015 eLife}
\end{frame}

\subsection{Goals and study design}
% questions: (1) which genes under what conditions: age, gender, genotype?
% (2) role in psychiatric disorders?

\begin{frame}{Our research study}
\begin{description}
\item[data/project] Common Mind Consortium
%\item[approach] human genomics
\item[questions] imprinted genes in the brain
\begin{itemize}
\item variation of parental bias across genes and individuals
\item regulators: age, gender, genotype (ancestry)
\item psychiatric disorders (SCZ, AFF)  
\end{itemize}
\item[participants] \alert{Ifat Keydar}, Eva Xia, Menachem Fromer, Doug Ruderfer, Ravi Sachinanandam, Andrew Chess
\end{description}
\end{frame}

\begin{frame}[label=cmc]{The Common Mind data}
% CommonMind: within cohort variation of age, gender, genotype, Dx; correlate
% with RNA-seq-based measure of parental bias
\includegraphics[height=0.75\textheight]{figures/by-me/commonmind-rna-seq/commonmind-rna-seq.pdf}
\end{frame}

% SEC: Results
\section{Results \& Discussion}
\subsection{Predictors of parental bias}

\begin{frame}{Calling imprinted genes}
\begin{columns}[t]
\begin{column}{0.5\textwidth}

\includegraphics[height=0.7\textheight]{figures/2016-08-01-ifats-filters/venn-total-finalf-called-1.pdf}
\end{column}

\begin{column}{0.5\textwidth}

\includegraphics[height=0.7\textheight]{figures/2016-08-01-ifats-filters/top-ranking-genes-1.pdf}
\end{column}
\end{columns}
\end{frame}

\begin{frame}{Fitting models to read count ratios}
\begin{columns}
\begin{column}{0.60\textwidth}

\includegraphics[width=\columnwidth]{figures/2016-08-23-glm-sampling-distributions/KCNK9-1.pdf}
\end{column}
\begin{column}{0.45\textwidth}
\begin{equation*}
Y_g = X \beta_g + \epsilon_g
\end{equation*}
\vfill
%\begin{tabular}{|r|l|}
%\hline
%response \(Y_g\) & transformation \\
%\hline
%read count ratio \(S_g\) & none \\
%\(Q_g\) & quasi-log \\
%\(R_g\) & rank \\
%\hline
%\end{tabular}
\footnotesize
\begin{itemize}
\item \(Y_g\) from read count ratio
\item \(X\) based on predictor(s)
\item \( \beta_g = 0 \; \Leftrightarrow \) no effect
\end{itemize}
\onslide<2>
\tiny
\vfill
\begin{tabular}{|r|l|}
\hline
predictor & levels \\
\hline
Age & \\
Gender & [Female], Male\\
Dx & [AFF], Control, SCZ\\
Ancestry.1-5 & \\
Institution & [MSSM], Penn, Pitt\\
PMI & \\
RIN & \\
RNA\_batch & [0], A, B, C, D, E, F, G, H\\
\hline
\end{tabular}

%\includegraphics[scale=\platefigscale]{figures/by-me/monoall-dependencies-2/designed/designed.pdf}

\end{column}
\end{columns}
\end{frame}

\begin{frame}[plain]
\begin{columns}[t]
\begin{column}{0.5\textwidth}

\includegraphics[height=\textheight]{figures/2016-06-22-extending-anova/reg-coef-wnlm-Q-1.pdf}
\end{column}

\begin{column}{0.5\textwidth}

\includegraphics[height=\textheight]{figures/2016-06-22-extending-anova/reg-coef-logi-S-filt-1.pdf}
\end{column}
\end{columns}
\end{frame}
% genome-wide patterns of variation, complex figure

% models and their fit to data (prediction-data overlay)

% data fit: model checking

% near collinearity, RNA quality, scatter plot matrix, likelihood surface

% designed experiment (mouse) vs observational study (human, CommonMind)
\begin{frame}[t, plain]
\begin{columns}[t]
\begin{column}{0.5\textwidth}
designed experiment

\includegraphics[scale=\platefigscale]{figures/by-me/monoall-dependencies-2/designed/designed.pdf}

\end{column}

\begin{column}{0.5\textwidth}
observational study
\includegraphics[scale=\platefigscale]{figures/by-me/monoall-dependencies-2/obs/obs.pdf}
\end{column}
\end{columns}
\vfill

\begin{columns}[t]
\begin{column}{0.5\textwidth}
%orthogonal

\includegraphics[height=0.35\textheight]{figures/from-others/tony-smith-die.png}
\end{column}

\begin{column}{0.5\textwidth}
%non-orthogonal

\includegraphics[height=0.35\textheight]{figures/from-others/tony-smith-new-piece.png}
\end{column}
\end{columns}
\end{frame}

\againframe{cmc}

% ANOVA
\begin{frame}{ANOVA is inconclusive}
\includegraphics[scale=\ownfigscale]{figures/2016-06-22-extending-anova/anova-fw-rv-wnlm-Q-1.pdf}
\end{frame}

% likelihood surface
\begin{frame}{Poor model identifiability}{The likelihood of one parameter is linked to
another}
\includegraphics[scale=\ownfigscale]{figures/2016-08-21-likelihood-surface/ll-surf-coefs-wnlm-Q-1.pdf}
\end{frame}

\begin{frame}{Complication: interaction}
{The effect of one depends on another}
\begin{columns}[t]
\begin{column}{0.4\textwidth}

\includegraphics[scale=\platefigscale]{figures/by-me/monoall-dependencies-2/obs-contextual/obs-contextual.pdf}
\end{column}

\begin{column}{0.6\textwidth}

\includegraphics[width=\columnwidth]{figures/2016-07-08-conditional-inference/beta-age-cond-wnlm-Q-2-1.pdf}
\end{column}
\end{columns}
\end{frame}

\begin{frame}
{The main results}
\begin{columns}[t]
\begin{column}{0.5\textwidth}
\begin{center}
biological effects

\includegraphics[width=\columnwidth]{figures/2016-08-08-imprinted-gene-clusters/segplot-wnlm-Q-99conf-1.pdf}

\tiny
model: wnlm.Q
\end{center}
\end{column}

\begin{column}{0.5\textwidth}
\begin{center}
\only<2>{agreement between models}

\includegraphics<2>[width=\columnwidth]{figures/2016-06-22-extending-anova/logi-S-filtered-wnlm-Q-compare-1.pdf}
\end{center}
\end{column}
\end{columns}
\end{frame}

\begin{frame}{Significantly affected genes}
\includegraphics<1>[scale=\ownfigscale]{figures/2016-10-03-permutation-test/p-values-1.pdf}
\visible<2>{
\tiny
\begin{tabular}{lllll}
Gene & Gene type & Chr & Coefficient & Known phenotype\\
\hline
ZDBF2 & protein coding & 2 & Age,  Ancestry.1 & \\
NAP1L5 & protein coding & 4 & GenderMale & \\
PEG10 & protein coding & 7 & DxSCZ & \\
MEST & protein coding & 7 & DxSCZ & Silver-Russell syndrome\\
KCNK9 & protein coding & 8 & Age & Birk-Barel mental retardation dysmorphism syndrome\\
INPP5F & protein coding & 10 & Age & cell motility; endocytic recycling\\
KCNQ1OT1 & antisense & 11 & GenderMale & Beckwith-Wiedemann syn.; Isol.~hemihyperplasia\\
MEG3 & lincRNA & 14 & GenderMale & Mat/pat 14q32.2 hypermeth/microdel syndrome\\
RP11-909M7.3 & lincRNA & 14 & DxSCZ & \\
AL132709.5 & miRNA & 14 & Ancestry.1 & \\
MAGEL2 & protein coding & 15 & Age & Prader-Willi syn.; Schaaf-Yang syn.;
Arthrogryposis \\
NDN & protein coding & 15 & GenderMale & Prader-Willi syndrome\\
PWRN1 & lincRNA & 15 & Ancestry.1 & Prader-Willi syndrome\\
UBE3A & protein coding & 15 & DxSCZ & Prader-Willi syn.; Angelman syn.; circadian rhythm\\
PEG3 & protein coding & 19 & GenderMale & \\
\end{tabular}

}
\end{frame}

%\begin{frame}{Comparison to a mouse study: agreement}
%\begin{columns}[t]
%\begin{column}{0.5\textwidth}
%\begin{center}
%present work
%
%\includegraphics[width=\columnwidth]{figures/2016-08-08-imprinted-gene-clusters/segplot-wnlm-Q-99conf-1.pdf}
%
%\end{center}
%\end{column}
%
%\begin{column}{0.5\textwidth}
%\begin{center}
%Perez et al 2015
%
%\includegraphics[height=0.3\textheight]{figures/from-others/perez-2015-elife-fig4b.png}
%\end{center}
%\end{column}
%\end{columns}
%\end{frame}
%
%\begin{frame}{Comparison to a mouse study: disagreement}
%\includegraphics[scale=\ownfigscale]{figures/2016-10-11-comparison-to-mouse-cerebellum/posterior-pp-vs-pval-wnlm-Q-1.pdf}
%\end{frame}
%
\begin{frame}{Summary of results}
\begin{enumerate}
\item parental bias of 30 imprinted genes was studied
\item age, gender and genetics regulate genes in various manner
\item bias of some genes is linked to schizophrenia 
\item caution: found limitations of employed models
\end{enumerate}
\end{frame}

% (G)LM vs BN model: complex dependency structure and latent biol and tech variables

% analysis of nonlinearity points to interactions

% significant departures from beta = 0

% direction of effect and clusters

\subsection{Refining the (epi)genetics of schizophrenia}

% return to main results; answer to 1st question: regulation by age,...

% Comparison to BRAIM-mouse: agreement: effect of age and genotype; age: both
% up- and downregulation; disagreement: specifics of age; gender in general

% 2nd question: SCZ; UBE3A previous evidence (CNVs); MEST

% Comparison to CommonMind SCZ differential expression; tighter dosage
% regulation of imprinted genes than on-imprinted ones?

% Imprinted brain hypothesis

% SEC: Future extensions

\begin{frame}{Comparison to overall expression analysis\(^\ast\)}
\includegraphics[height=0.8\textheight]{figures/2016-10-20-differential-expression-scz/venn-triple-1.pdf}

{\tiny\(\ast\)Fromer et al 2016}
\end{frame}

\begin{frame}{Epigenetics and tissue specificity}

\end{frame}

% modeling: account for dependencies and increase statistical power: borrowing of strength via global (genome-wide) model:
% BRAIM and AGK M2 or M3
\begin{frame}{Improving models}
\begin{columns}[t]
\begin{column}{0.3\textwidth}
present

\includegraphics[scale=\platefigscale]{figures/by-me/monoall-dependencies-2/obs-simple-general/obs-simple-general.pdf}
\end{column}
\begin{column}{0.3\textwidth}
proposed

\includegraphics[scale=\platefigscale]{figures/by-me/monoall-dependencies-2/obs-bayesian/obs-bayesian.pdf}
\end{column}
\begin{column}{0.6\textwidth}
\begin{itemize}
\item more power
\begin{itemize}
\item borrowing of strength
\item shared parameters
\end{itemize}
%\item utilize data at higher resolution
\item<2-> more flexibility
\begin{itemize}
\item interactions
\item tissue specificity
\item epigenetic marks 
\end{itemize} 
\begin{itemize}
\end{itemize}
\end{itemize}
\end{column}
\end{columns}

\end{frame}

% include and compare othe brain areas (CommonMind data)

% identify epigenetic marks of imprinting (PsychENCODE)

% eQTLs and methylation QTLs of imprinted genes vs GWAS SNPs

\end{document}


\begin{columns}[t]
\begin{column}{0.5\textwidth}

\end{column}

\begin{column}{0.5\textwidth}

\end{column}
\end{columns}

\documentclass{beamer}
%\documentclass[handout]{beamer}

% language settings
%\usepackage{fontspec, polyglossia}
%\setdefaultlanguage{magyar}

% common packages
\usepackage{amsmath, multimedia, hyperref, color, multirow}
%\usepackage{graphicx}

% TikZ
\usepackage{tikz}
%\usetikzlibrary{arrows.meta, decorations.pathmorphing, decorations.pathreplacing, shapes.geometric,mindmap}
%\usetikzlibrary{shapes.geometric,fadings,bayesnet}

% beamer styles
\mode<presentation>{
%\usetheme{Pittsburgh}
\usetheme{Copenhagen}
\usecolortheme{beaver}
%\usecolortheme{seahorse}
%\usefonttheme{structureitalicserif}
\setbeamercovered{transparent}
}
\setbeamertemplate{blocks}[rounded][shadow=true]
\AtBeginSection[]{
  \begin{frame}<beamer>{Contents}
    \tableofcontents[currentsection]
    %\tableofcontents[currentsection,currentsubsection]
  \end{frame}
}
%\useoutertheme[]{tree}

\title{Genomic Imprinting in the Human Brain}
\subtitle{Links to Aging, Gender, and Schizophrenia}
\author{Attila Guly\'{a}s-Kov\'{a}cs}
\date{Mount Sinai School of Medicine}

\newcommand{\platefigscale}[0]{0.7}
\newcommand{\ownfigscale}[0]{0.4}

\begin{document}

\maketitle

\section{Introduction}
\subsection{(Hypothetical) roles of imprinting}

% 1% of genes are imprinted in mice
% imprints are sex-specific methylated DNA regions generated in gametogenesis
% this results in parental bias in the expression of imprinted genes in the progeny
% complete bias: monoallelic expression
\begin{frame}{Genomic imprints during development}
\includegraphics[height=0.7\textheight]{figures/from-others/plasschaert-bartolomei-2014-fig1.jpeg}
\vfill
{\tiny Plasschaert \& Bartolomei 2014 Development.}
\end{frame}

% Igf2 and H19: among the first genes found imprinted in mice
% methylation imprint inhibits both the insulator for Igf2 and the promoter for H19
% embryonic/placental growth: paternally expressed Igf2 promotes, mat. exp. H19 inhibits it
% most imprinted genes play role in placental development and expressed in embryo/placenta
% further roles: postnatal growth, maternal behavior, metabolism, REM sleep
% this fits co-occurrence of placentation and evol. of imprinting
% kinship theory of imprinting: interactions (e.g. among siblings), inclusive fitness, relatedness
% maternal half siblings: closer relatedness of maternal genes than paternal genes => conflict
% resolution: growth promoting genes paternally biased, growth inhibiting maternally (e.g. Igf2, H19)
\begin{frame}{Imprinting and placentation}{}
\begin{columns}[t]
\begin{column}{0.4\textwidth}

parental expression bias

\includegraphics[width=\columnwidth]{figures/from-others/renfree-2012-fig2.jpg}

{\tiny Renfree et al 2012 Philos Trans R Soc Lond B}

\end{column}

\begin{column}{0.6\textwidth}

function, evolution,
\onslide<2>{kinship theory}

\includegraphics[width=\columnwidth]{figures/from-others/smits-et-al-2008-fig5.jpg}

{\tiny Smits et al 2008 Nat Genet}
\end{column}
\end{columns}
\begin{center}
\end{center}
\end{frame}

% shift away from normal parental bias in two possible directions => sister disorders
% general: growth disorders, mental defects
% Igf2/H19: Beckwith–Wiedeman syn, Silver-Russell syn
% Ube3a cluster 15q13: Angelman (mental retardation); Prader-Willi (obesity, mental defects)
% CNV in 15q13 increases risk of SCZ, a highly polygenic disorder with many common variants with mild risk
\begin{frame}[t]{Sister disorders, neuropsychiatric functions}
\includegraphics[width=0.65\textwidth]{figures/from-others/peters-2014-imprinting-fig1b.jpg}
{\tiny Peters 2014 Nat Rev Genet.}
\visible<1>{
\begin{columns}[t]
\begin{column}{0.5\textwidth}
{\footnotesize Angelman syndrome}

\includegraphics<1>[width=0.65\columnwidth]{figures/from-others/boy-with-a-puppet-Giovanni-Francesco-Caroto.jpg}

{\tiny Boy with a Puppet}
\end{column}
\begin{column}{0.5\textwidth}
{\footnotesize Prader-Willi syndrome}

\includegraphics<1>[width=0.65\columnwidth]{figures/from-others/Eugenia-Martínez-Vallejo-clothed-cropped.jpg}

{\tiny Eugenia Mart\'{i}nez Vallejo}
\end{column}
\end{columns}
}

\visible<2>{
\footnotesize genetic architecture of schizophrenia
\includegraphics[width=0.7\textwidth]{figures/from-others/sullivan-natrevgenet-2012-fig1b.jpg}
{\tiny Sullivan 2012 Nat Rev Genet.}
}
\end{frame}

% uses kinship theory to explain psychiatric conditions
% PEGs self-oriented cognition, - inc. fitness, autistic; MEGs mentalistic cognition, + incl. fitness, psychotic
% this hypothesis guides the study the role of imprinting in psychiatric disorders e.g. SCZ
\begin{frame}{The imprinted brain theory}
\includegraphics[width=0.7\textwidth]{figures/from-others/crespi-2008-fig3.png}
\vfill
{\tiny \raggedright{Crespi \& Badcock 2008 Behav Brain Sci.}}
\end{frame}

% clarifying the functional roles of imprinted genes requires characterizing their variation
% variation across devel. time, tissue type, gender,...
% variation in postnatal age, aging unclear especially in humans
% theoretical and experimental mouse studies
\begin{frame}{Changes in postnatal age: predictions, results}
\includegraphics[height=0.4\textheight]{figures/from-others/ubeda-2012-fig1.jpg}
\hfill
\includegraphics[height=0.4\textheight]{figures/from-others/ubeda-2012-fig3a.jpg}
{\tiny Ubeda 2012 Evolution}
\vfill
\includegraphics[height=0.3\textheight]{figures/from-others/perez-2015-elife-fig4b.png}
{\tiny Perez et al 2015 eLife}
\end{frame}

\subsection{Goals and study design}

% our study aims to characterize that variation in humans using CMC data
\begin{frame}{Our research study}
\begin{description}
\item[data/project] Common Mind Consortium
%\item[approach] human genomics
\item[questions] imprinted genes in the human brain
\begin{itemize}
\item variation of parental bias across genes and individuals
\item regulators: age, gender, genotype (ancestry)
\item psychiatric disorders (SCZ, AFF)  
\end{itemize}
\item[participants] \alert{Ifat Keydar}, Eva Xia, Menachem Fromer, Doug Ruderfer, Ravi Sachinanandam, Andrew Chess
\end{description}
\end{frame}

% observational study: explain the observed variation in parental bias with variation in age, gender, psych. condition
% e.g. gene g1 has stronger maternal bias in individual i1 than in i2
% parental bias not directly observed; RNA-seq read count ration S statistic; technical noise
\begin{frame}[label=cmc]{The Common Mind data}
% CommonMind: within cohort variation of age, gender, genotype, Dx; correlate
% with RNA-seq-based measure of parental bias
\includegraphics[height=0.75\textheight]{figures/by-me/commonmind-rna-seq/commonmind-rna-seq.pdf}
\end{frame}

% SEC: Results
\section{Results \& Discussion}
\subsection{Predictors of parental bias}

% based on read count and S we filtered genes and called imprinted genes
% we used the eCDF of S (shades of color) as well as a simple test for biallelic expr (black)
% 30 genes called imprinted, most previously know, few novel
\begin{frame}[label=filtering-calling]{Calling imprinted genes}
\begin{columns}[t]
\begin{column}{0.5\textwidth}

\includegraphics[height=0.7\textheight]{figures/2016-08-01-ifats-filters/venn-total-finalf-called-1.pdf}
\end{column}

\begin{column}{0.5\textwidth}

\includegraphics[height=0.7\textheight]{figures/2016-08-01-ifats-filters/top-ranking-genes-1.pdf}
\end{column}
\end{columns}
\end{frame}

% main question: explain variation of parental bias gaged with read count ratio S
% Y axis: S or Q, a transformed S; X axis: age of death
% green dots: data on KCNK9 gene: observations from hundreds of individuals
% phenomenological approach: fit several regression models to data and use the best fitting one(s)
% response Y depending on (explained by) predictor X and reg. coef beta
% predicted curve (black); the slope given by beta; no dependence => beta=0
% noise eps: stochastic variation about the predicted curve, which is not explained by X
% fitting: subjectively, "by eye": compare green cloud to density; objectively: least squares (ML)
% estimate of beta far from 0 and estimation error small => reject H0: beta=0 confidently
% confidence quantified with CIs and p-values
% complication: besides age many other predictors; multiple regression: X based on all of them
\begin{frame}{Explaining variation of parental bias with predictors}
\begin{columns}
\begin{column}{0.60\textwidth}

\includegraphics[width=\columnwidth]{figures/2016-08-23-glm-sampling-distributions/KCNK9-1.pdf}
\end{column}
\begin{column}{0.45\textwidth}
\begin{equation*}
Y_g = X \beta_g + \epsilon_g
\end{equation*}
\vfill
%\begin{tabular}{|r|l|}
%\hline
%response \(Y_g\) & transformation \\
%\hline
%read count ratio \(S_g\) & none \\
%\(Q_g\) & quasi-log \\
%\(R_g\) & rank \\
%\hline
%\end{tabular}
\footnotesize
\begin{itemize}
\item \(Y_g\) from read count ratio
\item \(X\) based on predictor(s)
\item \( \beta_g = 0 \; \Leftrightarrow \) no effect
\end{itemize}
\onslide<2>
\tiny
\vfill
\begin{tabular}{|r|l|}
\hline
predictor & levels \\
\hline
Age & \\
Gender & [Female], Male\\
Dx & [AFF], Control, SCZ\\
Ancestry.1-5 & \\
Institution & [MSSM], Penn, Pitt\\
PMI & \\
RIN & \\
RNA\_batch & [0], A, B, C, D, E, F, G, H\\
\hline
\end{tabular}

%\includegraphics[scale=\platefigscale]{figures/by-me/monoall-dependencies-2/designed/designed.pdf}

\end{column}
\end{columns}
\end{frame}

% ML fitting result: wmlm.Q fitted well for all genes and logi.S also fitted well for some genes
% Y axes lists imprinted genes; each panel corresponds to a predictor
% X axes: the H0: beta=0 (red), estimated betas + 99% CIs are shown
% before biol inference I discuss statistical properties that hinder/complicate that inference
\begin{frame}[plain, label=all-betas]
\begin{columns}[t]
\begin{column}{0.5\textwidth}

\includegraphics[height=\textheight]{figures/2016-06-22-extending-anova/reg-coef-wnlm-Q-1.pdf}
\end{column}

\begin{column}{0.5\textwidth}

\includegraphics[height=\textheight]{figures/2016-06-22-extending-anova/reg-coef-logi-S-filt-1.pdf}
\end{column}
\end{columns}
\end{frame}

% contrast designed experiment (mouse) vs observational study (human, CommonMind)
% IN BOTH CASES we have a response Y which either depends on somehow (arrows) or indep of 3 predictors (X_j)
% 3 corresponding betas describe these (in)dependencies; e.g beta2=0
% our data consist of observations on Y and X_j
% A CRUCIAL DIFFERENCE is in the dependencies (correlations) among X_j
% in designed experiments we determine X_j so we have the means to prevent correlations between them
% this is equilant to orthogonality of X_j; equal number (=1) of observations fall on each of the 8 verteces of a cube
% in obs studies we can only observe but not control X_j and so cannot prevent correlations
% correlations come from either direct dependence between X_j or indirect dependence mediated by e.g. Y
% either way: the result is non-orthognality; the cube becomes slanted, a parallelepiped
% example for X2 -> X3 dependency...
\begin{frame}[t, plain]
\begin{columns}[t]
\begin{column}{0.5\textwidth}
designed experiment

\includegraphics[scale=\platefigscale]{figures/by-me/monoall-dependencies-2/designed/designed.pdf}

\end{column}

\begin{column}{0.5\textwidth}
observational study
\includegraphics[scale=\platefigscale]{figures/by-me/monoall-dependencies-2/obs/obs.pdf}
\end{column}
\end{columns}
\vfill

\begin{columns}[t]
\begin{column}{0.5\textwidth}
%orthogonal

\includegraphics[height=0.35\textheight]{figures/from-others/tony-smith-die.png}
\end{column}

\begin{column}{0.5\textwidth}
%non-orthogonal

\includegraphics[height=0.35\textheight]{figures/from-others/tony-smith-new-piece.png}
\end{column}
\end{columns}
\end{frame}

% ...in the CMC data we see that age depends on another predictor, institution
\againframe{cmc}

% non-orthogonality => ANOVA fails
% we cannot estimate the component of variability due to each predictor in isolation from others
\begin{frame}{Consequence: ANOVA is inconclusive}
\includegraphics[scale=\ownfigscale]{figures/2016-06-22-extending-anova/anova-fw-rv-wnlm-Q-1.pdf}
\end{frame}

% non-orthogonality => difficult to differentiate between different submodels using the likelihood based on data
% log-likelihood is a function on multidim param space of several betas => difficult to visualize
% here only 2D slices of LL given by pairs of betas
% the contours of LL are quasi-ellipses delimiting equally likely submodels identified by (beta1, beta2)
% the peak of the LL surface is at the ML estimate of beta
% the ML estmate of beta_Age < 0, beta_RIN => parental bias decreases with Age and increases with RIN
% within the second contour line the (beta_Age, beta_RIN) are still quite likely
% but non-orthogonality of Age and RIN results in interdependence of likely values of beta_Age and beta_RIN
% so if beta_RIN=0.11 (quite likely), then the most likely value of beta_Age=0 => our conclusion changes on the age effect
\begin{frame}{Consequence: poor identifiability}
%{The likelihood of one parameter is linked to another}
\includegraphics[scale=\ownfigscale]{figures/2016-08-21-likelihood-surface/ll-surf-coefs-wnlm-Q-1.pdf}
\end{frame}

% two predictors interact when the effect of one predictor (X1) depends on the value (a or b) of another (X2) 
% this is distinct from correlation/non-orthogonality of predictors
% e.g. if X2=a then X1 has no effect on Y (beta1=0) but if X2=b then it does % (beta1>0)
% we found that effect of Age on S depended on both Institution and Gender at various degrees for differnt genes
% we didn't model interactions to avoid extra parameters and decrease in power => our conclusions on beta are too simplistic
\begin{frame}{Further complication: interaction}
{The effect of one predictor depends on another}
\begin{columns}[t]
\begin{column}{0.4\textwidth}

\includegraphics[scale=\platefigscale]{figures/by-me/monoall-dependencies-2/obs-contextual/obs-contextual.pdf}
\end{column}

\begin{column}{0.6\textwidth}

\includegraphics[width=\columnwidth]{figures/2016-07-08-conditional-inference/beta-age-cond-wnlm-Q-2-1.pdf}
\end{column}
\end{columns}
\end{frame}

\againframe[plain]{all-betas}

% Age, Gender, psych. condition (Dx) and genetics (Ancestry.1) all effect one gene or another
% all 4 predictors, in particular Age, may both increase or decrease parental bias
% SCZ is associated to parental bias of several genes consistent with imprinted brain theory
% there seems to be some variability within imprinted gene clusters suggesting they are not regulated exclusively in a concerted fashion
% comparing these results to those under the log.S model shows reasonable greement
\begin{frame}
{The main results}
\begin{columns}[t]
\begin{column}{0.5\textwidth}
\begin{center}
biological effects

\includegraphics[width=\columnwidth]{figures/2016-08-08-imprinted-gene-clusters/segplot-wnlm-Q-99conf-1.pdf}

\tiny
model: wnlm.Q
\end{center}
\end{column}

\begin{column}{0.5\textwidth}
\begin{center}
\only<2>{agreement between models}

\includegraphics<2>[width=\columnwidth]{figures/2016-06-22-extending-anova/logi-S-filtered-wnlm-Q-compare-1.pdf}
\end{center}
\end{column}
\end{columns}
\end{frame}

% p-values for the H0: beta=0 calculated in 2x2 methods
% method 1 vs 2 (wnlm.Q) and 3 vs 4 (logi.S): parametric vs non-parametric
% the agreement is good but not perfect => difficulty of making inferences based on multiple plausible models
% heuristic rule: method 1 AND method 2; ignore methods 3,4 because Andy doesn't like logi.S
\begin{frame}{Significantly affected genes}
\includegraphics<1>[scale=\ownfigscale]{figures/2016-10-03-permutation-test/p-values-1.pdf}
\visible<2>{
\tiny
\begin{tabular}{lllll}
Gene & Gene type & Chr & Coefficient & Known phenotype\\
\hline
ZDBF2 & protein coding & 2 & Age,  Ancestry.1 & \\
NAP1L5 & protein coding & 4 & GenderMale & \\
PEG10 & protein coding & 7 & DxSCZ & \\
MEST & protein coding & 7 & DxSCZ & Silver-Russell syndrome\\
KCNK9 & protein coding & 8 & Age & Birk-Barel mental retardation dysmorphism syndrome\\
INPP5F & protein coding & 10 & Age & cell motility; endocytic recycling\\
KCNQ1OT1 & antisense & 11 & GenderMale & Beckwith-Wiedemann syn.; Isol.~hemihyperplasia\\
MEG3 & lincRNA & 14 & GenderMale & Mat/pat 14q32.2 hypermeth/microdel syndrome\\
RP11-909M7.3 & lincRNA & 14 & DxSCZ & \\
AL132709.5 & miRNA & 14 & Ancestry.1 & \\
MAGEL2 & protein coding & 15 & Age & Prader-Willi syn.; Schaaf-Yang syn.;
Arthrogryposis \\
NDN & protein coding & 15 & GenderMale & Prader-Willi syndrome\\
PWRN1 & lincRNA & 15 & Ancestry.1 & Prader-Willi syndrome\\
UBE3A & protein coding & 15 & DxSCZ & Prader-Willi syn.; Angelman syn.; circadian rhythm\\
PEG3 & protein coding & 19 & GenderMale & \\
\end{tabular}

}
\end{frame}

% in agreement with us Perez et al also found that Age may both increase or decrease parental bias...
\begin{frame}{Comparison to a mouse study: agreement}
\begin{columns}[t]
\begin{column}{0.5\textwidth}
\begin{center}
present work

\includegraphics[width=\columnwidth]{figures/2016-08-08-imprinted-gene-clusters/segplot-wnlm-Q-99conf-1.pdf}

\end{center}
\end{column}

\begin{column}{0.5\textwidth}
\begin{center}
Perez et al 2015

\includegraphics[height=0.3\textheight]{figures/from-others/perez-2015-elife-fig4b.png}
\end{center}
\end{column}
\end{columns}
\end{frame}

% ...but the age associated genes identified by Perez et al differ from those that we found
% moreover, they didn't find any gene associated to Gender while we did
\begin{frame}{Comparison to a mouse study: disagreement}
\includegraphics[scale=\ownfigscale]{figures/2016-10-11-comparison-to-mouse-cerebellum/posterior-pp-vs-pval-wnlm-Q-1.pdf}
\end{frame}

\begin{frame}{Summary of results}
\begin{enumerate}
\item parental bias of 30 imprinted genes was studied
\item age, gender and genetics regulate genes in various manner
\item bias of some genes is linked to schizophrenia 
\item our statistical models have limitations
\end{enumerate}
\end{frame}


\subsection{Refining the (epi)genetics of schizophrenia}

% we found 4 imprinted genes with signif. association between SCZ--parental bias
% but only 1 of those 4 was found by the CMC to have association between SCZ--overall expression (i.e. not parent-specific)
% another imprinted gene was found with SCZ--overall expr. but not with SCZ--parental bias
% explanation: our method uses extra info (genotype) enhancing sensitivity for gageing overall expr. of imprinted genes 
% future diff-exp studies for imprinted genes might benefit from taking parental bias into account
\begin{frame}{Comparison to overall expression analysis\(^\ast\)}
\includegraphics[height=0.8\textheight]{figures/2016-10-20-differential-expression-scz/venn-triple-1.pdf}

{\tiny\(\ast\)Fromer et al 2016}
\end{frame}

\againframe{filtering-calling}

% modeling: account for dependencies and increase statistical power: borrowing of strength via global (genome-wide) model:
% BRAIM and AGK M2 or M3
\begin{frame}{Possibilities for future studies}
\begin{columns}[t]
\begin{column}{0.3\textwidth}
present: ``flat''

\includegraphics[scale=\platefigscale]{figures/by-me/monoall-dependencies-2/obs-simple-general/obs-simple-general.pdf}
\end{column}
\begin{column}{0.3\textwidth}
proposed: hierarch.

\includegraphics[scale=\platefigscale]{figures/by-me/monoall-dependencies-2/obs-bayesian/obs-bayesian.pdf}
\end{column}
\begin{column}{0.6\textwidth}
\begin{itemize}
\item more power
\begin{itemize}
\item borrowing of strength
\item shared parameters
\end{itemize}
%\item utilize data at higher resolution
\item<2-> more realism
\begin{itemize}
\item interactions
\end{itemize} 
\item<3-> more answers
\begin{itemize}
\item tissue specificity
\item DNA methylation
\end{itemize} 
\end{itemize}
\end{column}
\end{columns}

\end{frame}

% include and compare othe brain areas (CommonMind data)

% identify epigenetic marks of imprinting (PsychENCODE)

% eQTLs and methylation QTLs of imprinted genes vs GWAS SNPs

\end{document}


\begin{columns}[t]
\begin{column}{0.5\textwidth}

\end{column}

\begin{column}{0.5\textwidth}

\end{column}
\end{columns}

\documentclass[17pt,a1paper]{tikzposter} %Options for format can be included here
\usepackage{multicol}

\title{Genomic Imprinting in the Human Brain}
\author{Attila Guly\'{a}s-Kov\'{a}cs, Ifat Keydar and Andy Chess}
\institute{Departments of~Developmental~\& Regenerative Biology and ~Genetics \& Genomic Sciences}
%\institute{Department of~Developmental~\& Regenerative Biology; Department of~Genetics \& Genomic Sciences}
\titlegraphic{\includegraphics{figures/mount-sinai-logo.png}}

 %Choose Layout
\usetheme{Board}

\begin{document}

 % Title block with title, author, logo, etc.
\maketitle

\begin{columns}
\column{0.5}
\block{Introduction: the allelic bias of imprinted genes}{
\large
\begin{multicols}{2}
\begin{tikzfigure}[Renfree et al 2012 Philos Trans R Soc Lond B]
\includegraphics[width=\columnwidth]{figures/renfree-2012-fig2.jpg}
\end{tikzfigure}

Imprinting is thought to concern some 100 or more genes expressed in various tissues and
ages, notably much of the embryo, the placenta, and the adult brain.
Comprehensive mapping between imprinted genes, tissues and ages has
just begun.  Although the neuro-psychiatric function of imprinted genes is
unclear, they likely play some role in psychotic spectrum conditions.

Imprinted genes express their maternal and paternal allele asymmetrically; we
call this \emph{allelic bias}.  Our study infers allelic bias from RNA-seq
\emph{read count ratio} based on the CommonMind Consortium's data.

We aim at...
\begin{enumerate}
\item the number and identity of imprinted genes in the human dorsolateral
prefrontal cortex
\item the identification of predictors of allelic bias among age, ancestry, diagnosis of
schizophrenia or affective spectrum disorder, etc.
\end{enumerate}

\end{multicols}
}

\column{0.5}
\block{RNA-seq read count ratio reports on allelic bias}{
\includegraphics[width=0.9\colwidth]{figures/commonmind-rna-seq.pdf}
}
\end{columns}

\begin{columns}
\column{0.65}
\block{Allelic bias across individuals and genes}{
\begin{multicols}{2}
\innerblock{across all genes}{
\begin{center}
\includegraphics[width=0.95\columnwidth]{figures/complex-plot-c-1.png}
\end{center}
The figure shows, for each gene, a large inter-individual variation of
read count ratio, which originates partly from variation of allelic bias and 
partly from technical noise.  We scored and ranked genes based on their
inter-individual distribution function ECDF and inferred genes with the highest
ranks---among other criteria---to be imprinted.
}
\innerblock{across imprinted genes}{
\begin{center}
\includegraphics[width=0.95\columnwidth]{figures/S-age-b-1.png}
\end{center}
Shown are joint distributions of read count ratio and age for the 30 genes
we called imprinted, ordered according to gene score.  These distributions
appear to suggest that allelic bias decreases (e.g.~ZNF331) or increases
(e.g.~KCNK9) with age or is independent of age (e.g.~MEST).  However, the 
interdependence of age and other predictors calls for model-based inference.
}
\end{multicols}
}

\begin{subcolumns}
\subcolumn{0.7}
\block{Summary \& conclusions}{
\large
\begin{itemize}
\item at most 100 imprinted genes in the brain, several previously not
described 
\item technical noise substantially masks biological predictors of allelic bias
\item powerful mixed models suggest allelic bias depends on ancestry and age, but not
schizophrenia; these dependencies, however, vary across genes
\end{itemize}
}

\subcolumn{0.3}
\block{Thanks to...}{
Chaggai Rosenbluh\\
Doug Ruderfer\\
Eva Xia\\
Gabriel Hoffman\\
Menachem Fromer\\
Ravi Sachinanandam
}
\end{subcolumns}

\column{0.35}
\block{Predictors of allelic bias}{
Illustration of the \emph{fixed effects (left)} and the \emph{mixed (right)}
regression model.
Using these models we inferred the subset of predictor \(\{X_i\}\) on which the read count ratio
\(Y\) significantly depends.

\begin{center}
\includegraphics[scale=1.5]{figures/obs-simple-general.pdf}
\includegraphics[scale=1.5]{figures/obs-bayesian.pdf}
\end{center}

\innerblock{fixed effects model: weak significance}{
\begin{center}
\includegraphics[width=0.85\colwidth]{figures/tval-varpart-fixed-present-b-1.pdf}
\end{center}
}
\innerblock{mixed model: age \& ancestry effect}{

\begin{center}
\normalsize
\begin{tabular}{rrc}
term        &   \(\Delta\)AIC &        p-value \\
\hline
\hline
Age         &   1.8 &  6.5e-01 \\
\hline
\textbf{Age.Gene}    & \textbf{-23.4}    &  \textbf{2.5e-06} \\
\hline
Ances1      &  -0.2 &  1.3e-01 \\
\textbf{Ances1.Gene} & \textbf{-56.8} &  \textbf{4.8e-13} \\
\hline
Gender.fix  &   1.0 &  3.2e-01 \\
Gender.ran  &   2.0 &  1.0 \\
\textbf{Gender.Gene} &  \textbf{-5.8} &  \textbf{5.2e-03} \\
Dx.fix      &   3.9 &  9.5e-01 \\
Dx.ran      &   2.0 &  1.0  \\
Dx.Gene     &   0.5 &  2.1e-01 \\
\hline
\end{tabular}
\end{center}
}
}
\end{columns}

\end{document}



\block{Basic Block}{
Lorem ipsum dolor sit \coloredbox{amet, consectetur adipiscing elit, sed do
eiusmod tempor incididunt ut labore et dolore magna aliqua. Ut enim ad minim
veniam,} quis nostrud exercitation ullamco laboris nisi ut aliquip ex ea
commodo consequat. Duis aute irure dolor in reprehenderit in
\innerblock{voluptate velit}{ esse cillum dolore eu fugiat nulla pariatur.
Excepteur sint occaecat cupidatat non proident, sunt in culpa qui officia
deserunt mollit anim id est} laborum.}
\begin{columns}

 % FIRST column
\column{0.6}% Width set relative to text width

\block{Large Column}{Text\\Text\\Text Text Text}
\note{Note with default behavior}
\note[targetoffsetx=12cm, targetoffsety=-1cm, angle=20, rotate=25]
{Note \\ offset and rotated}

 % First column - second block
\block{Block titles with enough text will automatically obey spacing requirements }
{Text\\Text}

 % First column - third block
\block{Sample Block 4}{T\\E\\S\\T}

 % SECOND column
\column{0.4}
 %Second column with first block's top edge aligned with with previous column's top.

 % Second column - first block
\block[titleleft]{Smaller Column}{Test}

 % Second column - second block
\block[titlewidthscale=0.6, bodywidthscale=0.8]
{Variable width title}{Block with smaller width.}

 % Second column - third block
\block{}{Block with no title}

 % Second column - A collection of blocks in subcolumn environment.
\begin{subcolumns}
    \subcolumn{0.27} \block{1}{First block.} \block{2}{Second block}
    \subcolumn{0.4} \block{Sub-columns}{Sample subblocks\\Second subcolumn}
    \subcolumn{0.33} \block{4}{Fourth} \block{}{Final Subcolumn block}
\end{subcolumns}

 % Bottomblock
\block{Final Block in column}{
    Sample block.
}
\end{columns}
\block[titleleft, titleoffsetx=2em, titleoffsety=1em, bodyoffsetx=2em,%
 bodyoffsety=-2cm, roundedcorners=10, linewidth=0mm, titlewidthscale=0.7,%
 bodywidthscale=0.9, bodyverticalshift=2cm, titleright]
{Block outside of Columns}{Along with several options enabled}

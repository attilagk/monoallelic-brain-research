% You can submit either a single PDF file that includes the manuscript text
% and any display items, or separate files for text, figures and tables.
%
% 1500 words, excluding the introductory paragraph, online Methods,
% references and figure legends
%
% TODOs
% - cover letter

\documentclass[letterpaper]{article}
%\documentclass[12pt,letterpaper]{article}
%\setlength{\textwidth}{480pt}
%\setlength{\textheight}{630pt}
%\setlength{\voffset}{0pt}

\usepackage{amsmath, geometry, graphicx}
%\usepackage{float}
\bibliographystyle{plain}

% https://tex.stackexchange.com/questions/6758/how-can-i-create-a-bibliography-like-a-section
%\usepackage{etoolbox}
%\patchcmd{\thebibliography}{\section*}{\section}{}{}

\pagestyle{plain}

\title{Unperturbed Expression Bias of Imprinted Genes in Schizophrenia}

\author{Attila Guly\'{a}s-Kov\'{a}cs\(^{1,2,\ddagger}\), Ifat Keydar\(^{1,2,8,\ddagger}\), \\
Eva Xia\(^{1,3}\), Menachem Fromer\(^{2,4,9}\), Gabriel Hoffman\(^{2}\), Douglas
Ruderfer\(^{2,4,10}\), \\
CommonMind Consortium\(^{\S}\), Ravi Sachidanandam\(^{5}\), \\
Andrew Chess\(^{1,2,6,7,\ast}\)}

\date{Icahn School of Medicine at Mount Sinai (ISMMS)}

\begin{document}

\maketitle

\begin{description}
\item[1] Department of Cell, Developmental and Regenerative Biology, ISMMS 
\item[2] Institute for Genomics and Multiscale Biology, Department of Genetics and Genomic Sciences, ISMMS 
\item[3] Neuroscience Program, The Graduate School of Biomedical Sciences, ISMMS 
\item[4] Division of Psychiatric Genomics, Department of Psychiatry, ISMMS 
\item[5] Department of Oncological Sciences, ISMMS 
\item[6] Fishberg Department of Neuroscience, ISMMS 
\item[7] Friedman Brain Institute, ISMMS 
\item[8] Present affiliation: The Simon And Katya Michaeli Bioinformatics
Laboratory For The Research Of The Genome Department of Human Molecular
Genetics \& Biochemistry, Sackler Medical School, Tel Aviv University
\item[9] Present affiliation: Verily Life Sciences
\item[10] Present affiliation: Division of Genetic Medicine, Departments of
Medicine, Psychiatry and Biomedical Informatics, Vanderbilt University
\item[\(\ddagger\)] equal contribution 
\item[\(\S\)] full list of consortium members appears in the Author
Information section
\item[\(\ast\)] correspondence: andrew.chess@mssm.edu 
\end{description}

\clearpage

\section*{Abstract}

How gene expression correlates with schizophrenia across individuals is
beginning to be examined through analyses of RNA-seq from post-mortem brains
of individuals with disease and control brains.  Here we focus on variation in
allele-specific expression, following up on the CommonMind Consortium (CMC)
RNA-seq experiments of nearly 600 human dorsolateral prefrontal cortex (DLPFC)
samples.  Analyzing the extent of allelic expression bias---a hallmark of
imprinting---we find that the number of imprinted human genes is consistent
with lower estimates (\(\approx 0.5 \%\) of all genes) and thus contradicts
much higher estimates. Moreover, the handful of putatively imprinted
genes are all in close genomic proximity to known imprinted genes.
Joint analysis of the imprinted genes across hundreds of individuals allowed
us to establish how allelic bias depends on various factors.  We find that age
and genetic ancestry have gene-specific, differential effect on allelic bias.
In contrast, allelic bias appears to be independent of schizophrenia.  

\section*{Introduction}

The observation~\cite{Noor2015,Rees2014} that maternally derived
microduplications at 15q11-q13---harboring the imprinted gene UBE3A---may not
only cause Prader-Willi syndrome, but are also highly penetrant for
schizophrenia has raised the possibility that perturbation of regulation of
imprinted genes in general may play a role in psychotic disorders.  As it is
known that the extent of imprinting of individual genes varies over different
tissues we chose to analyze the DLPFC region, which controls complex cognitive and
executive functions and is known to display functional abnormalities in
schizophrenia.

A related question is the number of imprinted genes in the human brain.  Some
1,300 genes were estimated to be imprinted in the mouse
brain~\cite{Gregg2010a} but followup studies using mouse or human subjects
arrived at estimates that are lower with an order of
magnitude~\cite{Andergassen2017,Babak2015,Baran2015,DeVeale2012,Perez2015}.

We obtained DLPFC RNA-seq data from the CMC~\cite{Fromer2016a}
(http://www.synapse.org/CMC) and analyzed allele-specific
expression with the idea of (i) identifying imprinted genes in the adult human
brain and (ii) explaining the variability in allelic bias across 579
individuals in terms of their psychiatric diagnosis, age at death, etc.  This
was facilitated by the balanced case-control groups (258 SCZ, 267 Control, 54
bipolar or other affective/mood disorder, AFF) and the large age variability
in the cohort.

\section*{Results}

\subsection*{Identification of imprinted genes in the adult human brain}

For each individual \(i\) and gene \(g\) we quantified allelic bias based on
RNA-seq reads using a statistic called \emph{read count ratio} \(S_{ig}\)
(Fig.~1, Methods), which ranges from 0.5 to 1 indicating unbiased
biallelic expression (at 0.5), some allelic bias (at intermediate values) or
strictly monoallelic expression (at 1).  We corrected for a number
of factors this approach is known to be sensitive to.  We quality-filtered
RNA-seq reads and helped distinguish allele-specific reads using DNA
genotyping data before calculating \(S\) and then applied post hoc corrections
for mapping bias (Methods).

Of \(15584\) genes with RNA-seq data \(5307\) genes passed our filters designed to remove genes with
scarce RNA-seq data reflecting low expression and/or low coverage of
RNA-seq (Methods: ``Quality filtering'').
Fig.~2 presents the conditional empirical distribution
of \(S_{\cdot g}\) across all individuals given each gene \(g\).  The observed wide
\(S_{\cdot g}\) distributions suggest large across-individuals variation of allelic bias for all
genes, even if a substantial component of the \(S_{\cdot g}\) variation originates from technical sources.
Still, as expected, for many genes known to be imprinted in mice or in other human tissues
(referred to as \emph{known imprinted} genes like PEG10, ZNF331) the distribution of \(S_{\cdot g}\) was shifted
to the right signaling strong allelic bias (Fig.~2, upper
half).

To identify imprinted genes in the human adult DLPFC we defined the score of
each gene \(g\) as the fraction of individuals \(i\) for whom \(S_{ig}>0.9\).
We ranked all 5307 genes according to their score
(Fig.~2 bottom right).  An alternative definition of the
score, \(S_{ig}>0.7\), yielded similar ranking (Supplementary Fig.~11). The
heat map of the \(S_{\cdot g}\) distribution for ranked genes
(Fig.~2, lower left) shows that the top \(50\) genes,
which constitute \(\approx 1\%\) of all genes in our analysis, are
qualitatively different from the bottom \(\approx 99\%\) exhibiting strongly
right-shifted distribution of \(S_{\cdot g}\) characteristic to imprinting.

29 of the top-scoring 50 genes fell into previously described imprinted gene
clusters (Supplementary Fig.~1); 21 of these 29 are \emph{known imprinted}
genes while 8 are \emph{nearby candidates} defined as genes near
(\(<1\mathrm{Mb}\) of) \emph{known imprinted} ones but themselves previously
not shown to be imprinted (blue and green \(y\)-axis labels in
Fig.~3).   \emph{A priori} the expectation is that
\emph{known imprinted} genes and \emph{nearby candidates} are much more likely
to be imprinted  in the present data set than \emph{distant candidates}
defined as genes that neither belong to nor localize near \emph{known
imprinted} genes (Fig.~3, red \(y\)-axis labels).  We
combined this prior expectation with two tests based on our data to
distinguish imprinting from alternative causes of high read count ratio such
as mapping bias and cis-eQTL effects~\cite{Babak2015} (see Methods:
``Reference/non-reference allele test'' and ``Test for nearly unbiased allelic
expression'').  The results of both tests (X's and black bars,
Fig.~3) agreed well with the \emph{a priori} expected
status.  This prompted us to call imprinted in the adult human DLPFC those
genes in the top 50 that are either
\emph{known imprinted} or \emph{nearby candidates}.
We included also the \emph{known imprinted} gene UBE3A, which ranked below 50
but whose score was still substantial (Supplementary Fig.~2) yielding 30
imprinted genes (panel headers in Fig.~4-5).

\subsection*{Explaining the variability in allelic bias of imprinted genes}

Getting at the central question of our work Fig.~4 shows
that read count ratio is similarly distributed in the Control, SCZ and AFF
group for all 30 imprinted genes suggesting independence between allelic bias and diagnosis of
schizophrenia.  Similar pattern was observed for not imprinted genes
(Supplementary Fig.~12).

To support the above qualitative result on imprinted genes, we fitted several
fixed and mixed effects models~\cite{Hoffman2016} that model the dependence of
read count ratio jointly on all explanatory variables (Methods: ``Statistical
models---informal overview'', and beyond).  Such joint
models can capture much of the complex pattern of dependencies in genomic data
including those we observed within and between technical and biological
explanatory variables (Supplementary Table~1,
Supplementary Fig.~3).  For both the fixed and mixed class we
selected the model that fitted the data the best (unlm.Q/wnlm.Q for both fixed
and mixed models,
Supplementary Fig.~6-8).
Fixed and mixed models also agreed qualitatively on gene-specific coefficients
reporting effects/dependencies
(Supplementary Fig~9-10). We based final
inference on the selected mixed model because that gains power from letting
genes ``borrow strength from each other'' (Supplementary Fig.~5).

Based on the best fitting mixed model (henceforth ``the model'') we could
formally reject the hypotheses that read count ratio depends on diagnosis as
either main effect or interaction (see term \((1\mid\mathrm{Dx})\) and
\((1\,|\,\mathrm{Dx}:\mathrm{Gene})\) in Table~1,
respectively).  This key result is not due to low power. This is because in
the mixed model the Gene variable (which identifies the gene that a particular
data point corresponds to) is similar to the Dx variable (reporting on disease
status) in that they are both categorical and are modeled as random effects.
If Dx had an effect size that is comparable to the effect of Gene than that
effect would be detected by our model based inference since the effect of Gene
is highly significant. See \((1\mid\mathrm{Gene})\) in Table~1
and compare panels in Fig.~4).

Scatter plots suggested that the read count ratio depends negatively on age
for some imprinted genes, depends positively for others, and is independent of
age for the rest of imprinted genes (Fig.~5
and~Supplementary Fig.~4).  This apparent dependence might be indirect, i.e.~one
that is mediated by some variable(s) ``inbetween'' age and read count ratio
(Supplementary Fig.~3,~5) but the
model allowed us to isolate the direct component of age dependence: we found
that the gene-specific random age effect is indeed significant even if no
fixed effect---which would be shared by all imprinted genes---was supported
(see \((\mathrm{Age}\,|\,\mathrm{Gene})\) and \(\mathrm{Age}\), respectively,
in Table~1).

Based on the model we also predicted
gene-specific regression coefficients mediating the direct component of age
effect (Supplementary Fig.~10 top middle).  The predicted coefficients
agreed well with all but a few panels of Fig.~5 the latter of
which (e.g.~UBE3A) therefore represent purely indirect dependence.

The same type of analysis on the effects of ancestry principal components and
gender gave similar results: while the fixed effect, shared by all genes, of
these variables was negligible, three of the random, gene-specific, effects
received significant support.  These three, ordered by decreasing
statistical significance, are
\((\mathrm{Ancestry.1}\,|\,\mathrm{Gene})\),
\((\mathrm{Ancestry.3}\,|\,\mathrm{Gene})\) and
\((1\,|\,\mathrm{Gender}:\mathrm{Gene})\) (Table~1).  The
corresponding predicted random coefficients are presented in
Supplementary Fig.~10.

Although some of the random effects described above are statistically
significant (Table~1), their size is relatively subtle (Fig.~5).  Nonetheless,
we focussed on genes with very strong allelic bias, in other words nearly
monoallelic expression, because that is a hallmark of imprinted genes.  In
these genes even a subtle change in allelic bias may lead to a qualitative
change that carries some biological significance.

Finally, we fitted the same mixed models to two subsets of data, each
containing only 15 genes.  The results (Supplementary Table~3) are
qualitatively similar to those based on the full data set with 30 genes apart
from a few marked differences that are explained by the reduction in both the
number of data points and in the variability of certain effects across genes.

In summary age, ancestry, and to a lesser extent gender, are suggested by our
model-based analysis to exert effect on allelic bias in a way that the
direction and magnitude of the effect varies across genes.

\section*{Discussion}

The number of imprinted genes in the mammalian brain has been controversial:
some early genome wide studies~\cite{Gregg2010a,Gregg2010} estimated over a
thousand, suggesting that the number of imprinted genes in the brain is an
order of magnitude greater than in other tissues.  Later work cast doubt on
the methodology used and found that the number of imprinted genes in brain is
in line with expectations from studies of other tissues, identifying only a
handful of new candidate imprinted genes in brain~\cite{Baran2015,DeVeale2012,Perez2015}.
Based on 579 postmortem human DLPFC
samples we find evidence supporting only a handful of novel imprinted genes
all of which reside in genomic locations nearby to known imprinted genes.
Thus our results support those more recent studies that found no large excess
of imprinted genes in the brain.

The large size of our sample and the case-control makeup allowed us to explore
the potential for correlation of extent of imprinting in the DLPFC with
schizophrenia.  Although our approach gave strong support for dependence of
imprinting on age and ancestry, no dependence on schizophrenia was detected
either when we assumed that the dependence is the same for all imprinted genes
or that it varies across genes.  Thus our data indicate that imprinting in the
DLFPC does not play a significant role in schizophrenia in contradiction of
the ``imprinted brain'' hypothesis~\cite{Crespi2008}.  Given the complex
genetic architecture of schizophrenia~\cite{Sullivan2012} as well as technical
noise in postmortem brain RNA studies there could still be some correlation of
the extent of imprinting and schizophrenia.

We found that imprinting depends on ancestry in a gene
specific manner but the type of dependence that is shared by all imprinted
genes was not supported.  This is expected because the studied
ancestry variables must incorporate some of the cis expression
QTLs in imprinted genes such that those eQTLS perturb allelic bias in a gene
specific manner.

Our finding that imprinting depends on age in later adulthood is rather
intriguing given the quantitative relationship between epigenetics and
aging~\cite{Horvath2013}.  Age dependence of imprinting through early
postnatal life supported experimentally~\cite{Perez2015} but such dependence
during later adulthood has so far only been predicted~\cite{Ubeda2012} based
on a hypothesis that links ``genomic imprinting and the social
brain''~\cite{Isles2006}.  Previous genomics studies~\cite{Baran2015} were
statistically underpowered to address this question in humans.  Although our
age-related finding supports the ``social brain'' hypothesis, it leaves the
possibility open that the observed age related changes indicate merely the
loss of tight regulation of those genes with aging.



\section*{Methods}

\subsection*{Defining the read count ratio to quantify allelic bias}

We quantified allelic bias based on RNA-seq reads using a statistic called
\emph{read count ratio} \(S\), whose definition we
based on the total read count \(T\) and the \emph{higher read count} \(H\),
i.e.~the count of reads carrying only either the reference or the alternative SNP variant,
whichever is higher.  The
definition is
\begin{equation}
S_{ig} = \frac{H_{ig}}{T_{ig}}= \frac{\sum_s H_s}{\sum_sT_s},
\label{eq:S-definition}
\end{equation}
where \(i\) identifies an individual, \(g\) a gene, and the summation runs
over all SNPs \(s\) for which gene \(g\) is heterozygous in individual \(i\) (Fig.~1).
Note that if \(B_{ig}\) is the count or reads that map to the \(b_{ig}\) allele
(defined as above) and if we make the same distributional assumption as above, namely that \(B_{ig}\sim
\mathrm{Binom}(p_{ig}, T_{ig})\), then \(\mathrm{Pr}(H_{ig}=B_{ig}|p_{ig})\), the probability of correctly
assigning the reads with the higher count to the allele towards which
expression is biased, tends to 1 as \(p_{ig} \rightarrow 1\).  We took
advantage of this theoretical result in that we subjected only those genes to
statistical inference, whose read count ratio was found to be high and,
therefore, whose \(p_{ig}\) is expected to be high as well.

Fig.~1 illustrates the calculation of \(S_{ig}\) for the
combination of two hypothetical genes, \(g_1,g_2\), and two individuals,
\(i_1,i_2\).  It also shows an example for the less likely event that the lower rather
than the higher read count corresponds to the SNP variant tagging the higher
expressed allele (see SNP \(s_3\) in gene \(g_1\) in individual \(i_2\)).

Before we carried out our read count ratio-based analyses, however, we cleaned
our RNA-seq data by quality-filtering and by improving the accuracy of SNP
calling with the use of DNA SNP array data and imputation. In the following
subsections of Methods we describe the data, these procedures, as well as our
regression models in detail.

\subsection*{Brain samples, RNA-seq}

Human RNA samples were collected from the dorsolateral prefrontal cortex of
the CommonMind consortium from a total of \(579\) individuals after
quality control. Subjects included 267 control individuals, as well as 258
with schizophrenia (SCZ) and 54 with affective spectrum disorder (AFF).
RNA-seq library preparation uses Ribo-Zero (which selects against ribosomal
RNA) to prepare the RNA, followed by Illumina paired end library generation.
RNA-seq was performed on Illumina HiSeq 2000.

\subsection*{Mapping, SNP calling and filtering}

We mapped 100bp, paired-end RNA-seq reads (\(\approx50\) million reads per sample) using Tophat
to Ensembl gene transcripts of the human genome (hg19; February, 2009) with
default parameters and 6 mismatches allowed per pair (200 bp total). We
required both reads in a pair to be successfully mapped and we removed reads
that mapped to \(>1\) genomic locus. Then, we removed PCR replicates using the
Samtools rmdup utility; around one third of the reads mapped (which is
expected, given the parameters we used and the known high repeat content of
the human genome). We used Cufflinks to determine gene expression of Ensembl
genes, using default parameters. Using the BCFtools utility of Samtools, we
called SNPs (SNVs only, no indels). Then, we invoked a quality filter
requiring a Phred score \(>20\) (corresponding to a probability for an
incorrect SNP call \(<0.01\)).

We annotated known SNPs using dbSNP (dbSNP 138, October 2013). Considering all
579 samples, we find 936,193 SNPs in total, 563,427 (60\%) of which are novel.
Further filtering of this SNP list removed the novel SNPs and removed SNPs
that either did not match the alleles reported in dbSNP or had more than 2
alleles in dbSNP. We also removed SNPs without at least 10 mapped reads in at
least one sample. Read depth was measured using the Samtools Pileup utility.
After these filters were applied, 364,509 SNPs remained in 22,254 genes. These
filters enabled use of data with low coverage.  For the 579
samples there were 203 million reads overlapping one of the
364,509 SNPs defined above.  Of those 158 million (78\%) had genotype data
available from either SNP array or imputation.

\subsection*{Genotyping and calibration of imputed SNPs}

DNA samples were genotyped using the Illumina Infinium SNP array. We used
PLINK with default parameters to impute genotypes for SNPs not present on the
Infinium SNP array using 1000 genomes data.  We calibrated the
imputation parameters to find a reasonable balance between the number of genes
assessable for allelic bias and the number false positive
calls since the latter can arise if a SNP is
incorrectly called heterozygous.

We first examined how many SNPs were heterozygous in DNA calls and had a
discordant RNA call (i.e.~homozygous SNP call from RNA-seq) using different imputation
parameters. Known imprinted genes were excluded. We examined RNA-seq reads
overlapping array-called heterozygous SNPs which we assigned a heterozygosity
score \(L_\mathrm{het}\) of 1, separately from RNA seq data
overlapping imputed heterozygous SNPs, where the \(L_\mathrm{het}\) score could
range from 0 to 1.  After testing different thresholds
we selected an \(L_\mathrm{het}\) cutoff of 0.95 (i.e. imputation confidence
level of 95\%), and a minimal coverage of 7 reads per SNP. With these
parameters, the discordance rate (monoallelic RNA genotype in the context of a
heterozygous DNA genotype) was 0.71\% for array-called SNPs and 3.2\% for
imputed SNPs.

The higher rate of discordance for the imputed SNPs
is due to imputation error.  These were taken into
account in two ways.
First, we considered all imputed SNPs for a gene \(g\) and individual \(i\)
jointly.  Second, we excluded
any individual, for which one or more SNPs supported biallelic
expression.

%At this point, the matrix includes 147
%million data points covering 213,208 SNPs, of which 114 million (77\%) have
%imputation data.

\subsection*{Quality filtering}

\label{sec:filtering}

Two kind of data filters were applied sequentially: (1) a \emph{read
count-based} and (2) an \emph{individual-based}.  The read count-based filter
removes any such pair $(i,g)$ of individual $i$ and genes $g$ for which the
total read count $T_{ig}<t_\mathrm{rc}$, where the read count threshold
$t_\mathrm{rc}$ was set to 15. The individual-based filter removes any genes
$g$ (across all individuals) if read count data involving $g$ are available
for less than $t_\mathrm{ind}$ number of individuals, set to 25.
These final filtering procedures decreased the number of genes in the data from
\(15584\) to \(n=5307\).

\subsection*{Reference/non-reference allele test to correct for mapping bias
and eQTLs}

We designed this test to distinguish imprinting from alternative causes of
high read count ratio (Fig.~3): mapping bias or cis-eQTL
effects.  For any given gene this is a possibly compound test since there may
be multiple SNPs that are informative for the read count ratio (see Defining
the read count ratio above).

For a given gene the compound null hypothesis is that the observed high read
count ratio is due only to imprinting.  For each informative SNP this
hypothesis means that the reference and non-reference allele are associated
with equal probability to the \emph{higher read count}~\cite{Babak2015} (see
Methods: ``Defining the read count ratio to quantify allelic bias'').  Thus
for each SNP we assumed that the number of individuals for whom the reference
allele is associated to the \emph{higher read count} is binomially distributed
with probability parameter 0.5.  The we calculated the fraction of informative
SNPs for which the null hypothesis can be rejected at 0.05 significance level
and used this information to decide whether the compound null hypothesis for
the gene itself can be rejected.

\subsection*{Test for nearly unbiased allelic expression}

The null hypothesis of this test is that the higher read count
\(H_{ig}=S_{ig}T_{ig}\) for gene \(g\) and individual \(i\) is drawn from a
binomial distribution with a probability parameter \(p_{ig}\approx 0.5\)
suggesting nearly unbiased allelic expression.  More specifically, the test
was defined by the criteria
\begin{equation}
S_{ig} \le 0.6 \text{ and } \mathrm{UCL}_{ig} \le 0.7,
\label{eq:unbiased-test}
\end{equation}
where the 95\% upper confidence limit \(\mathrm{UCL}_{ig}\) for the expected
read count ratio \(p_{ig}\) was calculated assuming that the higher read count
\(H_{ig}\sim \mathrm{Binom}(p_{ig}, T_{ig})\), on the fact that binomial
random variables are asymptotically (as \(T_{ig}\rightarrow \infty\)) normal
with \(\mathrm{var}(H_{ig}) = T_{ig}p_{ig}(1-p_{ig})\), and on the equalities
\(\mathrm{var}(S_{ig}) = \mathrm{var}(H_{ig}/T_{ig}) =
\mathrm{var}(H_{ig})/T_{ig}^2\).  Therefore
\begin{equation}
\mathrm{UCL}_{ig} = S_{ig} + z_{0.975} \sqrt{\frac{S_{ig} (1 - S_{ig})}{T_{ig}}},
\end{equation}
where $z_{p}$ is the $p$ quantile of the standard normal distribution.

\subsection*{Data transformations}

We found transformations of the read count ratio data to be useful for fitting
our statistical models (Methods: ``Statistical models---informal overview'',
and beyond).  We used either (or none) of the following two transformations:

\begin{enumerate}
\item
The quasi-log transformation, defined as
\begin{equation}
\tau_Q(S_{ig};T_{ig}) \equiv Q_{ig} = - \log \left( 1 - S_{ig} \frac{T_{ig}}{T_{ig} + c}
\right),
\label{eq:Q}
\end{equation}
where \(S_{ig}\) and \(T_{ig}\) mean read count ratio and total read count for
individual \(i\) and gene \(g\);
\(\log\) means natural logarithm (base \(e\));  \(c\) is a pseudo read
count set to \(1\) in order to avoid zero in the parenthesis since the \(\log\)
function is undefined at \(0\).
\item 
The rank transformation
\begin{equation}
\tau_R(S_{ig};\{S_{jg}\}_j) \equiv R_{ig} = \frac{\# \{j: S_{jg}\le S_{ig}
\}}{\# j} \times 100.
\label{eq:R}
\end{equation}
Note that \(j\) may equal \(i\) in Eq.~\ref{eq:R}.
Thus, this transformation first ranks individual \(i\) among all individuals
\(j\) and then scales the ranks between 0 and 100.
\end{enumerate}

\subsection*{Statistical models---informal overview}
\label{sec:regression-overview-informal}

We modeled the dependence of read count ratio of imprinted genes jointly on
all biological and technical explanatory variables
(Supplementary Table~1) using several multiple regression models.  Based
on their structure our models can be classified into two sets of fixed and a
set of mixed regression models (Supplementary Fig.~5).  Furthermore
our models can be also classified based on non-structural properties (link
function, error distribution, weighting, and the data transformation to read
count ratio; see model classes in Supplementary Table~2).

Supplementary Fig.~5 explains that among the two fixed and the mixed
structural model class the mixed one is both more powerful and robust because
its random effects terms allow gene-specific parts of the model to ``borrow
strength from each other''.  The cost of the enhanced power in mixed models is
the lack of estimates and confidence intervals (as well as p-values) for
gene-specific coefficients (parameters), which the less powerful fixed models
do provide (Supplementary Fig.~9).  Instead of being estimated,
gene-specific coefficients in mixed models therefore can only be
\emph{predicted} without information on confidence surrounding them (see
Supplementary Fig.~10 for predicted gene-specific coefficients under a
mixed model).  Nonetheless, the low power and low robustness of fixed models
became apparent from results like those in Supplementary Fig.~9 so
we based our final inference (Table~1) on mixed modeling.
Note, however, that we found an overall qualitative agreement between mixed and
fixed models regarding gene-specific coefficients (compare
Supplementary Fig~9 and 10).

To select the best model within the mixed structural class we compared model
fit of the non-structural types (Supplementary Fig.~8) and found
that the unlm.Q and wnlm.Q types fitted the data the best.  Similar results
were obtained for fixed models
(Supplementary Fig.~6,~7).

\subsection*{Statistical models---formal overview}
\label{sec:regression-overview-formal}

Our fixed and mixed effects multiple regression models are all generalized
linear models (GLMs).  GLMs in general describe a conditional distribution of
a response variable \(y\) given a linear predictor \(\eta\) such that the
distribution is from the exponential family and that \(\mathrm{E}(y|\eta) =
g^{-1} (\eta)\), where \(g\) is some link function.  In the present context
the response \(y\) is the observed read count ratio that is possibly
transformed to improve the model's fit to the data.  We performed fitting with
the lme4 and stats R packages and tested several combinations of
transformations, link functions, and error distributions
(Supplementary Table~2).  For inference we used the best fitting
combination (\(\mathrm{unlm.Q}\), Supplementary Table~2) as assessed by
the normality and homoscedasticity of residuals (Supplementary Fig.~8,
also Supplementary Fig.~6,~7) as well
as by monitoring convergence.

In mixed GLMs the linear predictor \(\eta = X \beta + Z b\) and in fixed GLMS
\(\eta = X \beta\), where \(X, Z\) are design matrices containing data on
explanatory variables whereas \(\beta \) and \(b\) are fixed and random
vectors of regression coefficients that mediate fixed and random effects,
respectively (see Methods: ``Detailed syntax and semantics of mixed models''
and Supplementary Fig.~5 for details).

Besides the random effects term \(Zb\) the key difference between
the mixed and fixed models in this study is that the former describes
read count ratio \emph{jointly} for all imprinted genes and the latter
\emph{separately} for each imprinted gene.  An important consequence is that
our mixed models are more powerful because they can utilize 
information shared by all genes.  Therefore we preferred mixed models for final
inference and used fixed models only to guide selection among possible mixed
models (Methods: ``Model fitting and selection'').

\subsection*{Detailed syntax and semantics of mixed models}
\label{sec:mixed-mod}

Here we describe the detailed syntax and semantics of the normal linear mixed
models combined with a quasi-log transformation \(Q\) of read count ratio as
this combination was found to provide the best fit
(Supplementary Fig.~8).  We have data
on \(579\) individuals and \(30\) imprinted genes and so the response vector is
\(y=(Q_{i_1g_1},...,Q_{i_{579}g_1},Q_{i_1g_2},...,Q_{i_{579}g_2},...,Q_{i_1g_{30}},...,Q_{i_{579}g_{30}})\).
The model in matrix notation is
\begin{eqnarray}
\label{eq:mixed-mod-matrix}
y &=& X \beta + Z b + \varepsilon \\
\varepsilon_i &\overset{\text{i.i.d.}}{\sim}& \mathcal{N}(0, \sigma^2),\;
i=1,...,mn \\
b &\sim& \mathcal{N}(0, \Omega_b),
\end{eqnarray}
where the size of the covariance matrix \(\Omega_b\) depends on the number of
terms with random effects (the columns of \(Z\)).  Simply put: errors and
random coefficients are all normally distributed.

To clarify the semantics of Eq.~\ref{eq:mixed-mod-matrix} let us consider a
simple toy model with just a few terms in the linear predictor.  But before
expressing it in terms of Eq.~\ref{eq:mixed-mod-matrix} it is easier
to cast it in the compact ``R formalism'' of the stats and lme4 packages of the R
language as
\begin{equation}
\label{eq:toy-mod-r}
y \sim \overbrace{1 + \mathrm{Age}}^{\text{fixed effect}} +
\overbrace{\underbrace{(1 + \mathrm{Age} + \mathrm{Ancestry.1} \,|\,
\mathrm{Gene})}_{k=1} +
\underbrace{(1 \,|\, \mathrm{Dx}:\mathrm{Gene})}_{k=2}}^{\text{random
effects}}.
\end{equation}

First note that the random effect term labeled with \(k=1\) can be expanded
into \((1 \,|\, \mathrm{Gene}) + (\mathrm{Age}\,|\, \mathrm{Gene}) +
(\mathrm{Ancestry.1} \,|\, \mathrm{Gene})\).  The `\(1\)'s mean intercept
terms: one as a fixed effect and two as random effects.  The first random
intercept term \((1\,|\,\mathrm{Gene})\) expresses the gene-to-gene
variability in read count ratio (compare panels in Fig.~4
and~5), in other words the random effect of the
\(\mathrm{Gene}\) variable.  The second random intercept term
\((1\,|\,\mathrm{Dx}:\mathrm{Gene})\) corresponds to the interaction between
psychiatric diagnosis \(\mathrm{Dx}\) and \(\mathrm{Gene}\); it can be
interpreted as the \(\mathrm{Gene}\) specific effect of \(\mathrm{Dx}\)
or---equivalently---as \(\mathrm{Dx}\) specific gene-to-gene
variability.  This term is not likely to be informative as Fig.~4
suggests little \(\mathrm{Gene}\) specific effect of \(\mathrm{Dx}\).

We see that \(\mathrm{Age}\) appears twice: first as a fixed slope effect on \(y\) and
second as a \(\mathrm{Gene}\) specific random slope effect, denoted as
\((\mathrm{Age}\,|\,\mathrm{Gene})\).  The random effect appears to be
supported by Fig.~5 because the dependence of read count ratio
on \(\mathrm{Age}\) varies substantially among genes but the fixed effect is not
supported because the negative dependence seen for several genes is balanced
out by the positive dependence seen for others.  The model includes another
random slope effect: \((\mathrm{Ancestry.1}\,|\,\mathrm{Gene})\) with a
similar interpretation as \((\mathrm{Age}\,|\,\mathrm{Gene})\) but lacks a
fixed effect of \(\mathrm{Ancestry.1}\).

Now we are ready to write the toy model as an expanded special case of
Eq.~\ref{eq:mixed-mod-matrix} as
\begin{equation}
\label{eq:toy-mod-math}
y_{i} = \overbrace{\beta_0 + \mathrm{Age}_{i}
\beta_1}^{\text{fixed effects}} +
\overbrace{\underbrace{b_{0}^{(1)} + \mathrm{Age}_i b_1^{(1)} +
\mathrm{Ancestry.1}_i b_{2}^{(1)}}_{\mathrm{Gene}_i} +
\underbrace{b_{0}^{(2)}}_{\mathrm{Dx}_i:\mathrm{Gene}_i}}^{\text{random
effects}} + \varepsilon_i.
\end{equation}

As in the earlier R formalism the terms of the linear predictor
are grouped into fixed and random effects.  Within the latter group we have
two batches of terms indicated by the \(k\) superscripts on the random
regression coefficients \(b_j^{(k)}\).  The first batch
\(\{b_0^{(1)},b_1^{(1)},b_2^{(1)}\}\) corresponds to
\(\{(1\,|\,\mathrm{Gene}), (\mathrm{Age}\,|\,\mathrm{Gene}),
(\mathrm{Ancestry.1}\,|\,\mathrm{Gene})\}\) in Eq.~\ref{eq:toy-mod-r}, the
second batch contains only \(b_0^{(2)}\) corresponding to
\((1\,|\,\mathrm{Dx}:\mathrm{Gene})\).

Within the \(k\)th batch Eq.~\ref{eq:toy-mod-math} contains only a single
intercept coefficient \(b_0^{(k)}\) and, if random slope terms are also
present in the batch, only a single slope coefficient associated with the
variable \(\mathrm{Age}\) or \(\mathrm{Ancestry.1}\).  This is because only a
single level of the factor \(\mathrm{Gene}\) or the composite factor
\(\mathrm{Dx}:\mathrm{Gene}\) needs to be considered for the \(i\)th
observation; these levels are denoted as \(\mathrm{Gene}_i\) and
\(\mathrm{Dx}_i:\mathrm{Gene}_i\), respectively.  Implicitly however,
Eq.~\ref{eq:toy-mod-math} contains the respective coefficients for all levels
of these factors.  For example, there are \(n=30\) intercept coefficients
\(b_j^{(1)}\) each of which corresponds to a given gene.  So to generalize
Eq.~\ref{eq:toy-mod-math} we need \(J_k\) coefficients in the \(k\)th batch,
where \(J_k\) is the product of the number of factor levels and one plus the
number of random slope variables.  This way we can provide the expansion of
the general formula Eq.~\ref{eq:mixed-mod-matrix} using the semantics of the
toy model (Eq.~\ref{eq:toy-mod-r},~\ref{eq:toy-mod-math}) as
\begin{equation}
y_i = \overbrace{\sum_{j=0}^J x_{ij} \beta_j}^{\text{fixed effects}} +
\overbrace{\sum_{k=1}^K \sum_{j=0}^{J_k} z_{ij}^{(k)}
b_{j}^{(k)}}^{\text{random effects}} + \varepsilon_i.
\label{eq:mixed-mod-general}
\end{equation}

\subsection*{Model fitting and selection}
\label{sec:fitting}

Eq.~\ref{eq:mixed-mod-general} describes a large set of mixed models that
differ in one or more individual terms that constitute their linear predictor.
From this set we aimed to select the best fitting model under the Akaike
Information Criterion (AIC).

We used a heuristic search strategy in order to restrict
the vast model space to a relatively small subset of plausible models.  The
search was started at a model whose relatively simple linear predictor was composed of terms
using our prior results based on fixed effects models.  The same results
suggested a sequence in which further terms were progressively added to the
model to test if they improve fit.  Improvement was assessed by \(\Delta
\mathrm{AIC}\) and the \(\chi^2\)-test on the degrees of freedom that
correspond the evaluated term.  If fit improved the term was added otherwise
it was omitted.  Next, further terms were tested.  This iterative procedure
lead to the following model.
\begin{eqnarray*}
Q &\sim&
\mathrm{RIN} + (1 \,|\, \mathrm{RNA\_batch}) + (1 \,|\, \mathrm{Institution}) + (1 \,|\,
\mathrm{Institution}:\mathrm{Individual}) \\
&+& (1 \,|\, \mathrm{Gene}:\mathrm{Institution}) + (1 \,|\, \mathrm{Gender}:\mathrm{Gene}) \\
&+& (\mathrm{Age} + \mathrm{RIN} + \mathrm{Ancestry.1} + \mathrm{Ancestry.3} \,|\, \mathrm{Gene})
\end{eqnarray*}
We refer to this as the ``best
fitting model'' even thought it may represent only a local optimum in model
space.

\subsection*{Data availability}

Data and analytical results generated through the CommonMind Consortium are
available through the CommonMind Consortium Knowledge Portal:
doi:10.7303/syn2759792.  Intermediate results leading to the final results
published here are available from the authors at request.

\subsection*{Code availability}

All code developed by A.~Guly\'{a}s-Kov\'{a}cs is available at:\\
https://github.com/attilagk/monoallelic-brain-notebook

The corresponding lab
notebook can be browsed as a website at:\\
https://attilagk.github.io/monoallelic-brain-notebook

%\bibliography{library}
\begin{thebibliography}{10}

\bibitem{Noor2015}
Abdul Noor, Lucie Dupuis, Kirti Mittal, Anath~C. Lionel, Christian~R. Marshall,
  Stephen~W. Scherer, Tracy Stockley, John~B. Vincent, Roberto Mendoza-Londono,
  and Dimitri~J. Stavropoulos.
\newblock 15q11.2 duplication encompassing only the {UBE}3a gene is associated
  with developmental delay and neuropsychiatric phenotypes.
\newblock 36(7):689--693.

\bibitem{Rees2014}
Elliott Rees, James T~R Walters, Lyudmila Georgieva, Anthony~R Isles,
  Kimberly~D Chambert, Alexander~L Richards, Gerwyn Mahoney-Davies, Sophie~E
  Legge, Jennifer~L Moran, Steven~A McCarroll, Michael~C O'Donovan, Michael~J
  Owen, and George Kirov.
\newblock {Analysis of copy number variations at 15 schizophrenia-associated
  loci.}
\newblock {\em The British journal of psychiatry : the journal of mental
  science}, 204(2):108--14, feb 2014.

\bibitem{Gregg2010a}
Christopher Gregg, Jiangwen Zhang, Brandon Weissbourd, Shujun Luo, Gary~P
  Schroth, David Haig, and Catherine Dulac.
\newblock {High-resolution analysis of parent-of-origin allelic expression in
  the mouse brain.}
\newblock {\em Science (New York, N.Y.)}, 329(5992):643--8, aug 2010.

\bibitem{Andergassen2017}
Daniel Andergassen, Christoph~P Dotter, Daniel Wenzel, Verena Sigl, Philipp~C
  Bammer, Markus Muckenhuber, Daniela Mayer, Tomasz~M Kulinski, Hans-Christian
  Theussl, Josef~M Penninger, Christoph Bock, Denise~P Barlow, Florian~M
  Pauler, and Quanah~J Hudson.
\newblock Mapping the mouse allelome reveals tissue-specific regulation of
  allelic expression.
\newblock {\em eLife}, 6, August 2017.

\bibitem{Baran2015}
Yael Baran, Meena Subramaniam, Anne Biton, Taru Tukiainen, Emily~K. Tsang,
  Manuel~A. Rivas, Matti Pirinen, Maria Gutierrez-Arcelus, Kevin~S. Smith,
  Kim~R. Kukurba, Rui Zhang, Celeste Eng, Dara~G. Torgerson, Cydney Urbanek,
  Jin~Billy Li, Jose~R. Rodriguez-Santana, Esteban~G. Burchard, Max~A. Seibold,
  Daniel~G. MacArthur, Stephen~B. Montgomery, Noah~A. Zaitlen, and Tuuli
  Lappalainen.
\newblock {The landscape of genomic imprinting across diverse adult human
  tissues}.
\newblock {\em Genome Research}, 25(7), 2015.

\bibitem{DeVeale2012}
Brian DeVeale, Derek van~der Kooy, and Tomas Babak.
\newblock {Critical evaluation of imprinted gene expression by RNA-seq: A new
  perspective}.
\newblock {\em PLoS Genetics}, 8(3):e1002600, jan 2012.

\bibitem{Perez2015}
Julio~D Perez, Nimrod~D Rubinstein, Daniel~E Fernandez, Stephen~W Santoro,
  Leigh~A Needleman, Olivia Ho-Shing, John~J Choi, Mariela Zirlinger,
  Shau-Kwaun Chen, Jun~S Liu, and Catherine Dulac.
\newblock {Quantitative and functional interrogation of parent-of-origin
  allelic expression biases in the brain.}
\newblock {\em eLife}, 4:e07860, jan 2015.

\bibitem{Fromer2016a}
Fromer, Menachem and Roussos, Panos and Sieberts, Solveig K and Johnson,
Jessica S and Kavanagh, David H and Perumal, Thanneer M and Ruderfer, Douglas
M and Oh, Edwin C and Topol, Aaron and Shah, Hardik R and others.
\newblock {Gene expression elucidates functional impact of polygenic risk for
  schizophrenia.}
\newblock {\em Nature Neuroscience}, 19(11):1442, Sep 2016.

\bibitem{Babak2015}
Tomas Babak, Brian DeVeale, Emily~K Tsang, Yiqi Zhou, Xin Li, Kevin~S Smith,
  Kim~R Kukurba, Rui Zhang, Jin~Billy Li, Derek van~der Kooy, Stephen~B
  Montgomery, and Hunter~B Fraser.
\newblock Genetic conflict reflected in tissue-specific maps of genomic
  imprinting in human and mouse.
\newblock {\em Nature Genetics}, 47:544--549, May 2015.

\bibitem{Hoffman2016}
Gabriel~E Hoffman and Eric~E Schadt.
\newblock {variancePartition: interpreting drivers of variation in complex gene
  expression studies.}
\newblock {\em BMC Bioinformatics}, 17(1):483, nov 2016.

\bibitem{Gregg2010}
Christopher Gregg, Jiangwen Zhang, James~E. Butler, David Haig, and Catherine
  Dulac.
\newblock {Sex-Specific Parent-of-Origin Allelic Expression in the Mouse
  Brain}.
\newblock {\em Science}, 329(5992):682--685, aug 2010.

\bibitem{Crespi2008}
Bernard Crespi.
\newblock {Genomic imprinting in the development and evolution of psychotic
  spectrum conditions.}
\newblock {\em Biological reviews of the Cambridge Philosophical Society},
  83(4):441--93, nov 2008.

\bibitem{Sullivan2012}
Patrick~F Sullivan, Mark~J Daly, and Michael O'Donovan.
\newblock {Genetic architectures of psychiatric disorders: the emerging picture
  and its implications}.
\newblock {\em Nature Reviews Genetics}, 13(8):537--551, aug 2012.

\bibitem{Horvath2013}
Steve Horvath.
\newblock {DNA methylation age of human tissues and cell types.}
\newblock {\em Genome Biology}, 14(10):R115, oct 2013.

\bibitem{Ubeda2012}
Francisco Ubeda and Andy Gardner.
\newblock {A model for genomic imprinting in the social brain: elders.}
\newblock {\em Evolution; international journal of organic evolution},
  66(5):1567--81, may 2012.

\bibitem{Isles2006}
Anthony~R Isles, William Davies, and Lawrence~S Wilkinson.
\newblock {Genomic imprinting and the social brain.}
\newblock {\em Philosophical transactions of the Royal Society of London.
  Series B, Biological sciences}, 361(1476):2229--37, dec 2006.

\end{thebibliography}

\section*{Acknowledgements}

We thank Chaggai Rosenbluh, and Ephraim Kenigsberg for valuable feedback and discussions.

Data were generated as part of the CommonMind Consortium supported by funding
from Takeda Pharmaceuticals Company Limited, F. Hoffman-La Roche Ltd and NIH
grants R01MH085542, R01MH093725, P50MH066392, P50MH080405, R01MH097276,
RO1-MH-075916, P50M096891, P50MH084053S1, R37MH057881 and R37MH057881S1,
HHSN271201300031C, AG02219, AG05138 and MH06692. Brain tissue for the study
was obtained from the following brain bank collections: the Mount Sinai NIH
Brain and Tissue Repository, the University of Pennsylvania Alzheimer’s
Disease Core Center, the University of Pittsburgh NeuroBioBank and Brain and
Tissue Repositories and the NIMH Human Brain Collection Core. CMC Leadership:
Pamela Sklar, Joseph Buxbaum (Icahn School of Medicine at Mount Sinai), Bernie
Devlin, David Lewis (University of Pittsburgh), Raquel Gur, Chang-Gyu Hahn
(University of Pennsylvania), Keisuke Hirai, Hiroyoshi Toyoshiba (Takeda
Pharmaceuticals Company Limited), Enrico Domenici, Laurent Essioux (F.
Hoffman-La Roche Ltd), Lara Mangravite, Mette Peters (Sage Bionetworks),
Thomas Lehner, Barbara Lipska (NIMH).

\section*{Author Information}

\subsection*{CommonMind Consortium}

Barbara K Lipska\(^{11}\), Bernie Devlin\(^{12}\), Chang-Gyu Hahn\(^{13}\), David A Lewis\(^{12}\), Enrico
Domenici\(^{14}\), Eric Schadt\(^{15}\), Hardik R Shah\(^{15}\), Jessica S
Johnson\(^{4}\), Joseph D Buxbaum\(^{4}\),
Lambertus L Klei\(^{12}\), Mette A Peters\(^{16}\), Panos Roussos\(^{4}\),
Raquel E Gur\(^{13}\), Solveig K
Sieberts\(^{16}\), Thanneer M Perumal\(^{16}\), Vahram Haroutunian\(^{4}\)

\begin{description}
\item[11] Human Brain Collection Core, National Institutes of Health, NIMH, Bethesda, Maryland, USA
\item[12] Department of Psychiatry, University of Pittsburgh School of
Medicine, Pittsburgh, Pennsylvania, USADepartment of Human Genetics,
University of Pittsburgh, Pittsburgh, Pennsylvania, USA
\item[13] Neuropsychiatric Signaling Program, Department of Psychiatry,
Perelman School of Medicine, University of Pennsylvania, Philadelphia,
Pennsylvania, USA
\item[14] Laboratory of Neurogenomic Biomarkers, Centre  for Integrative
Biology (CIBIO), University of Trento, Trento, Italy
\item[15] Institute for Genomics and Multiscale Biology, Department of
Genetics and Genomic Sciences, Icahn School of Medicine at Mount Sinai, New
York, New York, USA
\item[16] Sage Bionetworks, Seattle, Washington, USA

\end{description}

\section*{Author Contributions}

A.G.K., I.K., A.C.~designed the study and interpreted results;  A.G.K.,
I.K.~implemented and performed statistical analysis; E.X.~performed molecular
genetics experiments; M.F., D.R.~were involved with the integration of other
CommonMind Consortium efforts with this project; G.H.~helped with multivariate
statistical modeling; R.S.~supported data storage and management; G.H., R.S.,
M.F., D.R.~gave feedback on the manuscript; A.G.K., A.C.~wrote the manuscript

\section*{Competing Interests}

None

\section*{Figure Legends}

\subsection*{Figure 1}
%\includegraphics[width=1.0\textwidth]{figures/by-me/commonmind-rna-seq-ms/commonmind-rna-seq-ms.pdf}

Quantifying allelic bias of expression in human
individuals using the RNA-seq read count ratio statistic \(S_{ig}\).
The strength of
bias towards the expression of the maternal (red) or paternal (blue) allele of
a given gene \(g\) in individual \(i\) is gauged with the count of RNA-seq
reads carrying the reference allele (small closed circles) and the count of
reads carrying an alternative allele (open squares) at all SNPs for which the
individual is heterozygous.  The allele with the \emph{higher read count} tends to
match the allele with the higher expression but measurement errors may
occasionally revert this tendency as seen for SNP \(s_3\) in gene \(g_1\) in
individual \(i_2\).


\subsection*{Figure 2}
%\includegraphics[scale=0.6]{figures/2016-07-19-genome-wide-S/complex-plot-1.png}

Across-individuals distribution of read count ratio \(S\) for each
gene indicates substantial variation of allelic bias and that \(<1\%\) of all
genes are imprinted.  The vertically arranged
four main panels present the empirical distribution of \(S_{\cdot g}\) across all
individuals given each gene \(g\).  The \emph{upper two panels} are distinct
representations (survival plot: \(1 - \mathrm{ECDF}\), and
``survival heatmap'') of the same three distributions corresponding to \(a\):
PEG10, \(b\): ZNF331, and \(c\): AFAP1.  PEG10 and ZNF331,
previously found to be imprinted in mice or in other human tissues, and one
for AFAP1, a gene without prior evidence.  The bottom two survival heatmaps present
the distribution of \(S_{\cdot g}\) for the top \(2\%\) and \(100\%\) of the
5307 analyzed genes.  These are ranked according to gene score defined as \(1
- \mathrm{ECDF}(0.9)\) in the \emph{bottom
far right panels}.  The score of PEG10, ZNF331, and AFAP1 is marked by
\(a,b,c\), respectively, in green circles.  As expected, PEG10 and ZNF331 both score
high and rank within the top 30 of all genes suggesting they are also imprinted in
the present context, the adult human DLFPC.  The bottom panels also indicate
that \(<1\%\) of all genes might be imprinted.

\subsection*{Figure 3}
%\includegraphics[scale=0.6]{figures/2016-08-01-ifats-filters/top-ranking-genes-1.pdf}

The top 50 genes ranked by the gene score defined, for gene \(g\), as \(1 -
F_g(0.9)\), where \(F_g\) is the empirical cumulative distribution function
(ECDF) for \(g\).  Thus \(1 - F_g(0.9)\), is the fraction of individuals \(i\)
for which \(S_{ig}>0.9\).  Note that the same ranking and score is shown in
the bottom half of Fig.~2.  The tan colored bars
indicate the fraction of individuals with nearly unbiased expression
(Eq.~\ref{eq:unbiased-test}).  Gene names (\(y\) axis) are colored according
to prior imprinting status: known imprinted (blue), nearby candidate (green),
and distant candidate (red).  ``X'' characters next to gene names indicate
mapping bias and/or cis-eQTL effects based on the reference/non-reference
allele test (Methods) while ``0'' indicates that total allele count was
insufficient for this test.

\subsection*{Figure 4}
%\includegraphics[scale=0.6]{figures/2016-11-01-plotting-distribution-of-s/S-Dx-strip-1.pdf}

Schizophrenia does not affect allelic bias of imprinted genes.
Distribution of read count ratio for Control, schizophrenic (SCZ), and
affectic spectrum (AFF) individuals within each gene that has been considered as imprinted in the DLPFC
brain area in this study.

\subsection*{Figure 5}
%\includegraphics[scale=0.6]{figures/2016-11-01-plotting-distribution-of-s/S-Dx-age-1.pdf}

Allelic bias depends differentially on age across imprinted genes.
The panels and colors are consistent with the imprinted genes and psychiatric diagnoses
presented in Fig.~4.  The differential dependence on age is apparent
when comparing PEG3 or ZNF331 (negative dependence) to KCNK9 or RP13-487P22.1
(positive dependence) or to NDN or NLRP2 (no dependence).

\section*{Tables}

\subsection*{Table 1}
%\label{tab:mod-sel}

\begin{tabular}{|cl|rl|l|}
\hline
\multicolumn{2}{|c|}{Hypothesis}                      & \multicolumn{2}{c|}{Results}             & Interpretation                               \\
response & predictor term                             & \(\Delta\)AIC   & p-value                &                                              \\
\hline
\(Q\)    &\((1\,|\,\mathrm{Gene})\)                   & \(-126.8\)      & \(8.5\times 10^{-28}\) & imprinted genes vary in allelic bias     \\
\(Q\)    &\((1\,|\,\mathrm{Dx})\)                     & \(2.0\)         & \(1.0\)                & similar allelic bias for Control, SCZ, AFF \\
\(Q\)    &\((1\,|\,\mathrm{Dx}:\mathrm{Gene})\)       & \(0.4\)         & \(0.21\)               & similar gene specific allelic bias for Control, SCZ, AFF    \\
\(Q\)    &\(\mathrm{Age}\)                            & \(1.3\)         & \(0.39\)               & no uniform effect of Age on allelic bias               \\
\(Q\)    &\((\mathrm{Age}\,|\,\mathrm{Gene})\)        & \(-18.9\)       & \(2.5\times 10^{-5}\)  & gene specific effect of Age on allelic bias                \\
\(Q\)    &\(\mathrm{Ancestry.1}\)                     & \(0.6\)         & \(0.24\)               & no uniform genetic effect on allelic bias               \\
\(Q\)    &\((\mathrm{Ancestry.1}\,|\,\mathrm{Gene})\) & \(-71.2\)       & \(4.6\times 10^{-16}\) & gene specific genetic effect on allelic bias               \\
\(Q\)    &\(\mathrm{Ancestry.3}\)                     & \(1.6\)         & \(0.54\)               & no uniform genetic effect on allelic bias               \\
\(Q\)    &\((\mathrm{Ancestry.3}\,|\,\mathrm{Gene})\) & \(-17.9\)       & \(3.8\times 10^{-5}\)  & gene specific genetic effect on allelic bias               \\
\(Q\)    &\((1\,|\,\mathrm{Gender})\)                 & \(2.0\)         & \(1.0\)                & no uniform Male-Female difference in allelic bias       \\
\(Q\)    &\((1\,|\,\mathrm{Gender}:\mathrm{Gene})\)   & \(-5.7\)        & \(5.5\times 10^{-3}\)  & gene specific Male-Female difference in allelic bias       \\
\hline
\end{tabular}

Hypothesis tests concerning the effect of various predictor terms on
the quasi log-translformed read count ratio \(Q\) of imprinted genes,
interpreted as effects on allelic bias.  Predictor terms whose effect on \(Q\)
is modeled as random are enclosed in parentheses.  Terms like
\((1\,|\,\mathrm{Dx})\) or \(\mathrm{Age}\) denote effects that are
\emph{uniform} across imprinted genes, while terms like
\((1\,|\,\mathrm{Dx}:\mathrm{Gene})\) or \((\mathrm{Age}\,|\,\mathrm{Gene})\)
mean that the effect is \emph{specific} to certain imprinted gene(s). Strongly
negative \(\Delta\mathrm{AIC}\) and small p-value indicate significant
dependence.  Note that the results are based on mixed models that contained
several terms besides the one tested for and shown here in the second column.
For details see Methods: ``Statistical models---informal overview'',
and beyond.

\end{document}

\documentclass[letterpaper]{article}
\usepackage{amsmath, geometry}

\title{Aging and genomic imprinting in the human brain}
\author{Attila}

\begin{document}
\maketitle

\section{Introduction}

Hello, World

\section{Methods}

\subsection{Data}

\(m\) individuals/samples, set \(\mathcal{G}\) of \(n\) genes

\subsection{Read count ratio}

\(\mathbf{S} = [S_{ig}]_{ig};\; i=1,...,m; \; g\in\mathcal{G}\)

\begin{equation}
S_{ig} = \frac{H_{ig}}{T_{ig}}= \frac{\sum_s H_s}{\sum_sT_s}
\end{equation}

\subsection{Regression models}

Regression analysis involved a subset \(\mathcal{G}_1\subset\mathcal{G}\) of
\(n_1\) genes.

The basic model, unlm.S, is
\begin{eqnarray}
\mathbf{S} &=& \mathbf{X} \boldsymbol{\beta} + \boldsymbol{\varepsilon},
\label{eq:unlm.S-matrix-form} \\
\varepsilon_{ig} &\overset{\mathrm{iid}}{\sim}& \mathrm{Norm}(0, \sigma^2_g)
\end{eqnarray}
where the response \(\mathbf{S}\) is an \(m\times n_1\) matrix of read count ratios,
\(\mathrm{X}\) is an \(m\times p\) design matrix, \(\boldsymbol{\beta}\) is a \(p\times n_1\) matrix of regression
coefficients, the random error \(\boldsymbol{\varepsilon}\) has the same
dimension as \(\mathbf{S}\), and gene \(g\in \mathcal{G}_1\).  Eq.~\ref{eq:unlm.S-matrix-form} may be given as
\begin{equation}
S_g = \mathbf{X} \beta_g + \varepsilon_g,
\label{eq:unlm.S-vector-form}
\end{equation}
where the vectors \(S_g, \beta_g, \varepsilon_g\)
are single columns taken from their respective matrix counterparts.

unlm.S was extended in several ways, yielding
\begin{enumerate}
\item four normal linear models unlm.S, unlm.R, wnlm.S, wnlm.S
\item two logistic models logi.S and logi2.S.
\end{enumerate}

The general form of the four normal linar models
(cf.~\ref{eq:unlm.S-vector-form}) is
\begin{equation}
W_g^{1/2} \tau(S_g) = W_g^{1/2} X \beta_g + \varepsilon_g.
\end{equation}
The extension here consists of \(W_g\), an \(m\times m\) diagonal matrix of
weights \(w_{ig}\) on the \(i\)-th diagonal position, and \(\tau\), a
transformation on read counts.  These quantities specify the four normal
linear models as laid out in Table~\ref{tab:nlm}.

\begin{table}
\begin{center}
\begin{tabular}{c|cc}
model & \(\tau\) & \(w_{ig}\) \\
\hline
unlm.S & identity & 1 \\
unlm.R & rank transf. & 1 \\
wnlm.S & identity & \(T_{ig}\) \\
wnlm.R & rank transf. & \(T_{ig}\) \\
\end{tabular}
\end{center}
\caption{Specificiation of four normal linear models based on read count
transformation \(\tau\) and weights \(w_{ig}\).}
\label{tab:nlm}
\end{table}

The logistic models, logi.S or logi2.S, share the general form
\begin{eqnarray}
S_g &=& \mu_g + c\, \varepsilon_g \\
\mu_g &=& h(X \beta_g) \\
\varepsilon_{ig} + \mu_g &\overset{\mathrm{iid}}{\sim}& \mathrm{Binom}(\mu_g, T_{ig}).
\label{eq:binom-error}
\end{eqnarray}
The link function \(h\) is \(h(u) = e^u / (1 + e^u)\) for logi.S and \(h(u) =
e^u / (2 + 2e^u)] + 1/2\) for logi2.S, and the scaling constant \(c=\) 1
 and \(1/2\), respectively.  Thus, the response \(S_g\) under logi2.S is scaled and shifted relative to
that under logi.S such that (with probabilty one) \(1/2\le S_{ig}\le 1\) under the former and
\(1/2\le S_{ig}\le 1\) under the latter.

Each of the six models has \(p\times n_1\) regression parameters corresponding to the
dimension of \(\boldsymbol{\beta}\).  This allows different behavior for
different genes since \(\beta_1\neq ...\neq\beta_{n_1}\) in general.
Therefore, the estimated regression coefficients are reported as \(\hat{\beta}_g =
(\hat{\beta}_{1g},...,\hat{\beta}_{jg},...,\hat{\beta}_{pg})\) for each gene \(g\), often
replacing index \(j\) with the name of the parameter such as \emph{Age} or
\emph{InstitutionPitt}.

A second set of six models was also
fitted, for which \(\boldsymbol{\beta}\) was constrained such that \(\beta_1 =
... = \beta_{n_1}\).  This was achieved by aggregating over genes
\(g\in\mathcal{G}_1\) the higher read count \(H'_i = \sum_g H_{ig}\), the
total read count \(T'_i = \sum_g T_{ig}\), redefining the read count ratio
as \(S'_i = H'_i / T'_i\), and replacing \(T_{ig}\) by \(T'_i\) in
Table~\ref{tab:nlm} and Eq.~\ref{eq:binom-error}.  Because \(S'_i\) is a
weighted average of \(\{S_{ig}\}_i\), results under these models are reported
as \(\hat{\beta}_\mathrm{WA} =
(\hat{\beta}_{1\mathrm{WA}},...,\hat{\beta}_{j\mathrm{WA}},...,\hat{\beta}_{p\mathrm{WA}})\).  A
third set of models is a slight variation of this second set in that
aggregation was done on a smaller subset of 8 genes selected at the initial
stage of the study.  Under these models the results are reported using the
WA.8 subscript instead of WA.

\section{Results}

\section{Discussion}

\end{document}

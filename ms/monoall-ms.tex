\documentclass[letterpaper]{article}
\usepackage{amsmath, geometry, graphicx, tikz}
\usetikzlibrary{arrows.meta}
\bibliographystyle{plain}

\title{Regulators of Genomic Imprinting in the Human Brain and Links to Schizophrenia}
\author{Attila Guly\'{a}s-Kov\'{a}cs\(^\ast\), Ifat Keydar\(^\ast\),\\
Eva Xia, Menachem Fromer, Doug Ruderfer,\\
Ravi Sachinanandam, Andrew Chess}
\date{Mount Sinai School of Medicine}

\begin{document}
\maketitle

\newpage

\maketitle

\begin{abstract}
Lorem ipsum...
\end{abstract}

\section{Introduction}

Genomic imprinting is the most prevalent in humans and other placental
organisms~\cite{Renfree2012} and, in line with this, well-known physiological
functions of imprinted genes include embryonic and placental development, body
growth, suckling, and maternal behavior~\cite{Plasschaert2014,Peters2014}.

At the molecular genetic level imprints are DNA methylation marks that are
specific to chromosomes and loci in the sense that those loci, called
imprinting control regions (ICR), are methylated on either the maternal or the
paternal chromosome, i.e.~the chromosome originating from the mother or
father, respectively~\cite{Plasschaert2014}.  Each ICR controls typically a
cluster of imprinted genes within a few hundred kilobases such that expression
of either the maternal or the paternal allele is repressed.  We refer to this
form of expression as \emph{parental bias} to emphasize that the
relative expression level of one allele of some gene, at least when averaged
over multiple cells in a tissue, may take some intermediate value between 0
and 1.  Thus imprinted genes are near 0 or 1 and so are biased or even
monoallelic expressed, whereas most non-imprinted genes are near \(1/2\), that
is, essentially unbiased.  Because phenotype is known to be rather sensitive
to the dose of imprinted genes~\cite{McNamara2013}, parental bias appears
as a form of fine regulator of overall expression level between full intensity
(unbiased case) and half intensity (monoallelic case).

Why natural selection favors parental bias for imprinted genes remains
debated~\cite{Wilkins2003,McDonald2005,Keverne2015} but the most mature of all
theories, kinship or conflict theory~\cite{Wilkins2003}, provides a flexible
framework for interpreting past studies on imprinted genes and formulating
hypotheses and predictions regarding their detailed regulation and
physiological function.  The theory assumes that all imprinted genes contribute to
inter-individual interactions in a highly dose-dependent fashion, and explains
parental bias with the conflicting interests of paternal and maternal genes,
which arise from sexual asymmetries in those interactions~\cite{Wilkins2003}.
A well-known asymmetry is the disproportionate importance of mothers in
nurturing offspring in Placentalia.  Kinship theory thus explains why
overexpression disorders of paternally or maternally biased genes in children
abnormally promote or inhibit, respectively, their
growth~\cite{Plasschaert2014,Peters2014}.

Since different inter-individual interactions take place in various
developmental stages and are mediated by various organs, kinship theory
explains the non-uniform pattern of ``imprintedness'' and parental bias over
various ages~\cite{Bourke2007} and tissue types that is seen for several
imprinted genes~\cite{Plasschaert2014,Peters2014}.  For other genes such
patterns await discovery.  The theory predicts relaxation of parental conflict
with age~\cite{Ubeda2012} and so raises the hypothesis that change in parental
bias is linked to aging.  A study on young adult mice partially supports that
hypothesis~\cite{Perez2015} but experimental evidence from humans, including
older individuals, is missing.

In the framework of kinship theory the question of aging is closely knit with
imprinted genes' role in social interactions and in the underlying psychiatric
functions~\cite{Ubeda2012,Wilkins2003}, whose importance has been increasing
in human evolution.  Indeed, most human imprinted gene syndromes are
characterized by not only growth disorder but also mental retardation and
psychiatric dysfunction~\cite{Plasschaert2014,Peters2014}.  More precisely,
paternally and maternally biased genes are suggested to play antagonistic
roles not only in growth but also in psychiatric functions~\cite{Crespi2008a},
since overexpression of the former is associated with autistic while that of the
latter with psychotic spectrum disorders.  For example, maternally derived
microduplications at 15q11-q13 may not only cause the Prader-Willi
syndrome~\cite{Peters2014}---whose symptoms include obesity and
psychosis---but also enhance the risk for
schizophrenia~\cite{Ingason2011,Sullivan2012}, which is perhaps the most
devastating psychotic spectrum disorder.

The link to schizophrenia is remarkable because 15q11-q13 is one of the very
few regions whose copy number variation is highly penetrant, while an
estimated thousands of SNPs exist, each of which confers only small risk
separately~\cite{Sullivan2012}.  Recently the CommonMind Consortium produced
and shared\footnote{http://commonmind.org/} genome-wide data sets on gene
expression in the dorsolateral prefrontal cortex (DLPFC) of hundreds of
schizophrenic and control individuals, and identified some 650 differentially
expressed genes~\cite{Fromer2016a}.  Our present work extends that analysis
with a special focus on imprinted genes and characterizes the variation of
parental bias across genes and individuals.  We find that \(\approx
1\%\) of all genes are imprinted in humans, which agrees closely with the most
recent mouse studies~\cite{Perez2015,DeVeale2012} but contradicts an earlier
report from mouse~\cite{Gregg2010a}.  We identify genes, such as UBE3A in 15q11-q13,
whose bias varies with age and/or psychiatric condition.  We also observe
variation with respect to gender and ancestry.

\section{Methods}

\subsection{Study design}

We used allele-specific RNA-seq data from the CommonMind project to probe the parental bias of allelic
expression for each gene \(g\) that is substantially expressed in the DLPFC.
We exploited the fact that single nucleotide polymorphisms (SNPs) provide a
way to distinguish between the two parental copies.  For a heterozygous SNP
\(s\) in a gene the reference and the alternative variant tags the maternal
and paternal gene copy or the other way around.  When parental bias towards
positively biased allele \(b\) (which may be maternal or paternal) is strong
then it seems likely that the count of reads tagged by the corresponding SNP
variant (which may be reference or alternative) is also relatively high.

This intuition may be formalized by introducing \(p\), the fraction of
transcripts from the parent towards which expression is biased.  \(p\) is
interpretable as parental bias.  Note that \(1/2\le p\le 1\).  Let \(b\)
identify the positively biased allele, which has the higher number of
transcripts of the two alleles.  Now let \(B\) denote the count of RNA-seq
reads that map to the \(b\) allele while \(T\) denote the total read count.
Then, assuming that \(B\) is a binomial random variable with given expected
relative frequency \(p\) and denominator \(T\), the probability that \(B\) is
the higher of the two read counts, i.e.~\(B \ge T - B\), is \[\sum_{x:T/2\le
x\le T} {x \choose T} p^x (1 - p)^{T-x}.\] The fact that this probability
increases with \(p\)---that is, parental bias---supports our intuition.

These considerations motivated us to quantify parental bias using a statistic called
\emph{read count ratio} \(S\), whose definition we
based on the total read count \(T\) and the \emph{higher read count} \(H\)
(i.e.~the count of reads carrying only the reference or the alternative SNP variant,
whichever is higher).  The
definition is
\begin{equation}
S_{ig} = \frac{H_{ig}}{T_{ig}}= \frac{\sum_s H_s}{\sum_sT_s},
\label{eq:S-definition}
\end{equation}
where \(i\) identifies an individual, \(g\) a gene, and the summation runs
over all SNPs \(s\) for which gene \(g\) is heterozygous in individual \(i\) (Fig.~\ref{fig:study-design}).
Note that if \(B_{ig}\) is the count or reads that map to the \(b\) allele (as
above) and if we make the same distributional assumption as above, namely that \(B_{ig}\sim
\mathrm{Binom}(p_{ig}, T_{ig})\), then \(\mathrm{Pr}(H_{ig}=B_{ig}|p_{ig})\), the probability of correctly
assigning the reads with the higher count to the allele towards which
expression is biased, tends to 1 as \(p_{ig} \rightarrow 1\).  We took
advantage of this theoretical result in that we subjected only those genes to
statistical inference, whose read count ratio was found to be high and,
therefore, whose \(p_{ig}\) is expected to be high as well.

\begin{figure} \begin{center}
\includegraphics[width=1.0\textwidth]{figures/by-me/commonmind-rna-seq/commonmind-rna-seq.pdf}
\end{center} \caption{ Quantifying parental expression bias in human
individuals using the read count ratio statistic \(S\).  The bias towards the
expression of the maternal (red) or paternal (blue) copy of some gene in
individual is measured based on the count of RNA-seq reads carrying the
reference allele (small closed circles) and the count of reads carrying an
alternative allele (open squares) at all SNPs in that gene for which the
individual is heterozygous.  The differences in the unobserved parental
expression bias between individual \(i_1\) and \(i_2\) arise only from
biological differences such as their disease status (black and gray
silhouettes), age, or gender.  In addition to these, the differences in the
observed read count ratio also reflect variation from technical sources like
tissue preparation, or RNA sequencing.  } \label{fig:study-design}
\end{figure}

Fig.~\ref{fig:study-design} illustrates the calculation of \(S_{ig}\) for the
combination of two hypothetical genes, \(g_1,g_2\), and two individuals,
\(i_1,i_2\).  It also shows an example for the less likely event that the lower rather
than the higher read count corresponds to the SNP variant tagging the higher
expressed allele (see SNP \(s_3\) in gene \(g_1\) in individual \(i_2\)).

Using the read count ratio we analyzed two aspects of the variation of
parental bias.  First, we ranked all expressed genes according to the
\emph{gene score}, a summary statistic that quantifies how right-shifted the
distribution of \(S_{ig}\) is for any given gene \(g\), and combined that
information with prior evidence for imprinting.  Second, given a gene \(g\)
that we called imprinted, we sought to identify biological regulators and
psychiatric consequences of parental bias by studying the conditional
distribution of \(S_{\cdot g}\) given observations \(x_{\cdot 1},...,x_{\cdot
p}\) on features of study individuals that are not gene-specific. We call
these features \emph{predictors} because we used them in a regression model
framework.  Predictors and their various levels are listed in
Table~\ref{tab:predictors}.

\begin{table}
\begin{center}
\begin{tabular}{r|l}
predictor & levels\\
\hline
Age &  \\
Institution & [MSSM], Penn, Pitt\\
Gender & [Female], Male\\
PMI & \\
Dx & [AFF], Control, SCZ\\
RIN &  \\
RNA\_batch & [A], B, C, D, E, F, G, H, 0\\
Ancestry.1 & \\
\vdots & \\
Ancestry.5 &  \\
\end{tabular}
\caption{ \emph{Left column:} predictors (explanatory variables) of read count ratio for the study of
regulation and consequences of parental bias.  \emph{Right column:} levels of
each factor valued (categorical) predictor.  Square brackets \([...]\) surround the baseline
level against which other levels are contrasted.  \emph{Abbreviations:} PMI: post-mortem interval; Dx:
disease status; AFF: affective spectrum disorder; SCZ: schizophrenia; RIN: RNA
integrity number; RIN2: the square of RIN; Ancestry.\(k\): the \(k\)-th
eigenvalue from the decomposition of genotypes indicating population structure}
\label{tab:predictors}
\end{center}
\end{table}

Before we carried out our read count ratio-based analyses, however, we cleaned
our RNA-seq data by quality-filtering and by improving the accuracy of SNP
calling with the use of DNA SNP array data and imputation. In the following
subsections of Methods we describe the data, these procedures, as well as our
regression models in detail.

\subsection{Data}

\subsubsection{Brain samples}

Human RNA samples were collected from the dorsolateral prefrontal cortex of
the CommonMind consortium from a total of \(579\) individuals after
quality control. Subjects included 267 control individuals, as well as 258
with schizophrenia (SCZ) and 54 with affective spectrum disorder (AFF).
RNA-seq library preparation uses Ribo-Zero (which selects against ribosomal
RNA) to prepare the RNA, followed by Illumina paired end library generation.
RNAseq was performed on Illumina HiSeq 2000.

\subsubsection{RNA-seq, mapping, SNP calling and filtering}

We mapped 100bp, paired-end reads (\(\approx50\) million reads per sample) using Tophat
to Ensembl gene transcripts of the human genome (hg19; February, 2009) using
default parameters with 6 mismatches allowed per pair (200bp total). We
required both reads in a pair to be successfully mapped and we removed reads
that mapped to \(>1\) genomic locus. Then, we removed PCR replicates using the
Samtools rmdup utility; around one third of the reads mapped (which is
expected, given the parameters we used and the known high repeat content of
the human genome). We used Cufflinks to determine gene expression of Ensembl
genes, using default parameters. Using the BCFtools utility of Samtools, we
called SNPs (SNVs only, no indels). Then, we invoked a quality filter
requiring a Phred score \(>20\) (corresponding to a probability for an
incorrect SNP call \(<0.01\)).

We annotated known SNPs using dbSNP (dbSNP 138, October 2013). Considering all
579 samples, we find 936,193 SNPs in total, 563,427 (60\%) of which are novel.
Further filtering of this SNP list removed the novel SNPs and removed SNPs
that either did not match the alleles reported in dbSNP or had more than 2
alleles in dbSNP. We also removed SNPs without at least 10 mapped reads in at
least one sample. Read depth was measured using the Samtools Pileup utility.
After these filters were applied, 364,509 SNPs remained in 22,254 genes. These
filters enabled use of data with low coverage.  For the 579
samples there were 203 million reads overlapping one of the
364,509 SNPs defined above.  Of those 158 million (78\%) had genotype data
available from either SNP array or imputation.

\subsubsection{Genotyping and calibration of imputed SNPs}

DNA samples were genotyped using the Illumina Infinium SNP array. We used
PLINK with default parameters to impute genotypes for SNPs not present on the
Infinium SNP array using 1000 genomes data.  We calibrated the
imputation parameters to find a reasonable balance between the number of genes
subjected to calling imprinting and the number false positive
calls since the latter can arise if a SNP is
incorrectly called heterozygous.

We first examined how many SNPs were heterozygous in DNA calls and had a
discordant RNA call (i.e.~homozygous SNP call from RNA-seq) using different imputation
parameters. Known imprinted genes were excluded. We examined RNA-seq reads
overlapping array-called heterozygous SNPs which we assigned a heterozygosity
score \(L_\mathrm{het}\) of 1, separately from RNA seq data
overlapping imputed heterozygous SNPs, where the \(L_\mathrm{het}\) score could
range from 0 to 1.  After testing different thresholds
we selected an \(L_\mathrm{het}\) cutoff of 0.95 (i.e. imputation confidence
level of 95\%), and a minimal coverage of 7 reads per SNP. With these
parameters, the discordance rate (monoallelic RNA genotype in the context of a
heterozygous DNA genotype) was 0.71\% for array-called SNPs and 3.2\% for
imputed SNPs.

The higher rate of discordance for the imputed SNPs
is due to imputation error.  These were taken into
account in two ways.
First, we considered all imputed SNPs for a gene \(g\) and individual \(i\)
jointly.  Second, we excluded
any individual, for which one or more SNPs supported biallelic
expression.

%At this point, the matrix includes 147
%million data points covering 213,208 SNPs, of which 114 million (77\%) have
%imputation data.

\subsubsection{Quality filtering}

\label{sec:filtering}

Two kind of data filters were applied sequentially: (1) a \emph{read
count-based} and (2) an \emph{individual-based}.  The read count-based filter
removes any such pair $(i,g)$ of individual $i$ and genes $g$ for which the
total read count $T_{ig}<t_\mathrm{rc}$, where the read count threshold
$t_\mathrm{rc}$ was set to 15. The individual-based filter removes any genes
$g$ (across all individuals) if read count data involving $g$ are available
for less than $t_\mathrm{ind}$ number of individuals, set to 25.
These final filtering procedures decreased the number of genes in the data from
\(15584\) to \(n=5307\).

\subsection{Statistical analysis}

\subsubsection{Test for nearly unbiased expression}

This test was defined by the criterion
\begin{equation}
S_{ig} \le 0.6 \text{ and } \mathrm{UCL}_{ig} \le 0.7,
\label{eq:unbiased-test}
\end{equation}
where the 95\% upper confidence limit \(\mathrm{UCL}_{ig}\) for the expected
read count ratio \(p_{ig}\) was calculated based on the assumption
that the higher read count \(H_{ig}=S_{ig}T_{ig}\sim \mathrm{Binom}(p_{ig},
T_{ig})\), on the fact that binomial random variables are
asymptotically (as \(T_{ig}\rightarrow \infty\)) normal with
\(\mathrm{var}(H_{ig}) = T_{ig}p_{ig}(1-p_{ig})\), and on the equalities
\(\mathrm{var}(S_{ig}) = \mathrm{var}(H_{ig}/T_{ig}) =
\mathrm{var}(H_{ig})/T_{ig}^2\).  Therefore
\begin{equation}
\mathrm{UCL}_{ig} = S_{ig} + z_{0.975} \sqrt{\frac{S_{ig} (1 - S_{ig})}{T_{ig}}},
\end{equation}
where $z_{p}$ is the $p$ quantile of the standard normal distribution.

\subsubsection{Regression models}
\label{sec:methods-regression}

Let \(m\) denote the number of individuals/samples and \(\mathcal{G}\) the set
of \(n=5307\) genes that passed quality filtering.  Regression analysis
involved a subset \(\mathcal{G}_1\subset\mathcal{G}\) of \(n_1=30\) genes
called as imprinted.

The basic model, unlm.S, is
\begin{eqnarray}
\mathbf{S} &=& \mathbf{X} \boldsymbol{\beta} + \boldsymbol{\varepsilon},
\label{eq:unlm.S-matrix-form} \\
\varepsilon_{ig} &\overset{\mathrm{iid}}{\sim}& \mathrm{Norm}(0, \sigma^2_g)
\end{eqnarray}
where the response \(\mathbf{S}\) is an \(m\times n_1\) matrix of read count ratios,
\(\mathrm{X}\) is an \(m\times p\) design matrix, \(\boldsymbol{\beta}\) is a \(p\times n_1\) matrix of regression
coefficients~(Table~\ref{tab:predictors}), the random error \(\boldsymbol{\varepsilon}\) has the same
dimension as \(\mathbf{S}\), and gene \(g\in \mathcal{G}_1\).  Eq.~\ref{eq:unlm.S-matrix-form} may be given as
\begin{equation}
S_g = \mathbf{X} \beta_g + \varepsilon_g,
\label{eq:unlm.S-vector-form}
\end{equation}
where the vectors \(S_g, \beta_g, \varepsilon_g\)
are single columns taken from their respective matrix counterparts.

unlm.S was extended in several ways, yielding
\begin{enumerate}
\item six normal linear models (Table~\ref{tab:nlm})
\item two logistic models logi.S and logi2.S.
\end{enumerate}

The general form of the normal linar models
(cf.~\ref{eq:unlm.S-vector-form}) is
\begin{equation}
\mathbf{W}_g^{1/2} \tau(S_g) = \mathbf{W}_g^{1/2} \mathbf{X} \beta_g + \varepsilon_g.
\label{eq:nlm-general}
\end{equation}
The extension here consists of \(\mathbf{W}_g\), an \(m\times m\) diagonal matrix of
weights \(w_{ig}\) on the \(i\)-th diagonal position, and \(\tau\), a
transformation on read counts.  Besides the trivial identity transformation
(i.e.~no transformation) two kinds of transformation were used: the rank
transformation and a quasi-log transformation \(\tau_Q\) defined as
\begin{equation}
\tau_Q(S_{ig};T_{ig}) \equiv Q_{ig} = - \log \left( 1 - S_{ig} \frac{T_{ig}}{T_{ig} + c}
\right),
\label{eq:Q}
\end{equation}
with pseudo read count \(c=1\) to avoid zero in the parenthesis since the \(\log\)
function is undefined at \(0\).

\begin{table}
\begin{center}
\begin{tabular}{r|cc}
 & \multicolumn{2}{c}{weights \(w_{ig}\)} \\
 & 1 & \(T_{ig}\) \\
\hline
no transformation & unlm.S & wnlm.S \\
quasi-log transf. & unlm.Q & wnlm.Q \\
rank transf. & unlm.R & wnlm.R \\
\end{tabular}
\end{center}
\caption{Specification and notation of normal linear models based on the weight
\(w_{ig}\) on each observation and the
transformation \(\tau\) applied to the set \(\{S_{ig} | g\}\) of read count
ratios for a given gene \(g\).}
\label{tab:nlm}
\end{table}

The logistic models, logi.S or logi2.S, share the general form
\begin{eqnarray}
S_g &=& \mu_g + c\, \varepsilon_g
\label{eq:logi-general}
\\
\mu_g &=& h(\mathbf{X} \beta_g)
\label{eq:glm-mean-predictor}
\\
\varepsilon_{ig} + \mu_g &\overset{\mathrm{iid}}{\sim}& \mathrm{Binom}(\mu_g, T_{ig}).
\label{eq:binom-error}
\end{eqnarray}
The link function \(h\) is \(h(u) = e^u / (1 + e^u)\) for logi.S and \(h(u) =
e^u / (2 + 2e^u)] + 1/2\) for logi2.S, and the scaling constant \(c=\) 1
 and \(1/2\), respectively.  Thus, the response \(S_g\) under logi2.S is scaled and shifted relative to
that under logi.S such that (with probabilty one) \(1/2\le S_{ig}\le 1\) under the former and
\(0\le S_{ig}\le 1\) under the latter.

Each of the eight models (including normal linear and logisitc models) has \(p\times n_1\) regression parameters corresponding to the
dimension of \(\boldsymbol{\beta}\).  This allows different behavior for
different genes since \(\beta_1\neq ...\neq\beta_{n_1}\) in general.
Therefore, the estimated regression coefficients are reported as \(\hat{\beta}_g =
(\hat{\beta}_{1g},...,\hat{\beta}_{jg},...,\hat{\beta}_{pg})\) for each gene \(g\), often
replacing index \(j\) with the name of the parameter such as \emph{Age} or
\emph{InstitutionPitt} (Table~\ref{tab:predictors}).

A second set of eight models was also
fitted, for which \(\boldsymbol{\beta}\) was constrained such that \(\beta_1 =
... = \beta_{n_1}\).  This was achieved by aggregating over genes
\(g\in\mathcal{G}_1\) the higher read count \(H'_i = \sum_g H_{ig}\), the
total read count \(T'_i = \sum_g T_{ig}\), redefining the read count ratio
as \(S'_i = H'_i / T'_i\), and replacing \(S_g\) by \(S'=(S'_1,...,S'_m)\) in
Eq.~\ref{eq:unlm.S-vector-form},~\ref{eq:nlm-general},~\ref{eq:logi-general}, and \(T_{ig}\) by \(T'_i\) in
Table~\ref{tab:nlm} and Eq.~\ref{eq:binom-error}.  Note that such aggregation
simplifies the matrix variables in Eq.~\ref{eq:unlm.S-matrix-form} to the
corresponding vector variables in Eq.~\ref{eq:unlm.S-vector-form}.  Because \(S'_i\) is a
weighted average of \(\{S_{ig}\}_i\), results under these models are reported
as \(\hat{\beta}_\mathrm{WA} =
(\hat{\beta}_{1\mathrm{WA}},...,\hat{\beta}_{j\mathrm{WA}},...,\hat{\beta}_{p\mathrm{WA}})\).

These \(2\times 8\) models are all multiple regression ones with \(p<1\)
parameters.  Two corresponding sets of models with a single Age parameter
(\(p=1\)) were also fitted but the results were only used for graphical
comparison of model fits in terms of predictions
Fig.~\ref{fig:predicted-curves} but not for quantitative inference.

\section{Results}

\subsection{Genome- and population-wide variation of parental bias}

\begin{figure}
\begin{center}
\includegraphics[scale=0.6]{figures/2016-07-19-genome-wide-S/complex-plot-1.png}
\end{center}
\caption{
Using the read count ratio statistic \(S\) to report on variation of parental
bias across individuals and genes.  \emph{Upper half}, from top to bottom: (1)
kernel density estimate; (2) the graph of the survival function 1 - ECDF,
where ECDF means empirical cumulative distribution function; note color scale
for heat map and green filled circles marking genes' score; (3) the heat map
representation of the survival function.  \emph{Lower half}, main panels: heat
map of the survival function for all 5307 analyzed genes ranked according to
their score; right side panels: gene score.
}
\label{fig:ranking-genes}
\end{figure}

\(5307\) genes passed our filters designed to remove genes with scarce
RNA-seq data reflecting low expression and/or technical problems pertaining to
RNA sequencing.  On these genes we performed exploratory statistical analysis based
on the read count ratio statistic \(S_{ig}\), whose results (below) we
interpreted in terms of the variation of parental bias both across genes \(g\)
and individuals \(i\).

Fig.~\ref{fig:ranking-genes} presents the conditional empirical distribution
of \(S_{\cdot g}\) given each gene \(g\).  Each of the three plots of the
upper half shows in a distinct representation the distribution for the same
three genes, while the main panels of the lower half present, in the most
compact representation, the distribution for all 5307 genes.  Two of the three
genes in the upper half, PEG10 and ZNF331, are \emph{known imprinted genes} in
the sense that they had previously been found imprinted in various tissue
types, developmental stages and species.  The third, AFAP1, is a
\emph{candidate gene} as it lacks such evidence.  For all three genes
\(S_{\cdot g}\) varies considerably within its theoretical range
\([\frac{1}{2}, 1]\).  This suggests variation of parental bias across study
individuals, although some component of the variation of \(S_{\cdot g}\) must
originate from technical sources.  Later subsections present our modeling and
detailed analysis of the across-individual variation on genes called
imprinted.  The rest of the present subsection provides account on the calling
of imprinted genes in the context of adult human DLPFC.

We called imprinted genes on two grounds: first, on a gene score that we
interpret as the gene's tendency to parental bias and, second, on whether the
gene had been described previously as imprinted in some context or is located near (within 1
megabase of) such known imprinted genes.  We defined gene score as the
location statistic \(1 - \mathrm{ECDF}_g(0.9)\), the fraction of individuals
\(i\) for which \(S_{ig}>0.9\).  This score is shown in side plots of the
lower half of Fig.~\ref{fig:ranking-genes} as gray filled circles or, for the
three genes mentioned above, as larger green circles (the latter are also
present in the second from top graph).  Based on the score genes were ranked;
the heat map of empirical distribution of \(S_{\cdot g}\) of ranked genes
suggests that the top \(\approx 50\) genes, which constitute \(\approx 1\%\)
of all genes in our analysis, are qualitatively different from the bottom
\(\approx 99\%\) suggesting that most of them are imprinted.  Consistent with
this, the top-scoring genes tended to cluster around genomic locations that
had been previously described as imprinted gene clusters
(Fig.~\ref{fig:clusters}).

\begin{figure}
\begin{center}
\includegraphics[scale=0.6]{figures/2016-08-01-ifats-filters/top-ranking-genes-1.pdf}
\caption{}
\label{fig:top-genes}
\end{center}
\end{figure}

The set of top scoring 50 genes was highly enriched in genes known to be
imprinted, marked by blue in Fig.~\ref{fig:top-genes}, or genes located near
such known genes (green).  In the top 50 set 27 such genes were found; we
called these and three more relatively high-scoring ``blue'' genes (NLRP2,
IGF2, and UBE3A in Fig.~\ref{fig:known-genes}) imprinted in the context of
human adult DLFPC.

Our classification implies that the remaining 23 genes (red in
Fig.~\ref{fig:top-genes}) in the top 50 set are not imprinted.  This was
supported by two additional lines of evidence.  First we exploited our
SNP-array data to identify genes for each individual that we had falsely
called heterozygous based on the RNA-seq measurements (TODO: details).
Second, we looked at an aspect of the empirical distribution of \(S_{\cdot
g}\) that is more complex than the gene score.  This meant testing the
hypothesis for each gene \(g\) and individual \(i\) that allelic expression is
(nearly) unbiased (Eq.~\ref{eq:unbiased-test}).  The fraction of individuals
for which the test was \emph{not} rejected tended to be much higher for the
``red'' genes in the top 50 (black bars in Fig.~\ref{fig:top-genes})
consistent with our classification.

More than a third of
all previously known imprinted genes (within our 5307-sized gene set) had a low score
based on \(S_{\cdot g}\) in the present context of human adult DLPFC.  This
suggests that imprinting of these genes is context-dependent or that they are
in fact imprinted in the present context but our method classified them
falsely.

\subsection{Selection among predictive models of parental bias}
\label{sec:results-regression}

The sources and psychological consequences of the variation of parental bias
in the 30 genes called imprinted was analyzed further through the detailed
characterization of how the read count ratio depends on the biological and
technical explanatory variables referred to as predictors
(Table~\ref{tab:predictors}).

Fig.~\ref{fig:S-age-gender} shows patterns of dependence (or independence) of
the read count ratio \(S_{\cdot g}\) for a given gene \(g\) on age and gender.  From
this visual inspection it seems that for several genes age is informative to
the distribution of \(S_{\cdot g}\) in terms of both the location (e.g.~the mean of
\(S_{\cdot g}\)) and scale (e.g.~variance); such apparent dependence on gender is not
clear.

\begin{figure}
\begin{center}
\includegraphics[scale=0.6]{figures/2016-06-26-trellis-display-of-data/S-age-gender-1.png}
\caption{}
\label{fig:S-age-gender}
\end{center}
\end{figure}

This qualitative result, however, is greatly complicated by the association
among predictors: taking only pairwise associations the situation is already
complex given the observed strong association between age and gender with each
other and with many other predictors (Fig.~\ref{fig:predictor-associations})
but higher order associations might also exist in the data.
The correct interpretation of plots like Fig.~\ref{fig:S-age-gender} also depends on
the amount of data, i.e.~the total read count \(T_{ig}\), based on which the
read count ratio \(S_{ig}\) was calculated.  Fig.~\ref{fig:weight-of-evidence}
shows how \(T_{ig}\) varies both within a gene and across genes.

\begin{figure}
\begin{center}
\includegraphics[scale=0.6]{figures/2016-08-23-glm-sampling-distributions/KCNK9-1.png}
\end{center}
\caption{Fitting four families of generalized linear models to the same data
set.  In each panel the horizontal axis is age and the vertical axis is the
possibly transformed read count ratio.  For all panels the green scatter
represents the same data set (KCNK9 gene, cf.~Fig\ref{fig:S-age-gender})
except that in the bottom right panel the observed read count ratio is
transformed according to Eq.~\ref{eq:Q}.  Given each one of four model
families the predicted read count ratio is represented by thick black curves.
The probability mass or density of the sampling distribution is indicated by a
cyan-to-magenta gradient and various prediction limits by gray lines.  Note
that for demonstration the depicted fitted models are all simple in the sense
that they contain age as a sole predictor.  Quantitative inference, however,
was based on the corresponding multiple regression models including all
predictors (Table~\ref{tab:predictors}).
  }
\label{fig:predicted-curves}
\end{figure}

These results motivated us to model the dependence of read count ratio for a
given gene on all predictors jointly in a multiple regression framework.  More
specifically, we used generalized linear models (GLMs).  With GLMs we sought
to find a reasonable compromise between simplicity and generality.  While that
simplicity facilitates parameter estimation and tests for independence,
generality allows fitting several GLM families, among which the best family or
families can then be selected based on the fit.  Thus the purpose of fitting
was twofold: (1) model selection, as well as (2) inference of dependencies
between the read count ratio and predictors given the selected model family or
families.

Out of the eight GLM families two were logistic (logi.S, logi2.S in
Fig.~\ref{fig:predicted-curves} top).  These have several desirable
theoretical properties for count-based data: they give zero probability for values
\(S_{\cdot g}>1\), they account for the observed dependence of the variance of
\(S_{\cdot g}\) on its mean, and are able to take into account the observed total
read counts.  These properties follow from the fact that logistic models are
natural extensions of simple binomial models conditioned on the observed total
read count \(T_{ig}\).  The remaining six fitted GLM families contain various
normal linear models (see Fig.~\ref{fig:predicted-curves} bottom for two of
these).  Although these are in general less suitable for count-type data they
are robust and easily interpretable. The families of normal linear models of
this study are characterized by two features: (i.)~whether or not they are
weighted by \(T_{ig}\) and (ii.)~the transformation, if any, that had been
applied to \(S_{ig}\) before the fit.  Transformation proved critical (see
below) but weighting had little impact either on model fit or parameter
estimates (not shown) so we removed unweighted models (unlm.Q, unlm.R, unlm.S)
reasoning that the mentioned small differences under weighted models reflect their
improved power due to their ability to utilize information in the observed
total read count.

We addressed model fit by using both Akaike Information Criterion (AIC) and
diagnostic graphs based on standardized deviance residuals.  However, only the
latter approach turned out conclusive because the estimate for AIC appeared
more biased for certain GLM families than for others, which rendered comparison
based on the information criterion unreliable for this problem.  In
particular, the homoscedasticity (constant error variance) assumed by normal
linear models
was strongly violated for wnlm.S because of the dependence of scale
of the read count ratio \(S\) on its location
(Fig.~\ref{fig:S-age-gender}).  The bottom left panel of Fig.~\ref{fig:predicted-curves}
shows this dependence on age for KCNK9, while the bottom right panel
illustrates how the dependence of scale but not that of location is abolished
by the quasi-log transformation \(Q\) (Eq.~\ref{eq:Q}).  The resulting
homoscedasticity afforded the wnlm.Q model good fit for all genes in terms of
three distinct diagnostics based on the residuals
(Fig.~\ref{fig:qqnorm-wnlm.Q},~\ref{fig:homoscedas-wnlm.Q},~\ref{fig:influence-wnlm.Q}).
Rank transformation \(R\) and the fitting of wnlm.R lead to a much lesser
improvement than \(Q\) and wnml.Q (not shown).

The same model checking diagnostics for the logistic models suggested good
fit under logi.S for 8 of the 30 genes
(Fig.~\ref{fig:qqnorm-logi.S},~\ref{fig:homoscedas-logi.S},~\ref{fig:influence-logi.S}).
These genes are highlighted with bold font in the top of Fig.~\ref{fig:pval-wnlm.Q}.
For subsequent analysis we selected both the logi.S model family for these 8
genes and the wnlm.Q family for all 30 genes.

\subsection{Biological predictors and interpretation}

We selected four biological terms in our linear predictor
(Eq.~\ref{eq:nlm-general}, \ref{eq:glm-mean-predictor}).  Of these terms Age
and Ancestry.1 represent the corresponding continuous predictors whereas
GenderMale and DxSCZ are “treatment” levels of the corresponding categorical
predictors (Gender and Dx, respectively) contrasted with the corresponding
control levels (GenderFemale and DxControl, see Table~\ref{tab:predictors}).
We left out DxAFF because of the small number of AFF individuals appeared to
substantially decrease the power of the hypothesis test described below.

For each term \(j\), and each gene \(g\) called imprinted, we tested the null
hypothesis \(\mathcal{H}^0_{jg} : \beta_{jg} = 0\) meaning that the read count
ratio is independent of that term (Fig~\ref{fig:pval-wnlm.Q}).  The
probabilistic interpretation of the rejection of \(\mathcal{H}^0_{jg}\) is
simply that parental bias depends on the predictor.  The
mechanistic interpretation is more complicated because such probabilistic
dependence may manifest from either of the following causal relations: (1) the
predictor regulates the parental bias of the gene; (2) parental bias regulates
the predictor; (3) they both regulate or are both regulated by a third,
latent, variable.

\begin{figure}
\begin{center}
\includegraphics[scale=0.6]{figures/2016-10-03-permutation-test/p-values-wnlm-Q-1.pdf}
\end{center}
\caption{}
\label{fig:pval-wnlm.Q}
\end{figure}

Given a model family (wnlm.Q or possibly logi.S) each hypothesis was tested both parametrically
and non-parametrically (see axes labeled with ``t-distribution'' and
``permutations'', respectively,
on Fig.~\ref{fig:pval-wnlm.Q},~\ref{fig:pval} and~\ref{fig:pval-tdist-vs-perm}). The p-value estimates showed good agreement across estimation methods
(parametric vs. non-parametric, Fig.~\ref{fig:pval}). This agreement could be expected from the regularity
of the likelihood function (Fig.~\ref{fig:ll-surf-explain}), since that allows
the parametric method---which is based
on asymptotic properties of maximum likelihood estimation---to work well already at the current
sample size. The agreement across models was weaker but still reasonable (Fig.~\ref{fig:pval}); the test
under logi.S appeared more powerful and/or more biased than that under wnlm.Q for most cases. Using a
heuristic decision rule (gray area in Fig.~\ref{fig:pval-wnlm.Q}) we rejected
the marginal hypothesis \(\mathcal{H}^0_{jg}\) if the conditional
hypotheses ``\(\mathcal{H}^0_{jg} |\text{parametric method}\)'' and
``\(\mathcal{H}^0_{jg} |\text{non-parametric method}\)'' were both rejected at size
\(\alpha = 0.05\) under the wnlm.Q model.

For each of the four terms (Age,...) the hypothesis of independence was rejected for some genes
(Fig.~\ref{fig:pval-wnlm.Q}). This appears to suggest that aging, gender and ancestry-related genetic variation regulates
parental bias of those genes whereas the risk of schizophrenia is regulated by it; but, as discussed
above, indirect causation involving a latent variable is also possible.

\begin{table}
\footnotesize
\begin{tabular}{lllll}
Gene & Gene type & Chr & Coefficient & Known phenotype\\
\hline
ZDBF2 & protein coding & 2 & Age,  Ancestry.1 & \\
NAP1L5 & protein coding & 4 & GenderMale & \\
PEG10 & protein coding & 7 & DxSCZ & \\
MEST & protein coding & 7 & DxSCZ & Silver-Russell syndrome\\
KCNK9 & protein coding & 8 & Age & Birk-Barel mental retardation dysmorphism syndrome\\
INPP5F & protein coding & 10 & Age & cell motility; endocytic recycling\\
KCNQ1OT1 & antisense & 11 & GenderMale & Beckwith-Wiedemann syn.; Isol.~hemihyperplasia\\
MEG3 & lincRNA & 14 & GenderMale & Mat/pat 14q32.2 hypermeth/microdel syndrome\\
RP11-909M7.3 & lincRNA & 14 & DxSCZ & \\
AL132709.5 & miRNA & 14 & Ancestry.1 & \\
MAGEL2 & protein coding & 15 & Age & Prader-Willi syn.; Schaaf-Yang syn.;
Arthrogryposis \\
NDN & protein coding & 15 & GenderMale & Prader-Willi syndrome\\
PWRN1 & lincRNA & 15 & Ancestry.1 & Prader-Willi syndrome\\
UBE3A & protein coding & 15 & DxSCZ & Prader-Willi syn.; Angelman syn.; circadian rhythm\\
PEG3 & protein coding & 19 & GenderMale & \\
\end{tabular}

\caption{}
\label{tab:signif-gene-effects}
\end{table}

Table \ref{tab:signif-gene-effects} presents some known properties of all genes for which read count ratio was found
to depend on at least one of the four biological terms. For some of these genes the dependence
is consistent with prior information on the gene's physiological or pathological role (cf.~Discussion).
For instance, association of PEG3 to gender is consistent with its role in morphological sexual
differentiation in the brain and in sexual and maternal behavior \cite{Broad2009}. Similarly, association of
UBE3A to schizophrenia is consistent with its suggested role as a risk factor for this and other
psychiatric conditions based on variation of its gene dose and copy number variation of its broader
locus 15q11q13 (PraderWilli/Angelman region) \cite{Sullivan2012, McNamara2013}.

\begin{figure}
\begin{center}
\includegraphics[scale=0.6]{figures/2016-08-08-imprinted-gene-clusters/segplot-wnlm-Q-99conf-1.pdf}
\end{center}
\caption{Biological effects: estimate and $99\%$ confidence interval for each
regression coefficient \(\beta_{jg}\) under the wnlm.Q model, where \(g\)
corresponds to a gene and \(j\) to a biological covariate (Age, Ancestry.1) or
a level of some biological factor (DxSCZ, GenderMale).}
\label{fig:biol-effects-wnlm.Q}
\end{figure}

Fig.~\ref{fig:biol-effects-wnlm.Q} and~\ref{fig:biol-effects-logi.S} extend
the above results in two ways. First, they present the maximum likelihood
estimates \(\hat\beta_{jg}\) as colored symbols, which are to be compared to
the dashed lines that represent the theoretical value \(\beta_{jg} = 0\) under
the null hypothesis.  For each regression term \(j\) the \(\hat\beta_{jg}\)
under wnlm.Q agreed reasonably with that under logi.S
(Fig.~\ref{fig:logi.S-wnlm.Q-compare}).  The estimates \(\hat\beta_{jg}\)
inform on the direction and size of the effect of each term \(j\) on read
count ratio of each significantly affected gene \(g\).  Along with these
quantities the 99\% confidence intervals are also presented (horizontal bars
in Fig.~\ref{fig:biol-effects-wnlm.Q} and~\ref{fig:biol-effects-logi.S}; see
also Fig.~\ref{fig:ll-surf-explain} for the illustration of the link between
confidence intervals and the geometry of the log-likelihood surface). For any
given term we found significantly affected genes with both negative
(e.g.~INPP5F) and positive (e.g.~KNCK9) direction.  For instance, the read
count ratio of INPP5F depended negatively on age, while that of KCNK9
positively, suggesting that aging may both down- and up-regulate parental
bias. Interestingly, we found that both genes for which the dependence on
DxSCZ is negative (UBE3A and RP11-909M7.3/MEG8) had been previously
established as maternally expressed, whereas both genes with positive
dependence (PEG10, MEST) as paternally expressed.

Of note is that the preceding analysis of overall expression based on the
CommonMind data~\cite{Fromer2016a} found only PEG10 and IGF2 to be
differentially expressed in schizophrenia out of the 30 genes we called
imprinted in the present study (Fig.~\ref{fig:diff-exp-scz}). PEG10, but not
IGF2, is among the 4 genes whose parental bias, rather than overall
expression, is significantly associated with schizophrenia.

The second extension in Fig.~\ref{fig:biol-effects-wnlm.Q}
and~\ref{fig:biol-effects-logi.S} is the arrangement of genes according to
their chromosomal location and containment in various imprinted gene clusters.
We found clusters (e.g. clus 14 on chr 7, clus 27 on chr 14, and clus 28 on
chr 15) within which some gene exhibited significant dependence on a given
predictor while all other genes did not. For a presumably causal predictor such
as age this result might mean that if the significant dependence indeed
reflects true association between that gene and the predictor, then all other
genes in the cluster are either independent or dependent only to a degree that
could not be detected as significant change. However, we did not observe two
significant dependencies of opposing direction in the same cluster, which was
expected given the central role of of a single imprinting control region in
the establishment and maintenance of imprinting and parental bias for all
genes in a cluster.

\subsection{Limitations}

The previous section focussed mainly on the question of dependence or independence between a gene's read
count ratio. As for the size of the effect, \(\hat\beta_{jg_1},
\hat\beta_{jg_2},...\)~were compared (Fig.~\ref{fig:biol-effects-wnlm.Q}) providing relative
effect sizes across genes \(g_1, g_2,...\)~for a given predictor \(j\). But how do predictors compare to each
other for any given gene? In particular, of the total variation in read count ratio what fraction is
explained by technical, and what fraction by biological predictors? Can we estimate the variation in
parental bias by removing the technical variation from all explained variation of read count ratio?
Unfortunately these questions proved unresolvable within the present experimental design and
fitted models. We performed ANOVA with the aim of assigning a component of variation to
each predictor but this failed because the reduction of residual sum of squares on the addition of
a predictor to the model strongly depended on the sequence in which predictors were added (see
Fig.~\ref{fig:anova} for a forward and reverse sequence). Such failure of ANOVA follows from non-orthogonality
of the terms (predictor variables and their levels) in the linear predictor
due to the dependencies among predictors (Fig.~\ref{fig:predictor-associations}).

We observed non-orthogonality not only among terms in the linear predictor but
also among the regression coefficients (Fig.~\ref{fig:ll-non-orthogonality}).
This means that the log-likelihood \(\ell\) of a given coefficient, such as
\(\beta_\mathrm{Age}\) depended on other coefficients.  The dependence of
\(\ell(\beta_\mathrm{Age})\) on \(\beta_\mathrm{RIN}\), the coefficient for
RNA integrity number, appeared particularly strong pointing to the sensitivity
of our analysis to RNA quality.

It is reasonable to assume for several predictors that their effects on the
read count ratio are not specific to any gene.  Such predictors are RIN
and the ones identifying RNA batches.  Accounting both for gene specific and
for aspecific effects at the same time cannot be achieved in the present GLM
framework (Fig.~\ref{fig:glm-vs-hierarch} left); therefore all fitted GLM
families mentioned so far assumed gene specific effect for all predictors.
Both under wnlm.Q and under logi.S RIN and the RNA\_batchB,...RNA\_batch0
predictors were found to differentially affect the read count ratio for
different genes
(Fig.~\ref{fig:all-effects-wnlm.Q},~\ref{fig:all-effects-logi.S}).  This result
is hard to explain mechanistically.  Rather, it seems to highlight the
mentioned limitation of the GLM framework.  One consequence of this limitation
is that, due to the discussed non-orthogonality of parameters, some technical
effects may manifest as biological ones and vice versa.

Another consequence of the lack of information sharing is relatively low
power.  More complex models of this characteristic would have even lower power
but complexity is often needed to account for interactions among predictors.
To assess interactions we studied the contextual dependence of read count
ratio on age
(Fig.~\ref{fig:interaction-wnlm.Q},~\ref{fig:interaction-logi.S}).  In the
context of different institutions or genders the estimates for
\(\beta_\mathrm{Age}\) were somewhat different from the marginal estimates
\(\hat\beta_\mathrm{Age}\) seen in Fig.~\ref{fig:biol-effects-wnlm.Q}
and~\ref{fig:biol-effects-logi.S}.  This suggests that interactions are
indeed present between age and some other predictors but the identity of these
predictors is uncertain due to the associations among them.

We also fitted model families (Fig.~\ref{fig:glm-vs-hierarch} middle) that
assume that the effects of all predictors are \emph{not} gene specific and so
pool even heterogeneous information together across genes.  But this approach
did not find significant dependence of read count ratio on any of the
biological predictors (Fig.~\ref{fig:all-effects-wnlm.Q}), which is consistent
with the our earlier result (Fig.~\ref{fig:biol-effects-wnlm.Q}) that each of
biological predictor affected relatively few genes significantly and these
effects were often of opposite direction.

\section{Discussion}

We have presented above the first genome-wide characterization of parental
bias in humans and, at the same time, the first genomics study that is
focussed specifically on imprinted genes' association with some psychiatric
disorder.  Our main finding, the association of several imprinted genes with aging
and/or schizophrenia, supports the prediction from kinship theory that the
roles of imprinted genes may dynamically change even at later stages of life
in order to mediate certain social interactions that require normal
psychiatric condition~\cite{Ubeda2012,Wilkins2003}.  Conversely---and regardless
the kinship theory---, our results suggest that perturbations to several
imprinted genes are associated with, and maybe causal to, schizophrenia and
possibly other psychiatric disorders.

If we accept that the main effect of parental bias is on overall expression
level, the finding that bias of UBE3A is negatively associated with
schizophrenia implies that its overexpression increases risk for the disorder
as it has been suggested based on copy number variation of 15q11-q13, the
Prader-Willi region~\cite{McNamara2013}.  Moreover, we find that the also
maternally biased RP11-909M7.3 is also negatively associated to schizophrenia
while two paternally biased genes, MEST and PEG10, are both positively
associated.  This is exactly the pattern that is predicted by kinship theory,
more precisely by its special case, the imprinted brain
theory~\cite{Crespi2008a}.

But if parental bias indeed gages overall expression, why did the
preceding differential expression analysis~\cite{Fromer2016a} miss association
to schizophrenia of three of the genes out of the four mentioned above?  It is
possible, of course, that our work falsely discovered those three associations
due to the limitations specific to our statistical models (see below).
However, it must be emphasized that only the present ``read count ratio
approach'' makes use of the available allele-specific information in RNA-seq
reads, which might afford much improved sensitivity relative to the
traditional ``overall expression approach'' used by~\cite{Fromer2016a}.  This
sensitivity gain might be necessary for the study of imprinted genes if their
expression level is indeed so tightly linked to phenotype~\cite{McNamara2013}.

Variation in a few imprinted genes, as we observe here, may seem inadequate
for a highly polygenic condition like schizophrenia.  But several points
indicate the opposite.  First, the psychological role of imprinted genes might
be disproportionately more significant than that of non-imprinted
ones~\cite{Crespi2008a} given that the complexity of placental mammalian brain
evolved parallel with imprinting~\cite{Renfree2012} and that they were
suggested to form functional networks with each other~\cite{Varrault2006},
whose functional impact typical gene association tests, such as ours,
underestimate since they assume independence of genes.  Second, our present
work filtered out a large portion of ``noisy'' genes, including imprinted
ones, several of which might be associated to schizophrenia and/or aging, etc.
Third, only the DLPFC is studied here but other brain areas likely exhibit
different patterns of imprinting and association to traits.

The interpretation of the association to gender and ancestry is less
straight-forward than that to age and schizophrenia.

\begin{itemize}
\item molecular mechanisms of regulation; our finding on Ancestry.1 point to
genetic effects; future work using methylation QTLs, supported by our findings
on SCZ and~\cite{Hannon2016}
\item bidirectional interactions: (1) children's growth and behavior
(suckling) and (2) maternal behavior~\cite{Keverne2015}; consistent with our finding on gender
\item further technical challenges: modeling similar to voom~\cite{Law2014}, RNA quality, our finding on RIN
\item limitations of our study due to human subjects; future challenge: bring mouse studies in
correspondence with human studies: modeling, evolutionary aspect
(conservation/divergence wrt imprinting)
\end{itemize}

\bibliographystyle{plain}
\bibliography{monoall-ms}

\newpage

\section{Supplementary Material}

\newpage

% Supplementary figures

\setcounter{figure}{0}
\makeatletter 
\renewcommand{\thefigure}{S\@arabic\c@figure}
\makeatother

\begin{figure}
\begin{center}
\includegraphics[scale=0.6]{figures/2016-08-08-imprinted-gene-clusters/score-genomic-location-1.png}
\end{center}
\caption{}
\label{fig:clusters}
\end{figure}

\begin{figure}
\begin{center}
\includegraphics[scale=0.6]{figures/2016-08-01-ifats-filters/known-genes-1.pdf}
\caption{}
\label{fig:known-genes}
\end{center}
\end{figure}

\begin{figure}
\begin{center}
\includegraphics[scale=0.6]{figures/2016-06-26-trellis-display-of-data/evar-scatterplot-matrix-2.png}
\end{center}
\caption{}
\label{fig:predictor-associations}
\end{figure}

\begin{figure}
\begin{center}
\includegraphics[scale=0.6]{figures/2016-06-26-trellis-display-of-data/S-age-tot-read-count-1.png}
\end{center}
\caption{}
\label{fig:weight-of-evidence}
\end{figure}

\begin{figure}
\begin{center}
\includegraphics[scale=0.6]{figures/2016-09-23-model-checking/qqnorm-wnlm-Q-1.pdf}
\end{center}
\caption{}
\label{fig:qqnorm-wnlm.Q}
\end{figure}

\begin{figure}
\begin{center}
\includegraphics[scale=0.6]{figures/2016-09-23-model-checking/qqnorm-logi-S-1.pdf}
\end{center}
\caption{}
\label{fig:qqnorm-logi.S}
\end{figure}

\begin{figure}
\begin{center}
\includegraphics[scale=0.6]{figures/2016-09-23-model-checking/homoscedas-wnlm-Q-1.pdf}
\end{center}
\caption{}
\label{fig:homoscedas-wnlm.Q}
\end{figure}

\begin{figure}
\begin{center}
\includegraphics[scale=0.6]{figures/2016-09-23-model-checking/homoscedas-logi-S-1.pdf}
\end{center}
\caption{}
\label{fig:homoscedas-logi.S}
\end{figure}

\begin{figure}
\begin{center}
\includegraphics[scale=0.6]{figures/2016-09-23-model-checking/influence-wnlm-Q-1.pdf}
\end{center}
\caption{}
\label{fig:influence-wnlm.Q}
\end{figure}

\begin{figure}
\begin{center}
\includegraphics[scale=0.6]{figures/2016-09-23-model-checking/influence-logi-S-1.pdf}
\end{center}
\caption{}
\label{fig:influence-logi.S}
\end{figure}

\begin{figure}
\begin{center}
\includegraphics[scale=0.6]{figures/2016-10-03-permutation-test/p-val-tdist-vs-perm-filt-iso-1.pdf}
\end{center}
\caption{}
\label{fig:pval-tdist-vs-perm}
\end{figure}

\begin{figure}
\begin{center}
\includegraphics[scale=0.6]{figures/2016-10-03-permutation-test/p-values-1.pdf}
\end{center}
\caption{}
\label{fig:pval}
\end{figure}

\begin{figure}
\begin{center}
\includegraphics[scale=0.6]{figures/2016-08-08-imprinted-gene-clusters/segplot-logi-S-99conf-1.pdf}
\end{center}
\caption{Biological effects: estimates and $99\%$ confidence intervals under
the logi.S model}
\label{fig:biol-effects-logi.S}
\end{figure}

\begin{figure}
\begin{center}
\includegraphics[scale=0.6]{figures/2016-08-21-likelihood-surface/explain-rll-wireframe-1.png}
\includegraphics[scale=0.6]{figures/2016-08-21-likelihood-surface/explain-rll-levelplot-B-1.png}
\end{center}
\caption{}
\label{fig:ll-surf-explain}
\end{figure}

\begin{figure}
\begin{center}
\includegraphics[scale=0.6]{figures/2016-06-22-extending-anova/logi-S-wnlm-Q-compare-1.pdf}
\end{center}
\caption{}
\label{fig:logi.S-wnlm.Q-compare}
\end{figure}

\begin{figure}
\begin{center}
\includegraphics{figures/by-me/monoall-dependencies-2/obs-simple-general/obs-simple-general}
\hspace{\fill}
\includegraphics{figures/by-me/monoall-dependencies-2/obs-simple-general-gene-aspec/obs-simple-general-gene-aspec}
\hspace{\fill}
\includegraphics{figures/by-me/monoall-dependencies-2/obs-bayesian/obs-bayesian}
\end{center}
\caption{}
\label{fig:glm-vs-hierarch}
\end{figure}

\begin{figure}
\begin{center}
\includegraphics[scale=0.6]{figures/2016-06-22-extending-anova/reg-coef-wnlm-Q-1.pdf}
\end{center}
\caption{}
\label{fig:all-effects-wnlm.Q}
\end{figure}

\begin{figure}
\begin{center}
\includegraphics[scale=0.6]{figures/2016-06-22-extending-anova/reg-coef-logi-S-filt-1.pdf}
\end{center}
\caption{}
\label{fig:all-effects-logi.S}
\end{figure}

\begin{figure}
\begin{center}
\includegraphics[scale=0.6]{figures/2016-08-21-likelihood-surface/ll-surf-coefs-wnlm-Q-1.png}
\end{center}
\caption{}
\label{fig:ll-non-orthogonality}
\end{figure}

\begin{figure}
\begin{center}
\includegraphics[scale=0.6]{figures/2016-07-08-conditional-inference/beta-age-cond-wnlm-Q-2-1.pdf}
\end{center}
\caption{}
\label{fig:interaction-wnlm.Q}
\end{figure}

\begin{figure}
\begin{center}
\includegraphics[scale=0.6]{figures/2016-07-08-conditional-inference/beta-age-cond-logi-S-2-skip-1.pdf}
\end{center}
\caption{}
\label{fig:interaction-logi.S}
\end{figure}

\begin{figure}
\begin{center}
\includegraphics[scale=0.6]{figures/2016-06-22-extending-anova/anova-effects-fw-rv-wnlm-Q-1.pdf}
\end{center}
\caption{}
\label{fig:anova}
\end{figure}

\begin{figure}
\begin{center}
\includegraphics[scale=0.6]{figures/2016-10-20-differential-expression-scz/venn-triple-1.pdf}
\end{center}
\caption{}
\label{fig:diff-exp-scz}
\end{figure}

%\begin{figure}
%\includegraphics[width=0.6\textwidth]{figures/2016-10-11-comparison-to-mouse-cerebellum/posterior-pp-vs-pval-wnlm-Q-1.pdf}
%\caption{Comparison of the effects of age and gender between the present work and a
%previous study~\cite{Perez2015} in the mouse cerebellum. }
%\label{fig:mouse-cerebellum}
%\end{figure}

% Supplementary tables

\setcounter{table}{0}
\makeatletter 
\renewcommand{\thetable}{S\@arabic\c@table}
\makeatother

\end{document}

\documentclass[letterpaper]{article}
%\documentclass[12pt,letterpaper]{article}
%\setlength{\textwidth}{480pt}
%\setlength{\textheight}{630pt}
%\setlength{\voffset}{0pt}

\usepackage{amsmath, geometry, graphicx, tikz, changes}
\usetikzlibrary{arrows.meta}
\bibliographystyle{plain}

\title{Variation of Genomic Imprinting in the Human Brain}
%\title{Regulators and Psychiatric Associates of Genomic Imprinting in the Human Brain}
\author{Attila Guly\'{a}s-Kov\'{a}cs\(^\ast\), Ifat Keydar\(^\ast\),
...,
%\\
%Eva Xia, Menachem Fromer, Doug Ruderfer,\\
%Ravi Sachinanandam,
Andrew Chess}
\date{Mount Sinai School of Medicine}

\linespread{1.2}

\begin{document}

\maketitle

\begin{abstract}
We present a
genome-wide analysis of human genomic imprinting based on
RNA-seq measurements of
parental bias in allele-specific expression in the
dorsolateral prefrontal cortex.  We find that the fraction of imprinted human
genes is consistent with lower (\(\approx 0.5\%\)) as opposed to higher
(\(\approx 5\%\)) estimates in mice.  Our analysis reveals that age up or
down-regulates allelic bias of some, but not all, imprinted genes
in adulthood
and
that allelic bias depends also on genetic variation.
These results support the
hypothesized role of imprinted genes in social interactions, which dynamically
change in the life course and depend on neuropsychological function.
\end{abstract}

\section{Introduction}

Genomic imprinting, leading to repression of either the maternal or paternal
allele, has reached its highest prevalence in humans and other placental
organisms~\cite{Renfree2012}.  In line with this, well-known physiological
functions of imprinted genes include embryonic and placental development, body
growth, suckling, and maternal behavior~\cite{Plasschaert2014,Peters2014}.
Genomic imprinting requires the placement of different epigenetic marks, such
as DNA methylation, at the respective alleles residing on the chromosomes
originating from the mother or father~\cite{Plasschaert2014}.  Indeed,
imprinted genes typically reside in clusters spanning hundreds of kilobases
and allele-specific differential epigenetic marks are found near specific
genes as well as in shared regulatory elements called imprinting control
regions. For non-imprinted genes expression is balanced such that the two
alleles are roughly equally expressed. By contrast, for imprinted genes
epigenetic marks lead to substantial degrees of
imbalance in the expression level of the two alleles.  We refer to the degree
of expression imbalance as \emph{allelic bias}; at its maximum expression is
completely monoallelic.

Why natural selection favors allelic bias for imprinted genes remains
debated~\cite{Wilkins2003,McDonald2005,Keverne2015} but the most mature of all
theories, kinship or conflict theory~\cite{Wilkins2003}, provides a flexible
framework for interpreting past studies on imprinted genes and formulating
hypotheses and predictions regarding their detailed regulation and
physiological function.  The theory assumes that all imprinted genes contribute to
inter-individual interactions in a highly dose-dependent fashion, and explains
allelic bias with the conflicting interests of paternal and maternal genes,
which arise from sexual asymmetries in those interactions~\cite{Wilkins2003}.
A well-known asymmetry is the disproportionate role of mothers in
nurturing offspring in Placentalia.  Kinship theory thus explains why
overexpression disorders of paternally or maternally biased genes in children
abnormally promote or inhibit, respectively, their
growth~\cite{Plasschaert2014,Peters2014}.

Since different inter-individual interactions take place in various
developmental stages and are mediated by various organs, kinship theory
explains the non-uniform pattern of ``imprintedness'' and allelic bias over
various ages~\cite{Bourke2007} and tissue types, which is seen for several
imprinted genes~\cite{Plasschaert2014,Peters2014}.  For other genes such
patterns await discovery.  The theory quantitatively predicts relaxation of
allelic conflict with age~\cite{Ubeda2012} and so raises the hypothesis that
change in allelic bias is linked to aging.  A study on newborn and young adult mice
partially supports that hypothesis~\cite{Perez2015} but experimental evidence
from humans, including older individuals, is missing.

In the framework of kinship theory the question of aging is closely tied to
the roles of imprinted genes in social interactions and in the underlying
psychiatric functions~\cite{Ubeda2012,Wilkins2003}, whose importance has been
increasing in human evolution.  Indeed, most human imprinted gene syndromes
are characterized by not only growth disorder but also mental retardation and
psychiatric dysfunction~\cite{Plasschaert2014,Peters2014}.  More precisely,
paternally and maternally biased genes are suggested to play antagonistic
roles not only in growth but also in psychiatric functions~\cite{Crespi2008a},
since overexpression of the former is associated with autistic while that of
the latter with psychotic spectrum disorders.  For example, maternally derived
microduplications at 15q11-q13 may not only cause the Prader-Willi
syndrome~\cite{Peters2014}---whose symptoms include obesity---but also highly
penetrant for schizophrenia~\cite{Ingason2011,Sullivan2012}, which is perhaps the most devastating
psychotic spectrum disorder.

Recently the CommonMind Consortium produced and
shared\footnote{www.commonmind.org} genome-wide data sets on genotype and gene
expression in the dorsolateral prefrontal cortex (DLPFC) of hundreds of
schizophrenic and control individuals, and identified some 650 differentially
expressed genes~\cite{Fromer2016a}. Our present work extends that analysis
with a focus on allele-specfic expression across the genome, allowing us to
determine allelic bias for each gene.  Based on that, we find that \(\approx
0.6\%\) of all genes are imprinted in the human DLPFC, the majority of which
had been reported to be imprinted in the context of one or another tissue
and/or species. We find a number of genes with allele specific expression
residing near clusters of known imprinted genes. Furthermore, our data suggest
that for several imprinted genes the variation of allelic bias across
individuals is explained by differences in {ancestry and} age in a manner that
depends on the gene.

\section{Methods}

\subsection{Defining the read count ratio to quantify allelic bias}

We quantifieied allelic bias using a statistic called
\emph{read count ratio} \(S\), whose definition we
based on the total read count \(T\) and the \emph{higher read count} \(H\),
i.e.~the count of reads carrying only either the reference or the alternative SNP variant,
whichever is higher.  The
definition is
\begin{equation}
S_{ig} = \frac{H_{ig}}{T_{ig}}= \frac{\sum_s H_s}{\sum_sT_s},
\label{eq:S-definition}
\end{equation}
where \(i\) identifies an individual, \(g\) a gene, and the summation runs
over all SNPs \(s\) for which gene \(g\) is heterozygous in individual \(i\) (Fig.~\ref{fig:study-design}).
Note that if \(B_{ig}\) is the count or reads that map to the \(b_{ig}\) allele
(defined as above) and if we make the same distributional assumption as above, namely that \(B_{ig}\sim
\mathrm{Binom}(p_{ig}, T_{ig})\), then \(\mathrm{Pr}(H_{ig}=B_{ig}|p_{ig})\), the probability of correctly
assigning the reads with the higher count to the allele towards which
expression is biased, tends to 1 as \(p_{ig} \rightarrow 1\).  We took
advantage of this theoretical result in that we subjected only those genes to
statistical inference, whose read count ratio was found to be high and,
therefore, whose \(p_{ig}\) is expected to be high as well.

Fig.~\ref{fig:study-design} illustrates the calculation of \(S_{ig}\) for the
combination of two hypothetical genes, \(g_1,g_2\), and two individuals,
\(i_1,i_2\).  It also shows an example for the less likely event that the lower rather
than the higher read count corresponds to the SNP variant tagging the higher
expressed allele (see SNP \(s_3\) in gene \(g_1\) in individual \(i_2\)).

\subsection{Data}

\subsubsection{Brain samples, RNA-seq}

Human RNA samples were collected from the dorsolateral prefrontal cortex of
the CommonMind consortium from a total of \(579\) individuals after
quality control. Subjects included 267 control individuals, as well as 258
with schizophrenia (SCZ) and 54 with affective spectrum disorder (AFF).
RNA-seq library preparation uses Ribo-Zero (which selects against ribosomal
RNA) to prepare the RNA, followed by Illumina paired end library generation.
RNA-seq was performed on Illumina HiSeq 2000.

\subsubsection{Mapping, SNP calling and filtering}

We mapped 100bp, paired-end RNA-seq reads (\(\approx50\) million reads per sample) using Tophat
to Ensembl gene transcripts of the human genome (hg19; February, 2009) with
default parameters and 6 mismatches allowed per pair (200 bp total). We
required both reads in a pair to be successfully mapped and we removed reads
that mapped to \(>1\) genomic locus. Then, we removed PCR replicates using the
Samtools rmdup utility; around one third of the reads mapped (which is
expected, given the parameters we used and the known high repeat content of
the human genome). We used Cufflinks to determine gene expression of Ensembl
genes, using default parameters. Using the BCFtools utility of Samtools, we
called SNPs (SNVs only, no indels). Then, we invoked a quality filter
requiring a Phred score \(>20\) (corresponding to a probability for an
incorrect SNP call \(<0.01\)).

We annotated known SNPs using dbSNP (dbSNP 138, October 2013). Considering all
579 samples, we find 936,193 SNPs in total, 563,427 (60\%) of which are novel.
Further filtering of this SNP list removed the novel SNPs and removed SNPs
that either did not match the alleles reported in dbSNP or had more than 2
alleles in dbSNP. We also removed SNPs without at least 10 mapped reads in at
least one sample. Read depth was measured using the Samtools Pileup utility.
After these filters were applied, 364,509 SNPs remained in 22,254 genes. These
filters enabled use of data with low coverage.  For the 579
samples there were 203 million reads overlapping one of the
364,509 SNPs defined above.  Of those 158 million (78\%) had genotype data
available from either SNP array or imputation.

\subsubsection{Genotyping and calibration of imputed SNPs}

DNA samples were genotyped using the Illumina Infinium SNP array. We used
PLINK with default parameters to impute genotypes for SNPs not present on the
Infinium SNP array using 1000 genomes data.  We calibrated the
imputation parameters to find a reasonable balance between the number of genes
assessable for allelic bias and the number false positive
calls since the latter can arise if a SNP is
incorrectly called heterozygous.

We first examined how many SNPs were heterozygous in DNA calls and had a
discordant RNA call (i.e.~homozygous SNP call from RNA-seq) using different imputation
parameters. Known imprinted genes were excluded. We examined RNA-seq reads
overlapping array-called heterozygous SNPs which we assigned a heterozygosity
score \(L_\mathrm{het}\) of 1, separately from RNA seq data
overlapping imputed heterozygous SNPs, where the \(L_\mathrm{het}\) score could
range from 0 to 1.  After testing different thresholds
we selected an \(L_\mathrm{het}\) cutoff of 0.95 (i.e. imputation confidence
level of 95\%), and a minimal coverage of 7 reads per SNP. With these
parameters, the discordance rate (monoallelic RNA genotype in the context of a
heterozygous DNA genotype) was 0.71\% for array-called SNPs and 3.2\% for
imputed SNPs.

The higher rate of discordance for the imputed SNPs
is due to imputation error.  These were taken into
account in two ways.
First, we considered all imputed SNPs for a gene \(g\) and individual \(i\)
jointly.  Second, we excluded
any individual, for which one or more SNPs supported biallelic
expression.

%At this point, the matrix includes 147
%million data points covering 213,208 SNPs, of which 114 million (77\%) have
%imputation data.

\subsubsection{Quality filtering}

\label{sec:filtering}

Two kind of data filters were applied sequentially: (1) a \emph{read
count-based} and (2) an \emph{individual-based}.  The read count-based filter
removes any such pair $(i,g)$ of individual $i$ and genes $g$ for which the
total read count $T_{ig}<t_\mathrm{rc}$, where the read count threshold
$t_\mathrm{rc}$ was set to 15. The individual-based filter removes any genes
$g$ (across all individuals) if read count data involving $g$ are available
for less than $t_\mathrm{ind}$ number of individuals, set to 25.
These final filtering procedures decreased the number of genes in the data from
\(15584\) to \(n=5307\).

\subsection{Test for nearly unbiased expression}

This test was defined by the criterion
\begin{equation}
S_{ig} \le 0.6 \text{ and } \mathrm{UCL}_{ig} \le 0.7,
\label{eq:unbiased-test}
\end{equation}
where the 95\% upper confidence limit \(\mathrm{UCL}_{ig}\) for the expected
read count ratio \(p_{ig}\) was calculated based on the assumption
that the higher read count \(H_{ig}=S_{ig}T_{ig}\sim \mathrm{Binom}(p_{ig},
T_{ig})\), on the fact that binomial random variables are
asymptotically (as \(T_{ig}\rightarrow \infty\)) normal with
\(\mathrm{var}(H_{ig}) = T_{ig}p_{ig}(1-p_{ig})\), and on the equalities
\(\mathrm{var}(S_{ig}) = \mathrm{var}(H_{ig}/T_{ig}) =
\mathrm{var}(H_{ig})/T_{ig}^2\).  Therefore
\begin{equation}
\mathrm{UCL}_{ig} = S_{ig} + z_{0.975} \sqrt{\frac{S_{ig} (1 - S_{ig})}{T_{ig}}},
\end{equation}
where $z_{p}$ is the $p$ quantile of the standard normal distribution.

\subsection{Regression analysis}
Before we carried out our read count ratio-based analyses, however, we cleaned
our RNA-seq data by quality-filtering and by improving the accuracy of SNP
calling with the use of DNA SNP array data and imputation. In the following
subsections of Methods we describe the data, these procedures, as well as our
regression models in detail.

\label{sec:methods-regression}

\subsubsection{Formulation of models}

Given a top-scoring gene \(g\), we
sought to identify biological regulators and psychiatric consequences of
allelic bias by studying the conditional distribution of \(S_{\cdot g}\) given
observations \(x_{\cdot 1},...,x_{\cdot p}\) on features of study individuals
that are not gene-specific. We call these features \emph{predictors} because
we used them in a regression model framework.  Predictors and their various
levels (if any) are listed in Table~\ref{tab:predictors}.

Let \(m\) denote the number of individuals/samples and \(\mathcal{G}\) the set
of \(n=5307\) genes that passed quality filtering.  Regression analysis
involved a subset \(\mathcal{G}_1\subset\mathcal{G}\) of \(n_1=30\) genes
called as imprinted.

\added{Outline:}
\begin{enumerate}
\item \added{modeling genes jointly/globally (mixed model) and
separately/locally}
\item \added{data transformations, link functions, error distributions}
\end{enumerate}


\subsubsection{Model fitting, selection and inference}

\added{Outline:}
\begin{enumerate}
\item \added{R package, convergence}
\item \added{Selection criteria and heuristic search}
\end{enumerate}

\section{Results}

\subsection{Genome- and population-wide variation of allelic bias}

A total of \(5307\) genes passed our filters designed to remove genes with
scarce RNA-seq data reflecting low expression and/or low coverage of RNA-seq.
Examining these genes, we performed exploratory statistical analysis based on
the read count ratio statistic \(S_{ig}\), whose results (below) we
interpreted in terms of the variation of allelic bias both across genes \(g\)
and across individuals \(i\).  Note that our later analyses
(Section~\ref{sec:results-regression} and below) used
information not only in \(S_{ig}\) but also in the total read count as well as
in data beyond RNA-seq.

Fig.~\ref{fig:ranking-genes} presents the conditional empirical distribution
of \(S_{\cdot g}\) given each gene \(g\).  Each of the three plots of the
upper half show in a distinct representation the same empirical distributions
based on data for three genes.  The main panels of the lower half present, in
the most compact representation, the distributions based on all data (5307
genes).  Two of the three genes in the upper half, PEG10 and ZNF331, are
\emph{known imprinted} genes in the sense that they had previously been found
imprinted in the context of some developmental stage, species, and tissue type
other than the adult human DLPFC.  The third, AFAP1, has not been reported to be imprinted
in any context.  For all three genes \(S_{\cdot g}\) varies considerably
within its theoretical range \([\frac{1}{2}, 1]\).  This suggests variation of
allelic bias across study individuals, although some component of the
variation of \(S_{\cdot g}\) must originate from technical sources.  Later
subsections present our modeling and detailed analysis of the
across-individual variation on genes called imprinted in the context of adult
human DLPFC.  The rest of the present subsection reports the calling of
imprinted genes.

We defined gene score as
the location statistic \(1 - \mathrm{ECDF}_g(0.9)\), the fraction of
individuals \(i\) for which \(S_{ig}>0.9\).  This score is shown in side plots
of the lower half of Fig.~\ref{fig:ranking-genes} as gray filled circles or,
for the three genes mentioned above, as larger green circles (the latter are
also present in the second from top graph).  Based on the score genes were
ranked; the heat map of empirical distribution of \(S_{\cdot g}\) of ranked
genes suggests that the top \(50\) genes, which constitute \(\approx
1\%\) of all genes in our analysis, are qualitatively different from the
bottom \(\approx 99\%\) suggesting that most of them are imprinted.
Consistent with this, the top-scoring genes tended to cluster around genomic
locations that had been previously described as imprinted gene clusters
(Fig.~\ref{fig:clusters}).

The set of top scoring 50 genes is highly enriched in known imprinted genes,
marked by blue in Fig.~\ref{fig:top-genes} and in \emph{nearby candidate}
genes (green) defined as being within 1Mb of a known imprinted gene. Within
the top 50, we find 29 such genes; 21 known imprinted genes and eight nearby
candidates.  In subsequent analysis (below) we also consider UBE3A as
demonstrating allelic bias consistent with imprinting in the context of human
adult DLFPC as evidenced by Fig.~\ref{fig:known-genes} even though its rank
falls outside of the top 50.

The remaining 21 genes in the top 50 are separated by \(>1\) Mb from some
known imprinted gene (termed \emph{distant candidates}, red in
Fig.~\ref{fig:top-genes}).  Upon further examination these distant candidate
genes are overwhelmingly likely not imprinted. The primary reason for this
conclusion is that we performed a test to see if there is reference allele
bias for all candidate genes. For any gene (known imprinted, or candidate) the
expectation is that when some allelic bias is detected, that should equally
favor the reference or non-reference allele since for a given individual who
is heterozygous at a given SNP in the genome it is reasonable to assume that
the chances are equal that the mother or that the father has the reference
allele. Most known imprinted genes and the nearby candidates display a
reference/non-reference distribution consistent with a binomial distribution
with a probability of 0.5 for both the reference and non-reference alleles.
However, and in sharp distinction, most distant candidates have distributions
of reference/non-reference that are not consistent with equal probabilities
(see genes marked with ``X'' in Fig.~\ref{fig:top-genes}).  Indeed, for most of
them the distribution is shifted towards the reference allele strongly
suggesting that mistaken genotyping, imputation or a mapping issue led to the
presence of these red genes in the list of the top 50 genes. One could argue
that we should have left these genes out of Fig.~\ref{fig:top-genes}, but we
thought it was important to show them and to indicate the reasons they are set
aside.   Note also, that we also tested the hypothesis for each gene \(g\) and
individual \(i\) that allelic expression is (nearly) unbiased
(Eq.~\ref{eq:unbiased-test}).  The fraction of individuals for which the test
was \emph{not} rejected tends to be much higher for the ``red'' genes in the
top 50 (black bars in Fig.~\ref{fig:top-genes}).

While the shifted distribution of reference/non-reference alleles leads us to
discount the possibility of imprinting, random monoallelic expression is still
a distinct possibility for these candidates as our studies of random
monoallelic expression in mice suggested that a substantial fraction
(\(40-80\%\)) of random monoallelically-expressed genes had a very strong bias
towards monoallelic expression of one of the two alleles \cite{Zwemer2012}.
Moreover, it is worth noting that three of these candidates are from the major
histocompatibility locus (HLA), which is notable for extensive polymorphism
and difficulties with allelic identification. For these three genes we also
analyzed them more thoroughly with HLA-specific methods for determining
haplotype based on RNA-seq \cite{Bai2014a} and genotype data \cite{Zheng2014}.
The high observed read count ratios for HLA genes appear to be driven by
eQTL-like effects, not by random monoallelic expression nor by imprinting
(manuscript in preparation).  Examining all the assessable known imprinted
genes, we find than \(\frac{1}{3}\)rd of them have a low gene score. This
suggests that these genes do not display imprinted expression in the human
adult DLPFC, consistent with many reports in the literature indicating that
known imprinted genes are often imprinted in some but not all tissues.  

\subsection{Regulators of allelic bias}
\label{sec:results-regression}

\added{Outline:}
\begin{enumerate}
\item \added{the best fitting data transformation, link function and error distribution was
selected for a fixed set of terms in the linear predictor (see Methods) for
both global and local models}
\item \added{global models (considering genes jointly) gave stronger results than
local ones because the former allows gene-wise data units to borrow
statistical strength from each other}
\item \added{terms in the linear predictor that significantly improved model fit are mainly technical ones;
including them in the model therefore corrects for a large technical noise}
\item \added{the most important biological term is 'Gene', meaning that
allelic bias varies greatly among imprinted genes regardless other variables}
\item \added{the remaining biological terms that significantly improving model fit are 
'Ancestry.1:Gene' and 'Age:Gene', suggesting that genes differ not only in
their mean/overall allelic bias but also in how bias depends on Ancestry.1 and
Age; in other words, Ancestry.1 and Age has a differential effect on allelic
bias when comparing genes}
\item \added{Ancestry.1 and Age alone was found to have little if any effect
meaning that genes do not share a common pattern of how allelic bias depends
on these variables}
\item \added{Dx (SCZ) and Gender had essentially no effect alone or in combination with
'Gene' (only 'Gender:Gene' lead to minor improvement in fit)}
\item \added{TODO: imprinted gene clusters and other biological effect ('Gene:Cluster', Age:Cluster, Ancestry.1:Cluster)}
\end{enumerate}


\subsection{\added{Inspecting allelic bias in individual imprinted genes}}

\added{Outline:}
\begin{enumerate}
\item \added{examples for imprinted genes with strong allelic bias
(monoallelically expressed) and those with moderate bias; discuss those genes}
\item \added{predicted effect of Age for each gene based on mixed model}
\item \added{note: similar prediction for Ancestry.1 is possible but seems much less
interesting because we do not know the genetic mechanisms linked to Ancestry.1}
\item \added{depending on the effect of 'Gene:Cluster', Age:Cluster,
Ancestry.1:Cluster inspect and discuss clusters}
\end{enumerate}

\section{Discussion}

We present the first genome-wide characterization of allelic bias of
expression in humans and, at the same time, the first genome-scale study of
the potential for imprinted genes' impact on a psychiatric disorder. Important
to mention first, but not surprising, is our finding that \(\approx
\frac{2}{3}\)rd of known imprinted genes display imprinting in the adult human
DLPFC. The fact that the remaining \(\approx \frac{1}{3}\)rd of known
imprinted genes do not display imprinting is consistent with many other
studies showing that imprinted genes often display imprinting in some but not
all tissues. We also find allelic bias consistent with imprinting for eight
novel candidate genes, all of which are nearby known imprinted genes.  Strong
allelic bias for \(\approx 0.6\%\) of all assessable genes suggests that they
are imprinted, which agrees closely with the most recent mouse
studies~\cite{DeVeale2012,Perez2015} but contradicts a controversial earlier
estimate of \(\approx 5\%\) also from mouse~\cite{Gregg2010a}.

We also examined the dependence of allelic bias on variables namely age,
diagnosis, gender and ancestry.
\added{Outline from this point:}
\begin{enumerate}
\item \added{we find (so far the strongest?) evidence that age regulates
imprinted genes in adulthood: biological significance in light of kinship
theory, social interactions and aging}
\item \added{we find a strong Ancestry.1:Gene effect; this shows how genetic
(Ancestry.1) and epigenetic (imprinting control region of a given gene)
mechanisms act together to shape gene expression}
\item \added{we do not detect effect of Dx (SCZ); lack of effect is
surprising given the involvement of several imprinted genes in psychiatric
disorders; alternatively the effect exists but has been masked by the large
technical noise in our study}
\item \added{even subtle variation in allelic bias and allele specific
expression might substantially affect biological function for some (imprinted)
genes while not for others; this must be borne in mind in interpreting our
results}
\end{enumerate}

\bibliographystyle{plain}
\bibliography{monoall-ms}

\newpage

\section{Figures, Tables and Supplementary Material}

\begin{table}[h]
\begin{center}
\begin{tabular}{r|l}
predictor & levels\\
\hline
Age &  \\
Institution & [MSSM], Penn, Pitt\\
Gender & [Female], Male\\
PMI & \\
Dx & [AFF], Control, SCZ\\
RIN &  \\
RNA\_batch & [A], B, C, D, E, F, G, H, 0\\
Ancestry.1 & \\
\vdots & \\
Ancestry.5 &  \\
\end{tabular}
\caption{ \emph{Left column:} predictors (explanatory variables) of read count
ratio.  \emph{Right column:} levels of each factor-valued (i.e.~categorical)
predictor.  Square brackets \([...]\) surround the baseline level against
which other levels are contrasted.  \emph{Abbreviations:} PMI: post-mortem
interval; Dx: disease status; AFF: affective spectrum disorder; SCZ:
schizophrenia; RIN: RNA integrity number;
Ancestry.\(k\): the \(k\)-th eigenvalue from the decomposition of genotypes
indicating population structure.}
\label{tab:predictors}
\end{center}
\end{table}

\begin{table}
\begin{tabular}{r|l}
\multicolumn{2}{c}{read count ratio (response variable)} \\
\hline
\(S_{ig}\) & untransformed ratio \\
\(Q_{ig}\) & \emph{q}uasi-log-transformed ratio \\
\(R_{ig}\) & \emph{r}ank-transformed ratio \\
\hline
\multicolumn{2}{c}{\replaced{link function and error distribution}{regression models}} \\
\hline
unlm.[S/Q/R] & \emph{u}nweighted \emph{n}ormal \emph{l}inear \emph{m}odel fitted to \(S_{ig}\)/\(Q_{ig}\)/\(R_{ig}\)\\
wnlm.[S/Q/R] & \emph{w}eighted \emph{n}ormal \emph{l}inear \emph{m}odel fitted to \(S_{ig}\)/\(Q_{ig}\)/\(R_{ig}\)\\
logi.S & standard \emph{logi}stic model fitted to \(S_{ig}\)\\
logi2.S & \emph{logi}stic model with \(\frac{1}{2}\times\) down-scaled link function, fitted to \(S_{ig}\) \\
\end{tabular}
\caption{Explanation of the nomenclature for read count ratio variables and
model names.}
\label{tab:model-names}
\end{table}

\begin{figure}[h]
\begin{center}
\includegraphics[width=1.0\textwidth]{figures/by-me/commonmind-rna-seq-ms/commonmind-rna-seq-ms.pdf}
\end{center}
\caption{ Quantifying allelic bias of expression in human
individuals using the RNA-seq read count ratio statistic \(S_{ig}\).  The strength of
bias towards the expression of the maternal (red) or paternal (blue) allele of
a given gene \(g\) in individual \(i\) is gauged with the count of RNA-seq
reads carrying the reference allele (small closed circles) and the count of
reads carrying an alternative allele (open squares) at all SNPs for which the
individual is heterozygous.  The allele with the higher read count tends to
match the allele with the higher expression but measurement errors may
occasionally revert this tendency as seen for SNP \(s_3\) in gene \(g_1\) in
individual \(i_2\).
%The differences in the unobserved allelic bias
%between individual \(i_1\) and \(i_2\) arise only from biological differences
%such as their disease status (black and gray silhouettes), age, or gender.  In
%addition to these, the differences in the observed read count ratio also
%reflect variation from technical sources like tissue preparation, or RNA
%sequencing. 
}
\label{fig:study-design}
\end{figure}

\begin{figure}[h]
\begin{center}
\includegraphics[scale=0.6]{figures/2016-07-19-genome-wide-S/complex-plot-1.png}
\end{center}
\caption{
Using the read count ratio statistic \(S\) to report on variation of allelic
bias across individuals and genes.  \emph{Upper half}, from top to bottom: (1)
kernel density estimate; (2) the graph of the survival function 1 - ECDF,
where ECDF means empirical cumulative distribution function; note color scale
for heat map and green filled circles marking genes' score; (3) the heat map
representation of the survival function.  \emph{Lower half}, main panels: heat
map of the survival function for all 5307 analyzed genes ranked according to
their score; right side panels: gene score.
}
\label{fig:ranking-genes}
\end{figure}

\begin{figure}[h]
\begin{center}
\includegraphics[scale=0.6]{figures/2016-08-01-ifats-filters/top-ranking-genes-1.pdf}
\caption{
The top 50 genes ranked by the gene score.  The score of gene \(g\) is \(1 -
\mathrm{ECDF}_g(0.9)\), the fraction of individuals \(i\) for which
\(S_{ig}>0.9\) and is indicated by the length of dark blue, dark green or dark
red bars.  Note that the same ranking and score is shown in the bottom half of
Fig.~\ref{fig:ranking-genes}.  The right border of the light blue, light green
and light red bars is at \(1 - \mathrm{ECDF}_g(0.8)\).  The length of the
black bars indicates the fraction of individuals passing the test of nearly
unbiased expression (Eq.~\ref{eq:unbiased-test}).  ``X'' characters next to
gene names indicate reference allele bias, while ``0'' indicate that
reference allele bias could not be determined due to small amount of data.
}
\label{fig:top-genes}
\end{center}
\end{figure}

\begin{figure}[h]
\begin{center}
\includegraphics[scale=0.6]{figures/2016-06-26-trellis-display-of-data/S-age-1.pdf}
\caption{
Variation of the read count ratio \(S_{ig}\) with age across hundreds of individuals
\(i\) (dots) and 30 genes \(g\) that have been considered as imprinted in the DLPFC
brain area in this study.
}
\label{fig:S-age}
\end{center}
\end{figure}

% Supplementary tables

\setcounter{table}{0}
\makeatletter 
\renewcommand{\thetable}{S\@arabic\c@table}
\makeatother

\setcounter{figure}{0}
\makeatletter 
\renewcommand{\thefigure}{S\@arabic\c@figure}
\makeatother

\begin{figure}[h]
\begin{center}
\includegraphics[scale=0.6]{figures/2016-08-08-imprinted-gene-clusters/score-genomic-location-1.png}
\end{center}
\caption{
Clustering of top-scoring genes in the context of human DLPFC around genomic locations that
had been previously described as imprinted gene clusters in other contexts.
}
\label{fig:clusters}
\end{figure}

\begin{figure}[h]
\begin{center}
\includegraphics[scale=0.6]{figures/2016-08-01-ifats-filters/known-genes-1.pdf}
\caption{Known imprinted genes ranked by the gene score (dark blue bars).
``Known imprinted'' refers to prior studies on imprinting in the context of
any tissue and developmental stage.  The length of the
black bars indicates the fraction of individuals passing the test of nearly
unbiased expression.}
\label{fig:known-genes}
\end{center}
\end{figure}

\begin{figure}[h]
\begin{center}
\includegraphics[scale=0.6]{figures/2016-06-26-trellis-display-of-data/evar-scatterplot-matrix-2.png}
\end{center}
\caption{
Distribution and inter-dependence of predictors.  The diagonal graphs of the
plot-matrix show the marginal distribution of six predictors (Age,
Institution,...)~while the off-diagonal graphs show pairwise joint
distributions.  For instance, the upper left graph shows that, in the whole
cohort, individuals' Age
ranges between ca.~15 and 105 years and most individuals around 75 years; the
bottom and right neighbor of this graph both show (albeit in different
representation) the joint distribution of Age and Institution, from which can
be seen that individuals from Pittsburg tended to be younger than those from
the two other institutions.
}
\label{fig:predictor-associations}
\end{figure}

\begin{figure}[h]
\begin{center}
\includegraphics[scale=0.6]{figures/2016-06-26-trellis-display-of-data/S-age-gender-1.png}
\caption{
Variation of the read count ratio \(S_{ig}\) with age and gender across hundreds of individuals
\(i\) (dots) and 30 genes \(g\) that have been considered as imprinted in the DLPFC
brain area in this study.
}
\label{fig:S-age-gender}
\end{center}
\end{figure}

\begin{figure}[h]
\begin{center}
\includegraphics[scale=0.6]{figures/2016-06-26-trellis-display-of-data/S-age-tot-read-count-1.png}
\end{center}
\caption{Variation of total RNA-seq read count across genes and individuals.}
\label{fig:weight-of-evidence}
\end{figure}

\begin{figure}[h]
\begin{center}
\includegraphics[scale=0.6]{figures/2016-09-23-model-checking/qqnorm-wnlm-Q-1.pdf}
\end{center}
\caption{
Checking the fit of wnlm.Q model: analysis of the normality of residuals.
}
\label{fig:qqnorm-wnlm.Q}
\end{figure}

\begin{figure}[h]
\begin{center}
\includegraphics[scale=0.6]{figures/2016-09-23-model-checking/homoscedas-wnlm-Q-1.pdf}
\end{center}
\caption{
Checking the fit of wnlm.Q model: analysis of homoscedasticity.
}
\label{fig:homoscedas-wnlm.Q}
\end{figure}

\begin{figure}[h]
\begin{center}
\includegraphics{figures/by-me/monoall-dependencies-2/obs-simple-general/obs-simple-general}
\hspace{\fill}
\includegraphics{figures/by-me/monoall-dependencies-2/obs-simple-general-gene-aspec/obs-simple-general-gene-aspec}
\hspace{\fill}
\includegraphics{figures/by-me/monoall-dependencies-2/obs-bayesian/obs-bayesian}
\end{center}
\caption{ General dependency structure of three regression model frameworks.
In all of threse model frameworks the regression coefficients
\(\beta_{1g},...,\beta_{3g}\) mediate, for a given gene \(g\), probabilistic
dependencies (arrows) between the response variable \(Y_g\) (read count ratio
for \(g\)) and the corresponding predictors \(X_1,...,X_3\).  For simplicity
but without loss of generality only 3 predictors are depicted.  The model
frameworks differ in how \(\beta_{jg_1},\beta_{jg_2},...\) relate to each
other for a given predictor (or a given \(j\)).  \emph{Left:} there is no
connection among \(\beta_{jg_1},\beta_{jg_2},...\) which means that the way
\(Y_{g}\), the read count ratio for gene \(g\) depends on predictor \(X_j\) is
completely separate from how the read count ratio for any other gene \(g'\)
(i.e.~\(Y_{g'}\)) depends on it.  Consequently no information may be shared
among gene-specific models.  \emph{Middle:} In this case
\(\beta_{jg_1}=\beta_{jg_2}=...\equiv\beta_j\) so that all genes are identical
with respect to how their read count ratio depends the predictors.  Thus genes
share all information in the data in the sense that the model forces them to
be identical.  \emph{Right:} Hierarchical Bayesian model where genes show both
variation as well as invariance with regards to depenencies.  The variation is
described by the dependence of \(\beta_j\) on the hyperparameter \(\gamma_j\),
whereas the invariance by the dependence of \(\gamma_j\) on \emph{its}
hyperparameter \(\xi_j\).  Only this model framework allows information
sharing among genes in a flexible way.  }
\label{fig:glm-vs-hierarch}
\end{figure}

\begin{figure}[h]
\begin{center}
\includegraphics[scale=0.6]{figures/2016-10-20-differential-expression-scz/venn-triple-1.pdf}
\end{center}
\caption{
Association of genes' expression to schizophrenia (SCZ) assayed by two RNA-seq
based approaches: total read count (overall expression, Nat Neurosci.~2016
Nov;19(11):1442-1453.)~and read count ratio
(allelic bias, present work).  When these approaches are compared for only
those genes that we find imprinted in the DLPFC in this study, 1 gene is found
associated to schizophrenia by both approaches, 1 by only overall expression,
and 3 by only allelic bias.
}
\label{fig:diff-exp-scz}
\end{figure}

%\begin{figure}[h]
%\includegraphics[width=0.6\textwidth]{figures/2016-10-11-comparison-to-mouse-cerebellum/posterior-pp-vs-pval-wnlm-Q-1.pdf}
%\caption{Comparison of the effects of age and gender between the present work and a
%previous study~\cite{Perez2015} in the mouse cerebellum. }
%\label{fig:mouse-cerebellum}
%\end{figure}

\end{document}
